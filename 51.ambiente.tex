% !TeX root = ./00.ppgcc-2020.tex

\section{Ambiente de Teste}\label{sec:ambiente}

Com o objetivo de avaliar esta proposta e averiguar os efeitos da distribuição
da detecção de novidades em um cenário \iot, construiu-se um ambiente
experimental de névoa (\fog).

Este ambiente é composto por três computadores de única placa (\emph{Single
Board Computer}) modelo Raspberry Pi 3 model B, equipados com o SoC
(\emph{system-on-chip} - sistema em um chip) de arquitetura ARM \emph{BCM2837} com
$4$ núcleos de processamento à frequência de $1.2GHz$, $32kB$ e $512kB$ de
memória \emph{cache} nível 1 e 2 respectivamente, $1GB$ de memória RAM,
armazenamento em cartão SD e conectados por rede cabeada \emph{Ethernet}.
% Switch
% Hélio:
%    Há algo estranho na descrição dos nós; parece que você está misturando nós e cores.
%    Talvez seja mais claro falar do número de nós, ao invés dos cores...
% ficou melhor!

A ideia central é criar um cluster simples simulando \emph{gateway} de uma rede
\iot com recursos limitados.
Este cluster armazenou todo o código-fonte, binários (compilados no mesmo cluster) e
\dataset.
Nesta configuração, o \dataset é armazenado no cartão SD do nó raiz e é lido para
cada execução do experimento.
Todos os experimentos foram executados neste cluster para isolamento de outras
variações imprevistas e para garantir que as comparações seriam justas com
\emph{software} e \emph{hardware} constante.

% The idea was to create a simple cluster simulating an \iot network with
% constrained resources at the edge of the network.
% This cluster stored all source code, binaries (compiled and linked in place) and
% data sets.
% In our setup, the data set is stored in the root's node SD card and is read for
% each experiment.
% All experiments were executed in this cluster for isolation of otherwise
% unforeseen variations and for safe software comparison with constant hardware.
% data sets, being accessed via our laboratory network over Secure Shell (SSH).

O \dataset \emph{Kyoto 2006+}\footnote{Disponível em
\url{http://www.takakura.com/Kyoto\_data/}}, tráfego das \emph{Honeypots} da
Universidade de Kyoto, é o foco deste trabalho.
Este \dataset contém dados ainda representativos (até 2015) e as característica
desejáveis de um conjunto de dados (realismo, validade, etiquetas previamente
definidas, alta variabilidade, reprodutibilidade e disponibilidade pública) são
atendidas \cite{KyotoDataset,Song2011kyoto}.

O segmento utilizado deste \dataset é o de Dezembro de 2015, contendo $7\:865\:245$ instâncias.
Deste segmento, são mantidas apenas instâncias associadas a tráfego normal ou
a ataques conhecidos, identificados por \nids que tenham mais de $10\:000$ instâncias
para significância, como feito previamente por \citeonline{Cassales2019a}.
% , this removes 46,390 instances. TODO: revisar, pois 7M != 700K.
As instâncias mantidas são normalizadas para que o valor de cada característica
original (como endereço IP, duração do fluxo, serviço) seja transposta para o
intervalo Real $[0, 1]$.

O conjunto resultante da operação recém descrita é então dividido em dois
conjuntos, treinamento e teste.
Para avaliar a detecção de ataques, o conjunto treinamento é composto apenas de
tráfego normal, contendo $72\:000$ instâncias.
O conjunto de teste possui $653\:457$ instâncias, sendo elas
$206\:278$ instâncias com classe ``$N$'' (normal) e
$447\:179$ instâncias com classe ``$A$'' (ataque).
%
% Não entendi essa questão de filtrar apenas os "normal" class no training set... Hélio
% acho que fez sentido... detectar o que não se aproxima de normal como novidade...
Destaca-se que essa manipulação do \dataset pode resultar em \emph{Overfitting} para a classe
normal e \emph{under-fitting} para a classe ataque, pois o sistema primeiro
precisa detectar um padrão novidade para então adicionar ao modelo. 
%O foco deste trabalho não é a otimização da detecção de intrusão é contexto de validar metodologia de construcao do \mfog - distribuição e paralelismo em ambiente em \fog.
Como o foco deste trabalho é na arquitetura e metodologia de paralelização e
distribuição em névoa, a otimização do processo de detecção de intrusão não foi
abordada, deixando a questão de escolha \dataset e do processo de manipulação
dele em aberto.

Quanto aos parâmetros do algoritmo \minas, ilustrados na Figura
\ref{fig:params}, os valores foram escolhidos por serem valores comuns para o
algoritmo presentes na literatura \cite{Faria2013Minas,Faria2016minas} e na
implementação de referência \refminas \cite{Faria2013source}.
Os parâmetros que não tem valores comuns na literatura foram escolhidos e ajustados até os
os resultados obtidos se aproximarem aos resultados da implementação de referência \refminas.

\begin{figure}[htb]
  \centering
  \begin{lstlisting}
    MinasParams minasParams = {
      .k=100, .dim=22, .precision=1.0e-08,
      .radiusF=0.25, .minExamplesPerCluster=20, .noveltyF=1.4,
      .thresholdForgettingPast = 10000,
    };
  \end{lstlisting}
  \caption{Parâmetros do algoritmo \minas.}
  \label{fig:params}
\end{figure}

Os parâmetros utilizados da literatura são
\texttt{k} número de \mclusters gerados pelo algoritmo de agrupamento,
\texttt{minExamplesPerCluster} número mínimo de exemplos para um \mcluster válido
e,
\texttt{thresholdForgettingPast} limite para remoção de exemplos do conjunto de
desconhecidos.

Os parâmetros escolhidos por aproximação de resultados são:
\texttt{precision} é o valor limite para melhora na distância global reduzindo as 
iterações no algoritmo de agrupamento (otimização),
\texttt{radiusF} fator que multiplica o desvio padrão das distâncias entre o
centro e cada exemplo formador do novo \mcluster definindo o raio do \mcluster
e,
\texttt{noveltyF} fator que multiplica o desvio padrão das distâncias do
\mcluster mais próximo distinguindo um novo padrão entre extensão e novidade.

% (k, minExamplesPerCluster, noveltyF, thresholdForgettingPast)
% (precision, radiusF)



% Para realização dos experimentos, diversas configurações de ambientes são
% propostas.
% Os ambientes selecionados são: local, 
% \notahl{o que muda na paralelização e na \textbf{distribuição} de instâncias
% do módulo \textbf{classificador}?}
% \hlhl{nuvem e névoa}.
% As configurações consistem na distribuição de módulos da implementação \mfog
% sendo executadas em combinações de ambientes nuvem e névoa com variada
% quantidade de nós.
  % dá-lhe Yoda! Hélio

% O ambiente local é composto por um único nó computacional, consistindo de um
% computador pessoal equipado com um processador de 8 núcleos, $16GB$ de memória e
% armazenamento em estado sólido (SSD) usado para o desenvolvimento e referência
% em comparações.
% O ambiente nuvem é provido pela utilização da infraestrutura de nuvem da
% Universidade Federal de São Carlos (Cloud{@}UFSCar\footnote{Disponível em
% \url{http://portalcloud.ufscar.br/servicos}}).
% O ambiente de névoa (\fog) é composto por computadores de única placa
% (\emph{Single Board Computer}) equipados com processador de arquitetura ARM de 4
% núcleos, $1GB$ de memória, armazenamento em cartão SD (\emph{SD-card}) e
% conectados por rede sem fio.

% A combinação de diferentes formas de distribuição dos nós tem por objetivo \hlhl{demonstrar padrões de
% latência} e qualidade que podem afetar implantações em ambientes reais que não
% são geralmente destacados quando os experimentos são realizados em um único
% nó ou ambiente.

% Faz parte também do ambiente de teste os conjuntos de dados (\datasets)
% \emph{KDD99}
% % \hlfa{e \emph{Kyoto 2006+}}
% e \emph{Kyoto 2006+}
% que foram selecionados por motivos distintos.

% O \dataset \emph{Kyoto 2006+} é o foco deste trabalho, pois contém dados ainda
% representativos (até 2015) e as característica desejáveis de um conjunto de
% dados (realismo, validade, etiquetas previamente definidas, alta variabilidade,
% reprodutibilidade e disponibilidade pública) são atendidas
% \cite{KyotoDataset,Song2011kyoto}.

% O \dataset \emph{KDD99} é amplamente utilizado em trabalhos de detecção de
% anomalia.
% Porém, como não possui mais a característica de realismo, uma vez que foi
% construído em 1998, neste trabalho o \dataset \emph{KDD99} é utilizado somente
% para que o leitor possa comparar com outros trabalhos
% \cite{Tavallaee2009,Protic2018KddKyoto}.

% Os dois \datasets mencionados e outros abordados em discussão e avaliados como
% relevantes são

% Por exemplo, o KDD original tem 41 atributos. A base é rotulada para 24
% tipos de ataques divididos em 4 grupos: DOS ...
% 
% O paper que analisou a fundo o KDD99 foi este: 
% 
% M. Tavallaee, E. Bagheri, W. Lu, and A. Ghorbani, “A detailed analysis
% of the KDD Cup 1999 data set,” in Proc. 2nd IEEE Symp. Comput. Intell.
% Secur. Defense Appl., 2009, pp. 1–6.
% 
% Qual é o defeito dele? tem muitos dados replicdos tanto na base de
% treinamento quento na de teste, o que gera viés nos resultados. Para
% resolver isso propuseram o NSL-KDD que é um subconjunto do KDD que evita
% esse problema

% \notake{
%   CICIDS2017 {https://www.unb.ca/cic/datasets/ids-2017.html} e
%   ISCXTor2016 {https://github.com/ahlashkari/CICFlowMeter}
% }
% \notafa{Uma sugestão seria usar datasets artificiais também a fim de avaliar
% outras caracteristicas tais como: nro de atributos, frequencia do surgimento de
% novidades, mudanças abruptas, graduais, etc. }
% \begin{table}[ht]
%   \caption{Sumário dos conjuntos de dados}
%   \centering
%   \begin{scriptsize}
%   \begin{tabularx}{\linewidth}{X|X|X|X}
%     Nome &
%       Origem &
%       Descrição &
%       Acesso Público \\
%     \hline
%     \hline
%     \emph{KDD99} \cite{Tavallaee2009,Protic2018KddKyoto} &
%       Captura de Fluxos de rede com ataques simulados &
%       41 atributos (sumário de fluxo), 23 classes, $4\;898\;431$ instâncias, $709$ MB &
%       \url{https://kdd.ics.uci.edu/databases/kddcup99/kddcup99.html} \\
%     \hline
%     \emph{Kyoto 2006+} \cite{Song2011kyoto,Protic2018KddKyoto}&
%       Captura de Fluxos de rede com HoneyPot &
%       23 atributos (sumário de fluxo), 3 classes, $7\;865\;245$ instâncias e $1.3$ GB (dez-2015) &
%       \url{https://www.takakura.com/Kyoto_data/new_data201704/} \\

%       %  0.000000 other 0 0 0 0.00 0.00 0.00 0 0 0.00 0.00 0.00 S0 00 0 -1
%       %  fdbd:f115:35b2:0424:40aa:098c:03e5:149b 3712
%       %  fdbd:f115:35b2:c891:7db9:2762:6182:03eb 445 00:00:00 tcp
    
%        \hline
%     % ISCXTor2016 \cite{Draper-Gil2016} &
%     %   Origem &
%     %   28 atributos,  &
%     %   \url{https://www.unb.ca/cic/datasets/tor.html} \\
%     % \hline
%     % ISCXVPN2016 \cite{Lashkari2017} &
%     %   Origem &
%     %   28 atributos,  &
%     %   \url{https://www.unb.ca/cic/datasets/tor.html} \\
%     % \hline
%     CICIDS2017 \cite{Sharafaldin2018cicids2017} &
%       Captura de Fluxos de rede com ataques simulados com perfil de trafego de 25
%       usuários normais e de 6 perfis de ataques durante 5 dias (1º dia sem ataque) &
%       80 atributos (sumário de fluxo extraído de CICFlowMeter), 15 classes,
%       $2\;830\;751$ instâncias e 1.2GB em arquivos \emph{pcap} e \emph{csv} &
%       \url{https://www.unb.ca/cic/datasets/ids-2017.html} \\
%     \hline
%     \emph{Radial Basis Function} (RBF) da biblioteca \emph{Massive Online Analysis} (MOA)
%     \emph{4CRE-V2} &
%       Sintético gerado por função RBF da biblioteca MOA com características de
%       mudança e evolução de conceito &
%       Atributos ($\mathbb{R}$), exemplos, classes, evoluções e mudanças configuráveis &
%       \url{https://sites.google.com/site/nonstationaryarchive/home} \\
%     \hline
%   \end{tabularx}
%   \label{tab:summary-dataset}
%   \end{scriptsize}
% \end{table}

