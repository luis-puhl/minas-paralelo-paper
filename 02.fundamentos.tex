\chapter{Fundamentos Científicos e Tecnológicos}

% cap 2: fundamentos científicos e tecnológicos
%     (pegue apenas os mais citados, siga a Elaine)
%     2.1. Computação em Nuvem, Fog e Edge
%     2.2. Plataformas de processamento distribuído
%         - arq labmda, kappa, (vide guilherme)
%         - MapReduce, Haddop, Spark, Storm
%     2.3. Apache Flink
%     2.4. Mineração de Dados
%     2.5. Mineração de Stream
%         - quem são, o que consomem
%         (BigFlow apud Gaber2005) Mining data streams: A Review.
%     2.6. Novelty Detection
%     2.7. O algoritmo Minas

\section{Computação em Nuvem, Borda e Névoa}

Para descrever os modelos de computação em nuvem, borda e névoa, é necessário
abordar o conceito de distância e densidade em rede. Distância pode ser definida
como número de saltos (\emph{hops}), latência, distância geográfica ou combinação destas.
Densidade é extraída da distância, projetando a mesma num hiper-espaço de maneira que os
nós com menor distância entre si fiquem mais próximos. Então quando existe um
grande número de nós numa mesma região diz-se que ela é densa, quando há poucos
nós em uma região, esparsa. Acredita-se que data centers, backbones e nuvens publicas
formem uma concentração de nós e quanto mais próximo do usuários finais (folhas)
mais esparso é esse hiper-espaço.

% Definir internet e rede (borda, centro, etc)
Classificando a internet por sua densidade podemos dizer que ao centro estão
os \emph{data centers} e nuvens públicas em seguida o núcleo interconectando redes diversas,
redes locais e a Borda composta pelos nós folha dentro de uma rede local.

O modelo de Computação em Nuvem (\emph{Cloud Computing})
permite alocar recursos como redes, servidores, armazenamento, aplicações e serviços
de maneira conveniente e seu provisionamento ágil concede elasticidade para atender
demandas variáveis com custo mínimo \cite{NIST2011}.

Alternativamente, a Computação na Borda (\emph{Edge Computing}) destaca-se no
processamento em tempo real de dados originários da própria borda além de atender
preocupações de segurança e privacidade \cite{Shi2016}.

\emph{Edge} é por vezes chamado de Computação em Névoa (\emph{Fog Computing})
contudo \citeonline{IEEECommunicationsSociety2018} indica diferenças como exclusão do
modelo \emph{cloud} e limitação a poucas camadas (por exemplo, somente os nós folha de uma rede)
no modelo \emph{edge} em direto contraste com a inclusão de \emph{cloud} e hierarquias
maiores. \emph{Fog} também atende gerenciamento da rede, armazenamento e controle.

% Fog works with the cloud, whereas edge is defined by the exclusion of cloud. Fog
% is hierarchical, where edge tends to be limited to a small number of layers. In
% additional to computation, fog also addresses networking, storage, control and
% acceleration. \cite{IEEECommunicationsSociety2018}

Esse modelo de computação distribuída desde os nós folha até o centro é motivado
pela mudança do \emph{statu quo} do fluxo dos dados na internet: tradicionalmente
os dados são produzidos pelos dispositivos de borda imediatamente enviados à 
\emph{cloud} (produção, \emph{upstream}),
que armazena e processa recursos derivados servido-os através de requisição-resposta
(consumo, \emph{downstream}) a mais clientes.
Com a ampliação da Internet das Coisas (\emph{Internet of Things}, IoT) e consequente
ampliação sem precedentes do volume de dados gerados, mudando o relação de consumo
e produção \cite{Shi2016}, arquiteturas tradicionais como \emph{cloud} podem não
ser capazes de lidar com esses dados por falta de banda o que leva as
propostas de distribuição vertical do processamento em \emph{fog} \cite{Bonomi2012, Dastjerdi2016}.

% Previous work such as micro datacen- ter [12], [13], cloudlet [14], and fog computing [15]
% [15] F. Bonomi, R. Milito, J. Zhu, and S. Addepalli,
% “Fog computing and its role in the Internet of things,” 
% in Proc. 1st Edition MCC Workshop Mobile Cloud Comput., Helsinki, Finland, 2012, pp. 13–16.

% Fog Computing: Helping the Internet of Things Realize Its Potential
% \cite{Dastjerdi2016}
% The Internet of Things (IoT) could enable
% innovations that enhance the quality of life, but it
% generates unprecedented amounts of data that
% precedented amounts of data that can be useful in many ways, par-
% are difficult for traditional systems, the cloud, and
% even edge computing to handle. Fog computing
% is designed to overcome these limitations.

\section{Mineração de Dados e Fluxo de Dados}

% Faria nem Silva definem ou citam data mining
% A data stream (DS) is a sequence of examples that arrive continuously. They are
% continuous, unbounded, flow at high speed and have a data distribution that
% may change over time \cite{Silva2013}. In DS scenarios, new concepts may
% appear and known concepts may disappear or evolve. \cite{Faria2016nd}

Mineração de Dados é o processo de descoberta de padrões em grandes conjuntos de
dados utilizando métodos derivados de aprendizagem de maquina, estatística e 
banco de dados. Uma caso de mineração de dados é \emph{Big Data} onde o conjunto
de dados não pode ser processado em um tempo relevante por \emph{hardware} e \emph{software}
comum, geralmente coincidente com o limite de armazenado na memória ou
armazenamento principal.

% Data stream mining is concerned with the extraction of knowledge from large
% amounts of continuously generated data in a non-stationary environment. Novelty
% detection (ND), the ability to identify new or unknown situations not
% experienced before, is an important task for learning systems, especially when
% data are acquired incrementally (Perner 2008). In data streams (DSs), where new
% concept can appear, disappear or evolve over time, this is an important issue to
% be addressed. ND in DSs makes it possible to recognize the novel concepts, which
% may indicate the appearance of a new concept, a change in known concepts or the
% presence of noise (Gama 2010).
% 
% Perner P (2008) Concepts for novelty detection and handling based on a case-based
% reasoning process scheme. Eng Appl Artif Intell 22:86–91
% 
% Gama J (2010) Knowledge discovery from data streams, vol 1, 1st edn. CRC press chapman hall, Atlanta

% dados massivamente e continuamente gerados e não persistentes. Fp

Além da dimensão de armazenamento outra dimensão que afeta a maneira como dados
são modelados e manipulados é o tempo. Um Fluxo de Dados (\emph{Data Stream}) é
uma sequência de registros produzidos a uma taxa muito alta, associadas ao tempo
real, ilimitados, que excede recursos de armazenamento e, portanto, pode ser
lida apenas uma vez durante processamento \cite{Gama2007}.

% 
% A data stream is a sequence of unbounded, real-time data records that are
% characterized by the very high data rate, which stresses our computational
% resources, and can be read only once by processing applications [13,8,1,9]
 
% [1] B. Babcock, S. Babu, M. Datar, R. Motwani, J. Widom, Models and issues in
% data stream systems. In: Proceedings of Principles of Database Systems
% (PODS’02), pp. 1–16, 2002. \cite{Babcock2002}

% [2] G. Boone, Reality mining: browsing reality with sensor networks.
% In: Sensors Online, vol. 21, 2004.

% [3] V. Cantoni, L. Lombardi, P. Lombardi, Challenges for data mining in
% distributed sensor networks. ICPR (1) 1000–1007, 2006

% [8] M.M. Gaber, A. Zaslavsky, S. Krishnaswamy, Mining data streams: a review.
% ACMSIGMOD Record, 34(2):18–26, 2005.
% 
% [9] M. Garofalakis, J. Gehrke, R. Rastogi, Querying and mining data streams:
% you only get one look a tutorial. In: Proceedings of the 2002 ACM SIGMOD
% International Conference on Management of Data, June 03–06, Madison, Wisconsin,
% 2002.
% 
% [13] S. Muthukrishnan, Data streams: algorithms and applications. In:
% Proceedings of the Four- teenth Annual ACM-SIAM Symposium on Discrete
% Algorithms, 2003.

Mineração de Fluxo de Dados é análogo à mineração de
dados e \emph{big data} com a restrição temporal onde um registro é unicamente 
associado um tempo, dessa forma além de não ser possível manipular o conjunto
de dados em memória, não é possível recuperar dados fora do intervalo de tempo associado
a eles.

% ----- wikipedia ------
% 
% Data mining is the process of discovering patterns in large data sets involving
% methods at the intersection of machine learning, statistics, and database
% systems.[1] Data mining is an interdisciplinary subfield of computer science and
% statistics with an overall goal to extract information (with intelligent
% methods) from a data set and transform the information into a comprehensible
% structure for further use.[1][2][3][4] Data mining is the analysis step of the
% "knowledge discovery in databases" process or KDD.[5] Aside from the raw
% analysis step, it also involves database and data management aspects, data
% pre-processing, model and inference considerations, interestingness metrics,
% complexity considerations, post-processing of discovered structures,
% visualization, and online updating.[1]
% 
% [1] "Data Mining Curriculum". ACM SIGKDD. 2006-04-30. Retrieved 2014-01-27.
% [2] Clifton, Christopher (2010). "Encyclopædia Britannica: Definition of Data Mining". Retrieved 2010-12-09.
% [3] Hastie, Trevor; Tibshirani, Robert; Friedman, Jerome (2009).
% "The Elements of Statistical Learning: Data Mining, Inference, and Prediction".
% Archived from the original on 2009-11-10. Retrieved 2012-08-07.
% [4] Han, Kamber, Pei, Jaiwei, Micheline, Jian (June 9, 2011).
% Data Mining: Concepts and Techniques (3rd ed.). Morgan Kaufmann. ISBN 978-0-12-381479-1.
% [5] Fayyad, Usama; Piatetsky-Shapiro, Gregory; Smyth, Padhraic (1996).
% "From Data Mining to Knowledge Discovery in Databases" (PDF). Retrieved 17 December 2008.

- quem são, o que consomem
(BigFlow apud Gaber2005) Mining data streams: A Review.

\section{Plataformas de processamento distribuído}

- arq labmda, kappa, (vide guilherme)
- MapReduce, Haddop, Spark, Storm

% Ferramentas de BigData}

O que são e para que servem essas ferramentas 

Breve descrição do MapReduce e Hadoop

Breve descrição do Spark

% Apache Spark}

% Spark is an open-source cluster computing framework with a large global user base.
% It is written in Scala, Java, R, and Python and gives programmers an Application Programming Interface (API)
% built on a fault tolerant, read-only multiset of distributed data items.
% In two years since its initial release (May 2014), it has seen wide acceptability for real-time,
% in-memory, advanced analytics — owing to its speed, ease of use, and the ability to handle
%  sophisticated analytical requirement
% https://dzone.com/articles/streaming-in-spark-flink-and-kafka-1

%  \input{02_1.spark.tex

% # Apache Spark e pyspark

% Apache Spark is an open-source distributed general-purpose cluster-computing
% framework. Spark provides an interface for programming entire clusters with
% implicit data parallelism and fault tolerance. Originally developed at the
% University of California, Berkeley's AMPLab, the Spark codebase was later
% donated to the Apache Software Foundation, which has maintained it since.

Apache Spark é um framework de código fonte aberto para computação
distribuída.[1] Foi desenvolvido no AMPLab da Universidade da Califórnia[2] e
posteriormente repassado para a Apache Software Foundation[3] que o mantém desde
então. Spark provê uma interface para programação de clusters com paralelismo e
tolerância a falhas.

% Apache Spark is a fast and general-purpose cluster computing system. It provides
% high-level APIs in Java, Scala, Python and R, and an optimized engine that
% supports general execution graphs. It also supports a rich set of higher-level
% tools including Spark SQL for SQL and structured data processing, MLlib for
% machine learning, GraphX for graph processing, and Spark Streaming.

-----------------------

Apache Spark \ cite{Zaharia} é um \emph{framework} para construção de sistemas
de computação distribuída em \emph{cluster} com garantias de tolerância a falhas
(execução em computadores não confiáveis) utilizando como premissas: paralelização
e localidade de dados, como 

api em Python (dataframe de pandas)
% ------------------------------------------------------------------------------------------------------


\section{Apache Flink}

Breve descrição do Flink (como esse vai ser usado, precisa explicar um pouco melhor - 2 paginas pelo menos):\\
- arquitetura\\
- modelo de programacao\\
- 1 pequeno exemplo de codigo explicando

\cite{Lopez2018}

% [80] ANDREONI LOPEZ, M., LOBATO, A. G. P., MATTOS, D. M. F., et al. “Um
% Algoritmo Não Supervisionado e Rápido para Seleção de Caracterı́sticas
% em Classificação de Tráfego”. In: XXXV SBRC’2017, Belém- Pará, PA,,
% 2017.
% [81] ANDREONI LOPEZ, M., LOBATO, A. G. P., DUARTE, O. C. M. B., et al.
% “An evaluation of a virtual network function for real-time threat detection
% using stream processing”. In: IEEE Fourth International Conference on
% Mobile and Secure Services (MobiSecServ), pp. 1–5, 2018. doi: 10.1109/
% MOBISECSERV.2018.8311440.
% [82] CHENG, Z., CAVERLEE, J., LEE, K. “You Are Where You Tweet: A Content-
% based Approach to Geo-locating Twitter Users”. In: Proceedings of the
% 19th ACM International Conference on Information and Knowledge Man-
% agement, CIKM ’10, pp. 759–768. ACM, 2010. ISBN: 978-1-4503-0099-5.
% [83] LOBATO, A. G. P., ANDREONI LOPEZ, M., DUARTE, O. C. M. B. “Um
% Sistema Acurado de Detecção de Ameaças em Tempo Real por Processa-
% mento de Fluxos”. In: SBRC’2016, pp. 572–585, Salvador, Bahia, 2016.


% \{Messa queueing systems - Sistemas de filas de mensagens}
% O que são, para que servem e citar alguns exemplos
% \{Kafka}
% 1 pag

% Apache Flink % Apache Flink is an open-source platform for distributed stream
% and batch data processing. Flink’s core is a streaming data flow engine that
% provides data distribution, communication, and fault tolerance for distributed
% computations over data streams. Flink also builds batch processing on top of the
% streaming engine, overlaying native iteration support, managed memory, and
% program optimization.

% Advantages of Flink:

% Flink streaming processes data streams as true streams, i.e. data elements are
% immediately “pipelined” through a streaming program as soon as they arrive. This
% allows performing flexible window operations on streams.

% Better memory management: Explicit memory management gets rid of the occasional spikes found in Spark framework.

% Speed: It manages faster speeds by allowing iterative processing to take place
% on the same node rather than having the cluster run them independently. Its
% performance can be further tuned by tweaking it to re-process only that part of
% data that has changed rather than the entire set. It offers up to a five-fold
% boost in speed when compared to the standard processing algorithm.

% Apache Spark is considered a replacement for the batch-oriented Hadoop system.
% But it includes a component called Apache Spark Streaming, as well. Contrast
% this with Apache Flink, which is a Big Data processing tool and it is known to
% process big data quickly with low data latency and high fault tolerance on
% distributed systems on a large scale. Its defining feature is its ability to
% process streaming data in real time.

\section{Detecção de Novidade}

Novelty Detection

breve descrição do que sao algoritmos para DN

ver se tem algum survey e citar

\section{O algoritmo MINAS}

breve descrição do MINAS \cite{deFaria2016}

ver paper da profa. Elaine

% discussão de 2020-02-01
Detecção de intrusão em redes
    - riscos de segurança
    % pontos de coleta de dados b´asicos para a maioria das estruturas de IoT sao Wireless Sensor Networks (WSN) e WSN baseadas em IP,
    % as quais s˜ao vulner´aveis e geram uma ameac¸a de seguranc¸a de alto n´ıvel (ADAT; GUPTA, 2018) Adat2018
    % (KASINATHAN et al., 2013), a detecc¸ ˜ao de assinaturas
    % (RAZA; WALLGREN; VOIGT, 2013; SHEIKHAN; BOSTANI, 2016) s˜ao propostos IDSs h´ıbridos com foco 
                % espec´ıfico em ataques de roteamento como sink-hole e redireciona- mento seletivo
    - técnicas de intrusão e tipos de ataques
    - mecanismo de detecção (análise de fluxo de rede -> detecção de anomalia)
    % Guilherme: A tarefa de detecção de intrusão consiste em descobrir, 
    %           determinar e identificar a utilização, duplicação, alteração ou destruição
    %           não autorizada de sistemas de informação (MUKKAMALA; SUNG; ABRAHAM, 2005) Mukkamala2005
    % deteccao por assinaturas (tambem chamada de misuse-detection), deteccao comportamental (tambem chamada de anomaly-detection)e deteccao hıbrida.(MODI et al., 2013).
    % implementacao usualmente ´e feita por meio de t´ecnicas de AM e MD (BUCZAK; GUVEN, 2016).
    % existem poucos trabalhos [..] online e deteccao de novidade ao problema [..] observado nas surveys (BUCZAK; GUVEN, 2016; MITCHELL; CHEN, 2014; MODI et al., 2013).
    % (FURQUIM et al., 2018), os autores implementam uma arquitetura de 3 camadas (WSN, Fog e Cloud)
    % (MIDI et al., 2017), os autores propuseram um IDS h´ıbrido [...] ativa apenas os m´odulos [...] especializado em um ataque espec´ıfic
    % (FAISAL et al., 2015)externo ou interno ao Smart Meter.dados do KDD99 [...] precis˜ao, Kappa, consumo de mem´oria, tempo e FAR
    % extensivamente em (Sommer; Paxson, 2010) e (MCHUGH, 2000), ´e dif´ıcil encontrar boas me- didas de avaliac¸ ˜ao para IDSs
    % (GAMA, 2010) afirma que no contexto de processamento de fluxo, as medidas tradicionais s˜ao impreci- sas.
Detecção de novidades
    % (PERNER, 2007)(GAMA, 2010). A
    - técnicas de Detecção de novidades
    - MINAS (incluir métricas) 
    % (FARIA et al., 2016)
    % ECSMiner (MASUD et al., 2011)
    % AnyNovel (ABDALLAH et al., 2016) s˜ao
    % medidas de Qualidade da Deteccao utilizadas foram Fnew, Mnew, Erro e a quantidade de exemplos rotulados por especialistas que cada t´ecnica requisitou (MASUD et al., 2011)
    - BigFlow (incluir métricas)
Processamento de Streams (big data)
    - cloud?
    % A arquitetura Lambda (MARZ; WARREN, 2015) de duas camadas de CPU (stream e batch) e camada de serviço
    % Kappa (KREPS, 2014) possui apenas um m´odulo de processamento on-line (apenas as camadas de processamento e servic¸o)
    - redes como stream
    - Atraso
    - Kafka/Spark/Flink
Redes IoT
    - Restrição hardware (Energia, CPU, Mem, Rede)
    - Consideração FOG vs Cloud
%/discussão
