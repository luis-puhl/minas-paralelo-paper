\chapter{Considerações Finais}\label{cha:final}

Este trabalho reúne conceitos de aprendizado de máquina com enfasê em detecção
de novidades em fluxos contínuos de dados e conceitos de processamento
distribuído de fluxos contínuos com o objetivo de unir a lacuna no estado da
arte desses conceitos à luz de uma implementação e avaliação no cenário de
fluxos de dispositivos da Internet das Coisas (\iot) em ambiente de computação
em névoa (\fog).

\nota{deixar claro que é futuro, voz singular, 3ª pessoa \\ metodologia do restante}

Na sequência deste trabalho continuamos desenvolvimento da implementação objeto
desse trabalho, \mfog, e contínua avaliação comparativa dos resultados
produzidos por ela com seu algoritmo base, MINAS.
Também continuamos os experimentos com os conjuntos de dados (\datasets)
diversos e configurações de distribuição de processamento em \fog variadas
extraíndo desses experimentos as métricas previamente discutidas.

% As métricas extraídas servirão, além de resultado final para este trabalho,
% para a construção de artigos usados para compra

Dessa forma contribuímos com uma adição de uma ferramenta para os interessados
em construir sistemas que dependem de detecção de novidades para seu bom
funcionamento, especialmente os baseados em dispositivos \iot e/ou com fluxos
contínuos que tradicionalmente sofrem com os ônus de latência e largura de banda
na comunicação dos resultados da borda para a nuvem onde são normalmente
executadas as técnicas de mineração usuais.


\section{Cronograma}\label{sec:crono}
% cap 5: Cronograma até a defesa
% (quando)

Nesta Seção apresentam-se as etapas previstas e sua distribuição temporal até o
final deste trabalho de pesquisa.

% \begin{itemize}
%   \item Enumerar métricas de qualidade de classificação e métricas de
%     escalabilidade encontradas na literatura;
%   \item Avaliar plataformas de processamento distribuído de fluxos como:
%     \begin{itemize}
%         \item \emph{Apache Kafka} com \emph{Python};
%         \item \emph{Apache Kafka Streams};
%         \item \emph{Apache Spark Streaming};
%         \item \emph{Apache Storm};
%         \item \emph{Apache Flink};
%     \end{itemize}
%   \item Implementar algoritmo MINAS sobre \emph{Apache Flink};
%   \item Executar a implementação com \emph{data sets} públicos relevantes;
%   \item Validar, por meio de comparação com o algoritmo original, se a
%     implementação gera os mesmos resultados quanto a classificação;
%   \item Extrair métricas de escalabilidade por meio de experimentação;
%   \item Avaliar estratégias de distribuição do algoritmo em \emph{fog} como:
%   \begin{itemize}
%     \item Detecção de novidade nas pontas;
%     \item Detecção de novidade na nuvem;
%     \item Detecção de novidade nas pontas e em nuvem;
%   \end{itemize}
%     e seu o impacto na qualidade de classificação e volume computado.
% \end{itemize}

\begin{enumerate}[label=\Alph*)]
  \item \label{task:Z} Exame de Qualificação;
  \item \label{task:A} Desenvolvimento da aplicação;
  \item \label{task:B} Validação da aplicação em contraste com a implementação
  MINAS original:
    \begin{itemize}
      \item preparação e, se necessário, adaptação da implementação
      original e \emph{data sets};
      \item comparação e, se necessário, ajustes à implementação.
    \end{itemize}
  \item \label{task:C} Experimentos com \emph{data sets} e estratégias de 
  distribuição em \emph{fog};
  \item \label{task:D} Submissão de artigos com resultados de (\ref{task:C}
  \item \label{task:E} Defesa da Disertação.
  % \item \label{task:E} Comparação com outros trabalhos, envolvendo:
  %   \begin{itemize}
  %     \item preparação e, se necessário, adaptação dos programas e \emph{data sets};
  %     \item comparação e publicação dos resultados.
  %   \end{itemize}
\end{enumerate}

\noindent\begin{ganttchart}[
  vgrid,
  time slot format=isodate-yearmonth,
  time slot unit=month,
  expand chart=\textwidth,
  inline,
  title height = 1,
  y unit title = 0.6cm,
  y unit chart = 0.7cm,
  bar height = .8,
  bar left shift=.05,
  bar right shift=-.05,
  bar/.style={fill=blue!55, rounded corners=3pt}
]{2020-03}{2020-07}
  \gantttitlecalendar{year, month} \\
  % \ganttmilestone{(\ref{task:Z}}{2020-03-02} \\
  \ganttbar{(\ref{task:Z}}{2020-03}{2020-03} \\
  \ganttbar{(\ref{task:A}}{2020-03}{2020-03} \\
  \ganttbar{(\ref{task:B}}{2020-03}{2020-04} \\
  \ganttbar{(\ref{task:C}}{2020-03}{2020-05} \\
  \ganttbar{(\ref{task:D}}{2020-04}{2020-07} \\
  \ganttbar{(\ref{task:E}}{2020-06}{2020-07} \\
\end{ganttchart}
