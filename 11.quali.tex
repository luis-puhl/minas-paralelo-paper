
\clearpage
{\Huge\textbf{Notas da Banca de Qualificação}}

\begin{itemize}
  \item Acentuação e frases longas;
  \item Erro de terminologia de ML:
    \begin{itemize}
      \item ``Mudança de Conceito'' como padrão;
      \item Separação de ``classificação'', ``detecção de novidade'' e ``detecção de anomalia'' (referenciar definição);
      \item Referências ao mesmo artigo no cap 2 (provável ser o livro do Gama);
      \item Definir com referência: ``stream'', ``data stream'', ``modelo'' ...
      \item ``Big Data Streams'';
    \end{itemize}
  \item Expansão da motivação, observar trabalhos de paralelização;
  \item Justificativa clara da distribuição e paralelismo das fases do MINAS:
    \begin{itemize}
      \item Motivar e justificar paralelismo;
      \item Identificar oportunidades de paralelismo;
      \item Clarificar a distribuição e escolha do que paralelizar;
      \item MINAS não armazena centro e raio, mas calcula à partir de $(ls, ss)$
      (soma linear e quadrada)
      \\ \textit{Isso abre oportunidade para uma função de redução (merge) de
      clusters}
    \end{itemize}
  \item Figuras com licença explicita de uso ou reconstruídas;
  \item Ver Albert Befit para processamento de Data Streams;
  \item Definir ``Conjunto mestre'' (master data set) ou ``fonte de verdade'';
  \item Apache Flink:
  \begin{itemize}
    \item ``Data Flow'' na explicação de Apache Flink;
    \item Justificar Apache Flink;
    \item Clarificar o TCP (source);
  \end{itemize}
  \item Revisar estrutura narrativa do chap. fundamentos;
  \item Abordar nos relacionados trabalhos sobre clustering e classificação
  distribuída, como SAMOA por Befit e X por Naldi;
  \item Remover derivados do minas dos relacionados;
  \item Aprofundar a distinção escrita entre o Casales e este trabalho;
  \item Detalhamento da comunicação entre os módulos;
  \item Definição e detalhamento da matriz de confusão;
  \item Sugestão de data sets sintéticos que realcem aspectos detecção e mudança
  (estressar algoritmo);
  \item [diverge] Prazo parece apertado para conclusões;
  \item Tabela sumarizando os algoritmos e técnicas tratados nos trabalhos relacionados;
  \item Data sets para detecção de intrusão:
    \begin{itemize}
      \item x 2014
      \item y 2016
    \end{itemize}
  \item Teste (métrica) de wincoxon eucutson(?);
  \item Mais referências para os trabalhos relacionados;
  \item Definição de ``Online'' vs ``Real Time'';
  \item Distinção mais clara entre ``modelo'' e ``algoritmo'';
  \item Adicionar métricas ao chap. fundamentos;
  \item Helio:
  \begin{itemize}
    \item Fonte de dados;
    \item Ação tomada após detecção (nova etiqueta ou novidade);
    \item Distribuição regional do modelo (efeitos de partições de rede);
    \item Arquitetura física;
    \item Ian Foster aborda processamento e aglomeração;
    \item Desambiguação de Fog e Edge;
  \end{itemize}
\end{itemize}

\begin{align}
  \mathtt{take \ } & (n,\ \mathbf{ls},\ \mathbf{ss})_{local}\textnormal{ \ and  \ }(n,\ \mathbf{ls},\ \mathbf{ss})_{remote}\\
  n &= \mathit{max}(n_l,\ n_r) \\
  \mathbf{ls} &= \frac{1}{2} (\mathbf{ls}_l + \mathbf{ls}_r) \\
  \mathbf{ss} &= \frac{1}{2} (\mathbf{ss}_l + \mathbf{ss}_r)
\end{align}

