% !TeX root = ./00.main.tex
\begin{resumo}

    Em um cenário de crescente número de dispositivos na Internet das Coisas
    (IoT) gerando proporcional crescimento no volume dos fluxos de dados
    gerados, são necessários métodos robustos para a mineração de fluxos
    contínuos de dados.
    % 
    Uma das áreas afetadas pelo crescimento vertiginoso do número de
    dispositivos e os fluxos associados a eles é a área de segurança da
    informação, onde são necessárias ferramentas de detecção de intrusão em
    redes que operem em ambientes de computação em névoa.

    \notahelio{isso é necessidade, ou é algo que pode ser interessante, por
    tratar os dados próximos da origem?}
    
    % 
    
    Estas ferramentas utilizam extensivamente algoritmos de detecção de novidade
    em fluxos de dados para identificar padrões no tráfego da rede
    \notahelio{paragrafo muito longo}
    Porém
    os algoritmos que tratam adequadamente dos desafios de detecção de novidade
    em fluxos de dados, como deriva
    e evolução de conceito e atualização contínua do modelo de classificação sem
    interferência de especialistas, ainda são pouco utilizados.
    % 

    \notahelio{Talvez dizer que algumas de suas características indicam que a
    implementação pode beneficiar-se de paralelismo...}

    O algoritmo de detecção de novidade em fluxo de dados MINAS tem recebido
    atenção de pesquisas recentes por tratar desses desafios de detecção de novidade
    em fluxos de dados, porém este
    algoritmo ainda não foi adaptado para processar grandes volumes de fluxos
    reais em ambiente de computação em névoa.
    % 
    O presente trabalho aborda essa lacuna propondo um sistema
    que implementa o algoritmo MINAS de maneira distribuída num contexto
    de detecção de intrusão e computação em névoa.
    % 
    % Avaliações mostram que este sistema pode reduzir a latência de detecção e
    % reduzir consumo de banda na comunicação entre a borda e a nuvem em relação
    % a sistemas que utilizam somente a nuvem para detecção de intrusão.
    Experimentos mostram que o algoritmo MINAS pode ser paralelizado e
    distribuído utilizando plataformas de processamento de fluxos como
    \emph{Apache Flink}.

\palavraschave{Detecção de Novidades, Detecção de Intrusão, Fluxos de Dados,
Computação Distribuída, Computação em Névoa, Internet das Coisas.}

\end{resumo}

\begin{abstract}
    
    In a scenario of growing number of devices connected to the Internet of Things (IoT)
    with proportional growth in the volume of data streams generated, robust
    methods are needed for mining streams continuous data.
    % 
    One of the areas affected by the huge growth in the number of devices
    and the streams associated with them is the information security, which needs
    network intrusion detection tools that operate
    in fog computing environments.
    % 
    These tools make extensive use of algorithms for novelty detection in data
    streams to identify treat patterns in network traffic, however algorithms
    that adequately address the challenges of novelty detection in data streams
    such as concept drift, concept evolution and continuous update of the
    classification model without expert interference are widely used.
    % 
    The MINAS algorithm addresses those novelty detection in data streams challenges and has received
    recent research attention, but it has not yet been adapted to process
    large volumes of real streams or to operate in a fog computing environment.
    % 
    The present work proposes a system that implements the MINAS algorithm
    in a distributed fog environment in the context of intrusion detection
    to addresses this gap.
    % 
    Preliminary work shows that it is possible to have a distributed
    version of the MINAS algorithm by using stream processing platforms
    such as Apache Flink.
    % this system can reduce detection latency and
    % reduce bandwidth consumption in the communication between the edge and the
    % cloud in relation to systems that use only the cloud to perform analysis.
    % % 
    % Experimentos mostram que o algoritmo MINAS pode ser paralelizado e
    % distribuído utilizando plataformas de processamento de fluxos como
    % \emph{Apache Flink}.

\keywords{Novelty Detection, Intrusion Detection, Data
Streams, Distributed Computing, Fog Computing, IoT devices}

\end{abstract}