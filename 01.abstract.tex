% !TeX root = ./00.main.tex

% \@ifundefined{resumo}{%
%     \newenvironment{resumo}%
%   {\ttfamily}%
%   {}
%     % {\begin{abstract}
%     % {\end{abstract}
% }{}

% % ajusta o espaçamento dos parágrafos do resumo}
% \@ifundefined{absparsep}{\setlength{\absparsep}{18pt}}

% \newcommand{\resumoname}{Resumo}
% \newenvironment{resumo}[1][Resumo]{%
%     \PRIVATEbookmarkthis{#1}
%     \renewcommand{\abstractnamefont}{\chaptitlefont}
%     \renewcommand{\abstractname}{\ABNTEXchapterupperifneeded{#1}}
%     \begin{abstract}
% }{\end{abstract}\PRIVATEclearpageifneeded}
% \newenvironment{resumo}[1]{\renewcommand{\abstractname}{#1}\begin{abstract}}{\end{abstract}}

\begin{abstract}

    Em um cenário de crescente número de dispositivos na Internet das Coisas
    (IoT), gerando proporcional crescimento no volume dos fluxos de dados
    gerados, são necessários métodos robustos para a mineração de fluxos
    contínuos de dados.
    %%
    Uma das áreas afetadas pelo crescimento vertiginoso do número de
    % \notahl{longo}
    dispositivos e os fluxos associados a eles é a área de segurança da
    informação, onde são necessárias ferramentas de detecção de intrusão em
    redes que operem em ambientes de computação em névoa, devido aos custos de
    comunicação associados a operar estas ferramentas 
    % \notahl{?}
    % \hlhl{
        somente em ambiente de nuvem
    % }
    .
    %%
    As ferramentas de detecção de intrusão utilizam extensivamente algoritmos de
    detecção de novidade em fluxos de dados para identificar padrões no tráfego
    da rede.
    %%
    Porém, os algoritmos que tratam adequadamente dos desafios de detecção de
    novidade em fluxos de dados, como mudança e evolução de conceito e
    atualização contínua do modelo de classificação sem interferência de
    especialistas, ainda são pouco utilizados.
    %%
    O algoritmo de detecção de novidade em fluxo de dados MINAS tem recebido
    atenção de pesquisas recentes por tratar desses desafios de detecção de novidade
    em fluxos de dados.
    %%
    No entanto, apesar de sua divisão em três partes semi-independentes, este
    algoritmo ainda não foi adaptado para processar grandes volumes de fluxos
    reais em ambiente de computação em névoa.
    % \notahl{a questão de processamento em nuvem vs névoa não está clara}
    % \nota{também a parte de paralelismo}
    %%
    O presente trabalho aborda essa lacuna, propondo um sistema
    que implementa o algoritmo MINAS de maneira distribuída num contexto
    de detecção de intrusão e computação em névoa.
    %%
    % Avaliações mostram que este sistema pode reduzir a latência de detecção e
    % reduzir consumo de banda na comunicação entre a borda e a nuvem em relação
    % a sistemas que utilizam somente a nuvem para detecção de intrusão.
    Experimentos mostram que o algoritmo MINAS pode ser paralelizado e
    distribuído utilizando plataformas de processamento de fluxos como
    \emph{Apache Flink}.

\palavraschave{Detecção de Novidades, Detecção de Intrusão, Fluxos de Dados,
Computação Distribuída, Computação em Névoa, Internet das Coisas.}

\end{abstract}

\renewcommand{\abstractname}{Abstract}
\begin{abstract}
    
    In a scenario of growing number of devices connected to the Internet of Things (IoT)
    with proportional growth in the volume of data streams generated, robust
    methods are needed for mining streams continuous data.
    %%
    One of the areas affected by the huge growth in the number of devices
    and the streams associated with them is the information security, which needs
    network intrusion detection tools that operate
    % \notahl{?}\hlhl{}
    in fog computing environments due to the cost of operating such tools
    in a cloud only environment.
    %%
    These tools make extensive use of algorithms for novelty detection in data
    streams to identify treat patterns in network traffic.
    However, algorithms in wide use do not
    adequately address the challenges of novelty detection in data streams,
    such as concept drift, concept evolution and continuous update of the
    classification model, without expert interference.
    %%
    % \notahl{?}\hlhl{}
    The MINAS algorithm addresses those novelty detection in data streams
    challenges and has received recent research attention.
    %%
    However, despite its division in three semi-independent parts, MINAS has
    not yet been adapted to process large volumes of real streams or to operate
    in a fog computing environment.
    %%
    The present work proposes a system that implements the MINAS algorithm
    in a distributed fog environment in the context of intrusion detection
    to addresses this gap.
    %%
    Preliminary work shows that it is possible to have a distributed
    version of the MINAS algorithm by using stream processing platforms
    such as Apache Flink.
    % this system can reduce detection latency and
    % reduce bandwidth consumption in the communication between the edge and the
    % cloud in relation to systems that use only the cloud to perform analysis.
    % Experimentos mostram que o algoritmo MINAS pode ser paralelizado e
    % distribuído utilizando plataformas de processamento de fluxos como
    % \emph{Apache Flink}.

\keywords{Novelty Detection, Intrusion Detection, Data
Streams, Distributed Computing, Fog Computing, IoT devices}

\end{abstract}