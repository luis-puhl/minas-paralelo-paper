\chapter{Cronograma}
\label{cha:crono}

% cap 5: Cronograma até a defesa
% (quando)

Neste capítulo apresentam-se as etapas previstas e sua distribuição temporal.

\begin{itemize}
  \item Enumerar métricas de qualidade de classificação e métricas de
    escalabilidade encontradas na literatura;
  \item Avaliar plataformas de processamento distribuído de fluxos como:
    \begin{itemize}
        \item \emph{Apache Kafka} com \emph{Python};
        \item \emph{Apache Kafka Streams};
        \item \emph{Apache Spark Streaming};
        \item \emph{Apache Storm};
        \item \emph{Apache Flink};
    \end{itemize}
  \item Implementar algoritmo MINAS sobre \emph{Apache Flink};
  \item Executar a implementação com \emph{data sets} públicos relevantes;
  \item Validar, por meio de comparação com o algoritmo original, se a
    implementação gera os mesmos resultados quanto a classificação;
  \item Extrair métricas de escalabilidade por meio de experimentação;
  \item Avaliar estratégias de distribuição do algoritmo em \emph{fog} como:
  \begin{itemize}
    \item Detecção de novidade nas pontas;
    \item Detecção de novidade na nuvem;
    \item Detecção de novidade nas pontas e em nuvem;
  \end{itemize}
    e seu o impacto na qualidade de classificação e volume computado.
\end{itemize}

\begin{enumerate}[label=\alph*)]
  \item \label{A} Desenvolvimento da aplicação;
  \item \label{B} Validação da aplicação em contraste com a implementação
  MINAS original:
    \begin{itemize}
      \item preparação e, se necessário, adaptação da implementação
      original e \emph{data sets};
      \item comparação e, se necessário, ajustes à implementação.
    \end{itemize}
  \item \label{C} Experimentos com \emph{data sets} e estratégias de 
  distribuição em \emph{fog};
  \item \label{D} Comparação com outros trabalhos, envolvendo:
    \begin{itemize}
      \item preparação e, se necessário, adaptação dos programas e \emph{data sets};
      \item comparação e publicação dos resultados.
    \end{itemize}
  \item \label{E} Preparação da base de dados para treinamento e teste: Mês 02
  \item \label{F} Teste ... : meses 04 a 07
  \item 
\end{enumerate}



Quadro (linhas: etapas x colunas: meses)

\begin{ganttchart}[
  vgrid,
  hgrid,
  bar/.append style={fill=green},
  bar incomplete/.append style={fill=red},
  progress=today,
  today=6,
  group progress label node/.append style={below=3pt}
]{1}{12}
  \gantttitle{Cronograma}{12} \\
  \ganttgroup{Group 1}{1}{10} \\
  \ganttbar[
  bar progress label font=\color{green!25!black}\sffamily
  ]{Subtask 1}{1}{3} \\
  \ganttbar[
  progress label text={$\displaystyle\frac{#1}{100}$}
  ]{Subtask 2}{5}{12}
\end{ganttchart}


\begin{ganttchart}[vgrid,
  time slot format=isodate-yearmonth,
  % compress calendar,
  x unit=0.9cm,
  title height = 1,
  y unit title = 0.6cm,
  bar height = .5,
  y unit chart = 0.6cm,
  bar left shift=.15, bar right shift=-.15,
  bar/.style={fill=blue!70, rounded corners=5pt}
]{2020-02}{2020-07}
  \gantttitlecalendar{year, month} \\
  \ganttbar{1}{2019-04}{2019-08} \\
  \ganttbar{3}{2019-04}{2020-06} \\
  \ganttbar{2}{2019-06}{2019-08} \\
  \ganttbar{4}{2019-09}{2019-11} \\
  \ganttbar{5}{2019-12}{2020-01} \\
  \ganttbar{6}{2020-02}{2020-05}  \\
  \ganttbar{7}{2020-05}{2020-07}
  %\ganttbar{\E{5}}{2019-10}{2019-11} \\
  %\ganttbar{\E{6}}{2019-12}{2020-03} \\
  %\ganttbar{\E{7}}{2020-02}{2020-03} \\
  %\ganttbar{\E{8}}{2020-04}{2020-05}  \\
  %\ganttbar{\E{9}}{2020-04}{2020-05} \\
  %\ganttbar{\E{10}}{2020-05}{2020-07}
\end{ganttchart}
