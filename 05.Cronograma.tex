\chapter{Cronograma}
\label{cha:crono}

% cap 5: Cronograma até a defesa
% (quando)

Neste capítulo apresentam-se as etapas previstas e sua distribuição temporal.

% \begin{itemize}
%   \item Enumerar métricas de qualidade de classificação e métricas de
%     escalabilidade encontradas na literatura;
%   \item Avaliar plataformas de processamento distribuído de fluxos como:
%     \begin{itemize}
%         \item \emph{Apache Kafka} com \emph{Python};
%         \item \emph{Apache Kafka Streams};
%         \item \emph{Apache Spark Streaming};
%         \item \emph{Apache Storm};
%         \item \emph{Apache Flink};
%     \end{itemize}
%   \item Implementar algoritmo MINAS sobre \emph{Apache Flink};
%   \item Executar a implementação com \emph{data sets} públicos relevantes;
%   \item Validar, por meio de comparação com o algoritmo original, se a
%     implementação gera os mesmos resultados quanto a classificação;
%   \item Extrair métricas de escalabilidade por meio de experimentação;
%   \item Avaliar estratégias de distribuição do algoritmo em \emph{fog} como:
%   \begin{itemize}
%     \item Detecção de novidade nas pontas;
%     \item Detecção de novidade na nuvem;
%     \item Detecção de novidade nas pontas e em nuvem;
%   \end{itemize}
%     e seu o impacto na qualidade de classificação e volume computado.
% \end{itemize}

\begin{enumerate}[label=\Alph*)]
  \item \label{task:A} Desenvolvimento da aplicação;
  \item \label{B} Validação da aplicação em contraste com a implementação
  MINAS original:
    \begin{itemize}
      \item preparação e, se necessário, adaptação da implementação
      original e \emph{data sets};
      \item comparação e, se necessário, ajustes à implementação.
    \end{itemize}
  \item \label{C} Experimentos com \emph{data sets} e estratégias de 
  distribuição em \emph{fog};
  \item \label{D} Comparação com outros trabalhos, envolvendo:
    \begin{itemize}
      \item preparação e, se necessário, adaptação dos programas e \emph{data sets};
      \item comparação e publicação dos resultados.
    \end{itemize}
\end{enumerate}

\begin{ganttchart}
  [vgrid, expand chart=\textwidth]
  {1}{12}
  \gantttitle{Cronograma 2020}{12} \\
  \gantttitlelist{1,...,12}{1} \\
  \ganttbar{A}{1}{4} \\
  \ganttbar{B}{5}{11}
\end{ganttchart}
