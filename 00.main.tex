% !TeX root = ./00.main.tex
\documentclass[dissertacao]{lib/ppgccufscar}
\usepackage[utf8]{inputenc}
\usepackage{color}
\usepackage{graphicx,url}
\usepackage{multirow}

\usepackage{booktabs}
\usepackage{graphicx}
\usepackage{subfigure}
\usepackage[table,xcdraw]{xcolor}

\usepackage[portuguese,ruled,vlined,linesnumbered]{algorithm2e}
\usepackage{algorithmic}

\usepackage{float}
\usepackage{amsmath}
\usepackage{tabularx}

\usepackage{pgfgantt}
\usepackage{enumitem}

% \usepackage{hyperref}
% \hypersetup{
%     colorlinks=true,
%     linkcolor=blue,
%     filecolor=magenta,
%     urlcolor=cyan,
% }

\titulo{Uma Implementação paralela do algoritmo de Detecção de Novidade em Streams MINAS}
\autor{Luís Henrique Puhl de Souza}
\orientador[Orientador]{Prof. Dr. Hermes Senger}
% \coorientador[Co-orientador]{Prof. Dr. Hermes Senger}
\areaconcentracao{Sistemas de Computação}
\data{Fevereiro/2020}

\begin{document}

\capa
\folhaderosto

% \noindent\makebox[\textwidth]{\includegraphics[width=\paperwidth]{folha_aprovacao.jpg}}

%\begin{agradecimentos}
%\end{agradecimentos}

%\epigrafe{Inserir depois.}{---}
% \epigrafe{O que está certo, não está errado...}

% \begin{resumo}
% \palavraschave{Artigos, Fichamento}
% \end{resumo}

% \begin{abstract}
% No one.
% \keywords{Articles, Review}
% \end{abstract}

% \listoffigures
% \listoftables

% \acronym{TI}{Tecnologia da Informação}
% \listofacronyms

%% sumario
\tableofcontents

%% capítulos
% !TeX root = ./00.main.tex
\chapter{Introdução}\label{cha:intro}

% ELEMENTOS QUE INTRO DEVE TER
% -contexto
% - problema
% - sua solucao
% - como vc vai fazer (metodologia/simples)
% - pq é importante (justificativa)
% - ja teve algum resultado preliminar (se sim, cita ele)
% - organizacao do trabalho( no capitulo 1 teremos... no capitulo 2 teremos ...)

% ------------------------------------------------------------------------------------------------------
% % hélio
% notas sobre a distribuição do modelo
% - cenário com o processamento de novidades local
% - cenário com ND na nuvem

% - minas totalmente active

% helio amanha:
% a. Cenário, b. Problema, c. Proposta e d. Resultado Esperado.
% hermes:
% a. Monitoramento, classificação,
% b. detecção de novidades,
% c. executar em nós multi-core de maneira escalável,
% d. minas com mesma qualidade porém escalável.

% helio:
% Técnicas de firewall tradicional: allow all, deny all.
% Firewall moderno usa modelos.
% Minas como modelo.

% hermes
% iot, nós expostos vitima de ataque, novas funcionalidades ->
% firewall rigido vs modelos -> streams ->
% novidade -> minas -> big data -> discussão de localidade de dados
% paralelização

% motivação: testes preliminares do guilherme que fez teste com tais bases
% teve resultados promissores

% - implementar
% - paralelizar
% - avaliar
% - avaliar aprendizado local ou global trade-offs

% % luis
% O minas abre espaço para diferentes estratégias de paralelismo e distribuição.

% % hermes
% escalabilidade do algoritmo

% Kafka particiona e expõem esse particionamento ao consumidor.
% Tentei usar Python + Kafka, mas não escalou.

% Detalhar a implementação

% % hermes, esqueça kafka, foque em arquivos
% ------------------------------------------------------------------------------------------------------

% Hermes 2020-02-04
% Paralelize o minas no flink.
% (não se preocupe com o uso, seja ele NIDS ou qualquer outra coisa)
% Use a detecção de intrusão apenas como validação do algoritmo.

% cenário 2: NIDS é meu foco, esses são os desafios de NIDS, o minas resolve. mas esse cenário é tragico
% ------------------------------------------------------------------------------------------------------

% cap 1:
%   - Objetivo paralelizar minas em plataforma de big-data capaz de consumir streams de forma eficiente
%   - Motivação: Minas é recente, com potencial em várias aplicações, por exemplo (NIDS, sensores, ...)
%       para isso deseja-se uma implementação eficiente (low power, ou usar todo hardware) e escalável (big-data)
%/hermes
% ------------------------------------------------------------------------------------------------------

% a. Cenário: Monitoramento, classificação,
% b. Problema: detecção de novidades,
% c. Proposta: executar em nós multi-core de maneira escalável,
% d. Resultado Esperado: minas com mesma qualidade porém escalável.
% 
% Linha de argumentação: iot, nós expostos vitima de ataque, novas funcionalidades ->
% firewall rigido vs modelos -> streams -> novidade -> minas ->
% big data -> discussão de localidade de dados e paralelização
% 
% motivação: testes preliminares do guilherme que fez teste com tais bases
% teve resultados promissores
% - implementar
% - paralelizar
% - avaliar
% - avaliar aprendizado local ou global trade-offs

% Se 3 palavras formam uma ideia atômica, tenha 2 ligações nas pontas
% 3 ou 4 linhas max, quebrar em sentenças
% muito Essa, Esse em começo de parágrafo

% \textit{A Internet das Coisas, tema frequentemente abordado nos últimos anos,
% atrelado ao crescimento sem precedentes do número de dispositivos conectados.
% Tais dispositivos captam informações através de sensores e outros meios de
% entrada e geram dados que podem trazer conhecimentos relevantes através de sua  
% análise.}
\acronym{IoT}{Internet of Things, Internet das Coisas}

A Internet das Coisas (\emph{Internet of Things} - IoT) é um sistema global de
dispositivos (máquinas, objetos físicos ou virtuais, sensores, atuadores e
pessoas) com capacidade de intercomunicação pela Internet sem depender de
interação com interface humano-computador tradicional.
O número de dispositivos categorizados como IoT na década passada teve
crescimento sem precedentes e, proporcionalmente, cresceu o volume de dados
gerados por esses dispositivos.

A análise desses dados pode trazer novos conhecimentos e foi um tema frequentemente
abordado por trabalhos de pesquisa na última década.
Além dos dados de sensores e atuadores esses dispositivos, quando subvertidos,
podem gerar outro tipo de tráfego, maligno à sociedade, como o gerado pela
\emph{botnet} mirai em 2016 \cite{Kambourakis2017}.
Nesse cenário são fatores que possibilitam a subversão desses dispositivos:
falta de controle sobre a origem do hardware e software embarcado nos
dispositivos além das cruciais atualizações de segurança.

% The Internet of Things (IoT) has been defined in Recommendation ITU-T Y.2060 (06/2012) as
% a global infrastructure for the information society, enabling advanced services
% by interconnecting (physical and virtual) things based on existing and evolving
% interoperable information and communication technologies.
\acronym{DS}{\emph{Data Stream}, Fluxo de Dados}

Com milhares de dispositivos em redes distantes gerando dados (diretamente
ligados a sua função original e também metadados produzidos como subproduto) com
volume e velocidade consideráveis formando fluxos contínuos de dados (\emph{Data
Stream} - DS) , técnicas de mineração de fluxos de dados
(\emph{Data Stream Mining}) são amplamente necessárias.
Essas técnicas são
aplicadas, por exemplo, em problemas de monitoramento e classificação de valores
originários de sensores para tomada de decisão tanto em nível micro, como
modificação de atuadores remotos, ou macro, na otimização de processos
industriais.
Analogamente, as mesmas técnicas de classificação podem ser aplicadas para os
metadados gerados pela comunicação entre esses nós e a Internet num serviço de
detecção de intrusão.

\acronym{ND}{\emph{Novelty Detection}, Detecção de Novidade}
\newcommand{\nd}{ND}
% \New

Técnicas de \emph{Data Stream Mining} envolvem mineração de dados
(\emph{Data Mining}), aprendizado de
máquina (\emph{Machine Learning}) e recentemente Detecção de Novidades
(\emph{Novelty Detection}, \nd).
ND além de classificar em modelos conhecidos
permite classificar novos padrões e, consequentemente, atuar corretamente mesmo
em face a padrões nunca vistos.
Essa capacidade é relevante em especial para o
exemplo de detecção de intrusão, onde novidades na rede podem distinguir novas
funcionalidades (entregues aos dispositivos após sua implantação em campo) de
ataques por agentes externos sem assinatura existente em bancos de
dados de ataques conhecidos.

% finalizo com "por exemplo"?

Análises como \emph{Data Stream Mining} e ND são tradicionalmente implementadas
sobre o paradigma de computação na nuvem
(\emph{Cloud Computing}) e, recentemente, paradigmas como computação em névoa
(\emph{Fog Computing}). Para \emph{fog}, além dos recursos em \emph{cloud}, são
explorados os recursos distribuídos pela rede desde o nó remoto até a
\emph{cloud}. Processos que dependem desses recursos são distribuídos de acordo
com características como sensibilidade à latência, privacidade,
consumo computacional ou energético.

\section{Motivação}\label{sec:motivo}

\nota{incompleto :(}
No contexto de detecção de novidades para fluxos de dados em \emph{fog}, uma
arquitetura recente proposta por \citeonline{Cassales2019a} mostra resultados promissores.

% baseada nos algoritmos de detecção de novidades em fluxo de dados ECSMiner, AnyNovel e MINAS \cite{Faria2016minas},
ECSMiner, AnyNovel e MINAS


A arquitetura proposta foi avaliada com conjunto de dados (\emph{data set}) \emph{Kyoto 2006+} 
composto de dados coletados de 348 \emph{Honeypots} (máquinas isoladas com diversos softwares
com vulnerabilidades conhecidas expostas à Internet com propósito de atrair
ataques) de 2006 até dezembro 2015.
O \emph{data set} \emph{Kyoto 2006+} contém 24 atributos, 3 etiquetas atribuídas por
detectores de intrusão comerciais e uma etiqueta
distinguindo o tráfego entre normal, ataque conhecido e ataque desconhecido
\cite{Cassales2019a}.
Contudo, o algoritmo MINAS ainda não foi implementado e avaliado com paralelismo
multi-processamento ou distribuído.

Outras propostas tratam do caso de grandes volumes e velocidades, como é o caso
de \citeonline{Viegas2019} que apresenta o \emph{BigFlow} no intuito de detectar
intrusão em redes \emph{10 Gigabit Ethernet}, que é um volume considerável
atualmente impossível de ser processado em um único núcleo de processador
(\emph{single-threaded}). Essa implementação é feita sobre uma plataforma
distribuída processadora de fluxos (\emph{Apache Flink}) executada em um cluster
com até 10 nós de trabalho, cada um com 4 núcleos de processamento totalizando
40 núcleos para atingir taxas de até $10,72 \ Gbps$.

% Além de apresentar uma implementação, \citeonline{Viegas2019} também apresenta o
% \emph{data set} \emph{MAWIFlow}. Esse conjunto é derivado do ponto de coleta
% F, localizado em um elo de comunicação entre o Japão e os EUA (\emph{Backbone})
% com capacidade de $1\ Gbps$, diariamente 15 minutos são capturados desde 2006
% sob supervisão de \citeonline{mawiSamplepointF} \cite{Fontugne2010}. O conjunto
% \emph{MAWIFlow} limita-se às coletas de 2016 ($7.9\ TB$) e estratificado para
% $1\%$ desse tamanho facilitando compartilhamento e avaliação por outros
% softwares. Esse conjunto contempla 158 atributos de nós e fluxos e etiquetado
% por \citeonline{Fontugne2010}.

% \acronym{Drift}{\emph{Concept Drift}, Deriva conceitual: variação temporal de um conceito conhecido}
% \acronym{Evolution}{\emph{Concept Evolution}, Conceitos emergentes: conceitos não  }

% O sistema \emph{BigFlow} é composto de dois estágios: extração de
% características (estatísticas de tráfego da rede) e aprendizado confiável de
% fluxo. O segundo estágio implementa um algoritmo de detecção de novidade
% utilizando classificadores já estabelecidos na biblioteca \emph{Massive Online Analysis framework} (MOA) \cite{MOA} com
% adição de um módulo de verificação que armazena valores classificados com baixa
% confiança para serem manualmente avaliados por um especialista \cite{Viegas2019}.
% A escolha dessa
% abordagem não é nova e visa tratar nuances do problema abordado como
% variação temporal de conceitos conhecidos (\emph{Concept Drift}) e
% conceitos emergentes (\emph{Concept Evolution})
% \cite{Faria2016nd}.
% Essas nuances causam redução de acurácia durante a
% avaliação inicial dos algoritmos tradicionais e devidamente mitigada com a
% atualização constante do modelo. Esses problemas são amplamente abordados e
% tratados em outros algoritmos como o MINAS \cite{Faria2016minas}.

Os trabalhos de \citeonline{Cassales2019a} e \citeonline{Viegas2019} abordam
detecção de intrusão em redes utilizando algoritmos de ND em DS porém com
perspectivas diferentes.
O primeiro observa \emph{IoT} e processamento em \emph{fog}, baseia-se em um
algoritmo genérico de detecção de novidade.
O segundo trabalho observa \emph{backbones} e processamento em \emph{cloud},
implementa o próprio algoritmo de detecção de novidade.
Essas diferenças deixam uma lacuna onde de um lado tem-se uma
arquitetura mais adequada para \emph{fog} com um algoritmo estado da arte de
detecção de novidades porém sem paralelismo e, de outro, tem-se um sistema
escalável de alto desempenho porém almejando outra arquitetura (\emph{cloud}) e
com um algoritmo menos preparado para os desafios de detecção de novidades.

% \nota{
% \\ deixar mais claro o contraste entre cassales e bigflow
% \\ abordar o **gap** no qual a minha pesquisa entra
% \\ mostrar que modelo bigflow não considera fog
% \\ arquiteutra distribuida em fog com minas e alto desempenho bigflow
% }

\section{Objetivos}\label{sec:objetivos}

Como estabelecido na \refsec{motivo}, a lacuna no estado da arte observada é
a ausência de uma implementação de algoritmo de detecção de
novidade que trate adequadamente os desafios de fluxo de dados contínuos e
considere o ambiente de computação em névoa aplicada à detecção de
intrusão.
Seguindo a comparação entre algoritmos desse gênero realizada por
\citeonline{Cassales2019a}, elege-se que o algoritmo MINAS \cite{Faria2016minas}
receberá o tratamento proposto.

Portanto, seguindo os trabalhos do Grupo de Sistemas Distribuídos e Redes
(GSDR), propõem-se a construção de uma aplicação que implemente o algoritmo MINAS
de maneira escalável e distribuível para ambientes de computação em névoa e a avaliação
dessa implementação com experimentos baseados na literatura e conjunto de dados
públicos relevantes.
O resultado esperado é uma implementação compatível em qualidade de
classificação ao algoritmo MINAS e passível de ser distribuído em um ambiente
de computação em névoa aplicado à detecção de intrusão.

% Hermes
% \nota{Citar que o MINAS já foi testado para detecção de intrusão [Guilherme2019].}
% \nota{Falar do resultado bem resumuidamente}
% \nota{Porem a implementação não distribuída.}

%Com a avaliação inicial formulou-se a questão \emph{``quais resultados podem ser esperados de um sistema que implementa um algoritmo de detecção de novidades em fluxo de dados?"} que engloba sucintamente
% os temas abordados nesse trabalho.

% c. Proposta: executar em nós multi-core de maneira escalável,
% d. Resultado Esperado: minas com mesma qualidade porém escalável.
% 
% Linha de argumentação: discussão de localidade de dados e paralelização
% - implementar
% - paralelizar
% - avaliar
% - avaliar aprendizado local ou global trade-offs

Com foco no objetivo geral, alguns objetivos secundários são propostos:

% \nota{citar ferramentas e a escolha só depois do python e kafka}

% \nota{entre flink e spark, outro grupo de pesquisa já está explorando spark}

% \begin{itemize}
% \item
% \end{itemize}

% \nota{
%     estudar o algoritmo\\
%     fazer uma implementação\\
%     pegar datasets relevantes\\
%     comparar a corretude  com o sequencial\\
%     avaliar desempenho e escalabilidade
% }
\begin{itemize}

    \item Implementar o algoritmo MINAS de maneira distribuída sobre uma plataforma de processamento
    distribuída de fluxos de dados;

    \item Avaliar a qualidade de detecção de intrusão em ambiente distribuído 
    conforme a arquitetura IDSa-IOT;
    
    \item Avaliar o desempenho da implementação em ambiente de computação em névoa.

\end{itemize}

% There is a need for real-time stream processing, as data is arriving as
% continuous flows of events; for example, cars in motion emitting GPS signals;
% financial transactions; the interchange of signals between cellphone towers; web
% traffic including things like session tracking and understanding user behavior
% on websites; and measurements from industrial sensors.
% https://dzone.com/articles/streaming-in-spark-flink-and-kafka-1

\section{Proposta Metodológica}

% (Metodologia) (como)

Para cumprir os objetivos citados na \refsec{objetivos}, foi identificado a necessidade
de um processo exploratório seguido de experimentação. Tal processo inclui a
revisão da literatura, tanto acadêmica quanto técnica, seguida da experimentação
através de implementação de aplicação e testes.

O foco da revisão da literatura acadêmica é em trabalhos que abordem:
processamento de fluxos de dados, classificação de fluxo de dados, detecção de
novidades em fluxo de dados e processamento e distribuído de fluxo de dados.
O objetivo da revisão é o estabelecimento do estado da arte desses assuntos
e para que alguns desses trabalhos sirvam para comparações e relacionamentos.
Além disso, desses trabalhos extrai-se métricas de qualidade de classificação
(por exemplo taxa de falso positivo e matriz de confusão) e métricas de
escalabilidade (taxa de mensagens por segundo e escalabilidade vertical ou
horizontal).

A revisão da literatura técnica foca em plataformas, ferramentas e técnicas
para realizar a implementação proposta.
Portanto, serão selecionadas plataformas de processamento distribuído de DS
e técnicas de aprendizado de máquina associadas a elas.
Dessa revisão também são obtidas técnicas ou ferramentas necessárias
para extração das métricas de avaliação bem como \emph{data sets}
públicos relevantes para detecção de novidades em DS.

Uma vez definidos o estado da arte, as ferramentas técnicas e os
\emph{data sets}, o passo seguinte é a experimentação.
Nesse passo é desenvolvida uma aplicação na plataforma escolhida que, com base no
algoritmo MINAS \cite{Faria2016minas}, classifica e detecta novidades em DS.
Também nesse passo a implementação é validada comparando os resultados de
classificação obtidos com os resultados de classificação do algoritmo original
MINAS.
Posteriormente, serão realizados experimentos com a implementação e variações em \emph{data sets} e
cenários de distribuição em \emph{fog} coletando as métricas de classificação e escalabilidade.

Ao final, a aplicação, resultados, comparações e discussões serão publicados
nos meios e formatos adequados como repositórios técnicos, eventos ou revistas
acadêmicas.

\section{Organização do trabalho}


% Flink  - dizer que vai usar, para fazer o que e porque escolheu 

% Kafka ?

% Raspberry - para todos

% A base - 

% 1.2 (Cassales)
%  Metodologia
% A seguir será descrita a metodologia que pretende-se utilizar para atingir aos objetivos e
% responder às questões de pesquisa propostas.
% Primeiramente será realizado um levantamento do estado da arte no que diz respeito aos
% Sistemas de Detecção de Intrusão e às técnicas de Detecção de Novidade aplicadas aos Fluxos
% Contı́nuos de Dados. Com base neste levantamento, serão determinadas as lacunas dos sistemas
% da área e quais as técnicas mais adequadas para a utilização neste trabalho. Dado o foco nas
% técnicas e na arquitetura, é necessário que se possua mais de uma base de dados para realizar
% 1.2 Metodologia
%  13
% a avaliação. Tendo em vista que existem bases de dados de segurança que não representam
% o cenário atual de maneira fidedigna, é necessário selecionar bases que possibilitem que os
% resultados sejam confiáveis.
% Além das bases, serão utilizadas diversas técnicas, as quais deverão ser avaliadas e estuda-
% das para que possam ser propostas melhorias ou mesmo uma nova técnica. Finalmente, serão
% implementadas melhorias e, se necessário, uma nova técnica e será feita avaliação do desempe-
% nho desta técnica.

% \section{Organização do trabalho}

O restante desse trabalho segue a estrutura:
\refcap{fundamentos} aborda conceitos teóricos e técnicos que embasam
esse trabalho;
\refcap{related} enumera e discute trabalhos relacionados e estabelece
o estado da arte do tema detecção de novidade em fluxos de dados e seu processamento;
\refcap{proposta} descreve a proposta de implementação, discute
as escolhas de plataformas e resultados esperados.
Também são distidos no \refcap{proposta} os desafios e resultados preliminares encontrados
durante o desenvolvimento do trabalho;
\refcap{final} adiciona considerações gerais e apresenta o plano de trabalho
e cronograma até a defesa.

\chapter{Fundamentos Científicos e Tecnológicos}

% cap 2: fundamentos científicos e tecnológicos
%     (pegue apenas os mais citados, siga a Elaine)
%     2.1. Computação em Nuvem, Fog e Edge
%     2.2. Plataformas de processamento distribuído
%         - arq labmda, kappa, (vide guilherme)
%         - MapReduce, Haddop, Spark, Storm
%     2.3. Apache Flink
%     2.4. Mineração de Dados
%     2.5. Mineração de Stream
%         - quem são, o que consomem
%         (BigFlow apud Gaber2005) Mining data streams: A Review.
%     2.6. Novelty Detection
%     2.7. O algoritmo Minas

\section{Computação em Nuvem, Borda e Névoa}

Para descrever os modelos de computação em nuvem, borda e névoa, é necessário
abordar o conceito de distância e densidade em rede. Distância pode ser definida
como número de saltos (\emph{hops}), latência, distância geográfica ou combinação destas.
Densidade é extraída da distância, projetando a mesma num hiper-espaço de maneira que os
nós com menor distância entre si fiquem mais próximos. Então quando existe um
grande número de nós numa mesma região diz-se que ela é densa, quando há poucos
nós em uma região, esparsa. Acredita-se que data centers, backbones e nuvens publicas
formem uma concentração de nós e quanto mais próximo do usuários finais (folhas)
mais esparso é esse hiper-espaço.

% Definir internet e rede (borda, centro, etc)
Classificando a internet por sua densidade podemos dizer que ao centro estão
os \emph{data centers} e nuvens públicas em seguida o núcleo interconectando redes diversas,
redes locais e a Borda composta pelos nós folha dentro de uma rede local.

O modelo de Computação em Nuvem (\emph{Cloud Computing})
permite alocar recursos como redes, servidores, armazenamento, aplicações e serviços
de maneira conveniente e seu provisionamento ágil concede elasticidade para atender
demandas variáveis com custo mínimo \cite{NIST2011}.

Alternativamente, a Computação na Borda (\emph{Edge Computing}) destaca-se no
processamento em tempo real de dados originários da própria borda além de atender
preocupações de segurança e privacidade \cite{Shi2016}.

\emph{Edge} é por vezes chamado de Computação em Névoa (\emph{Fog Computing})
contudo \citeonline{IEEECommunicationsSociety2018} indica diferenças como exclusão do
modelo \emph{cloud} e limitação a poucas camadas (por exemplo, somente os nós folha de uma rede)
no modelo \emph{edge} em direto contraste com a inclusão de \emph{cloud} e hierarquias
maiores. \emph{Fog} também atende gerenciamento da rede, armazenamento e controle.

% Fog works with the cloud, whereas edge is defined by the exclusion of cloud. Fog
% is hierarchical, where edge tends to be limited to a small number of layers. In
% additional to computation, fog also addresses networking, storage, control and
% acceleration. \cite{IEEECommunicationsSociety2018}

Esse modelo de computação distribuída desde os nós folha até o centro é motivado
pela mudança do \emph{statu quo} do fluxo dos dados na internet: tradicionalmente
os dados são produzidos pelos dispositivos de borda imediatamente enviados à 
\emph{cloud} (produção, \emph{upstream}),
que armazena e processa recursos derivados servido-os através de requisição-resposta
(consumo, \emph{downstream}) a mais clientes.
Com a ampliação da Internet das Coisas (\emph{Internet of Things}, IoT) e consequente
ampliação sem precedentes do volume de dados gerados, mudando o relação de consumo
e produção \cite{Shi2016}, arquiteturas tradicionais como \emph{cloud} podem não
ser capazes de lidar com esses dados por falta de banda o que leva as
propostas de distribuição vertical do processamento em \emph{fog} \cite{Bonomi2012, Dastjerdi2016}.

% Previous work such as micro datacen- ter [12], [13], cloudlet [14], and fog computing [15]
% [15] F. Bonomi, R. Milito, J. Zhu, and S. Addepalli,
% “Fog computing and its role in the Internet of things,” 
% in Proc. 1st Edition MCC Workshop Mobile Cloud Comput., Helsinki, Finland, 2012, pp. 13–16.

% Fog Computing: Helping the Internet of Things Realize Its Potential
% \cite{Dastjerdi2016}
% The Internet of Things (IoT) could enable
% innovations that enhance the quality of life, but it
% generates unprecedented amounts of data that
% precedented amounts of data that can be useful in many ways, par-
% are difficult for traditional systems, the cloud, and
% even edge computing to handle. Fog computing
% is designed to overcome these limitations.

\section{Mineração de Dados e Fluxo de Dados}

% Faria nem Silva definem ou citam data mining
% A data stream (DS) is a sequence of examples that arrive continuously. They are
% continuous, unbounded, flow at high speed and have a data distribution that
% may change over time \cite{Silva2013}. In DS scenarios, new concepts may
% appear and known concepts may disappear or evolve. \cite{Faria2016nd}

Mineração de Dados é o processo de descoberta de padrões em grandes conjuntos de
dados utilizando métodos derivados de aprendizagem de maquina, estatística e 
banco de dados. Uma caso de mineração de dados é \emph{Big Data} onde o conjunto
de dados não pode ser processado em um tempo relevante por \emph{hardware} e \emph{software}
comum, geralmente coincidente com o limite de armazenado na memória ou
armazenamento principal.

% Data stream mining is concerned with the extraction of knowledge from large
% amounts of continuously generated data in a non-stationary environment. Novelty
% detection (ND), the ability to identify new or unknown situations not
% experienced before, is an important task for learning systems, especially when
% data are acquired incrementally (Perner 2008). In data streams (DSs), where new
% concept can appear, disappear or evolve over time, this is an important issue to
% be addressed. ND in DSs makes it possible to recognize the novel concepts, which
% may indicate the appearance of a new concept, a change in known concepts or the
% presence of noise (Gama 2010).
% 
% Perner P (2008) Concepts for novelty detection and handling based on a case-based
% reasoning process scheme. Eng Appl Artif Intell 22:86–91
% 
% Gama J (2010) Knowledge discovery from data streams, vol 1, 1st edn. CRC press chapman hall, Atlanta

% dados massivamente e continuamente gerados e não persistentes. Fp

Além da dimensão de armazenamento outra dimensão que afeta a maneira como dados
são modelados e manipulados é o tempo. Um Fluxo de Dados (\emph{Data Stream}) é
uma sequência de registros produzidos a uma taxa muito alta, associadas ao tempo
real, ilimitados, que excede recursos de armazenamento e, portanto, pode ser
lida apenas uma vez durante processamento \cite{Gama2007}.

% 
% A data stream is a sequence of unbounded, real-time data records that are
% characterized by the very high data rate, which stresses our computational
% resources, and can be read only once by processing applications [13,8,1,9]
 
% [1] B. Babcock, S. Babu, M. Datar, R. Motwani, J. Widom, Models and issues in
% data stream systems. In: Proceedings of Principles of Database Systems
% (PODS’02), pp. 1–16, 2002. \cite{Babcock2002}

% [2] G. Boone, Reality mining: browsing reality with sensor networks.
% In: Sensors Online, vol. 21, 2004.

% [3] V. Cantoni, L. Lombardi, P. Lombardi, Challenges for data mining in
% distributed sensor networks. ICPR (1) 1000–1007, 2006

% [8] M.M. Gaber, A. Zaslavsky, S. Krishnaswamy, Mining data streams: a review.
% ACMSIGMOD Record, 34(2):18–26, 2005.
% 
% [9] M. Garofalakis, J. Gehrke, R. Rastogi, Querying and mining data streams:
% you only get one look a tutorial. In: Proceedings of the 2002 ACM SIGMOD
% International Conference on Management of Data, June 03–06, Madison, Wisconsin,
% 2002.
% 
% [13] S. Muthukrishnan, Data streams: algorithms and applications. In:
% Proceedings of the Four- teenth Annual ACM-SIAM Symposium on Discrete
% Algorithms, 2003.

Mineração de Fluxo de Dados é análogo à mineração de
dados e \emph{big data} com a restrição temporal onde um registro é unicamente 
associado um tempo, dessa forma além de não ser possível manipular o conjunto
de dados em memória, não é possível recuperar dados fora do intervalo de tempo associado
a eles.

% ----- wikipedia ------
% 
% Data mining is the process of discovering patterns in large data sets involving
% methods at the intersection of machine learning, statistics, and database
% systems.[1] Data mining is an interdisciplinary subfield of computer science and
% statistics with an overall goal to extract information (with intelligent
% methods) from a data set and transform the information into a comprehensible
% structure for further use.[1][2][3][4] Data mining is the analysis step of the
% "knowledge discovery in databases" process or KDD.[5] Aside from the raw
% analysis step, it also involves database and data management aspects, data
% pre-processing, model and inference considerations, interestingness metrics,
% complexity considerations, post-processing of discovered structures,
% visualization, and online updating.[1]
% 
% [1] "Data Mining Curriculum". ACM SIGKDD. 2006-04-30. Retrieved 2014-01-27.
% [2] Clifton, Christopher (2010). "Encyclopædia Britannica: Definition of Data Mining". Retrieved 2010-12-09.
% [3] Hastie, Trevor; Tibshirani, Robert; Friedman, Jerome (2009).
% "The Elements of Statistical Learning: Data Mining, Inference, and Prediction".
% Archived from the original on 2009-11-10. Retrieved 2012-08-07.
% [4] Han, Kamber, Pei, Jaiwei, Micheline, Jian (June 9, 2011).
% Data Mining: Concepts and Techniques (3rd ed.). Morgan Kaufmann. ISBN 978-0-12-381479-1.
% [5] Fayyad, Usama; Piatetsky-Shapiro, Gregory; Smyth, Padhraic (1996).
% "From Data Mining to Knowledge Discovery in Databases" (PDF). Retrieved 17 December 2008.

- quem são, o que consomem
(BigFlow apud Gaber2005) Mining data streams: A Review.

\section{Plataformas de processamento distribuído}

- arq labmda, kappa, (vide guilherme)
- MapReduce, Haddop, Spark, Storm

% Ferramentas de BigData}

O que são e para que servem essas ferramentas 

Breve descrição do MapReduce e Hadoop

Breve descrição do Spark

% Apache Spark}

% Spark is an open-source cluster computing framework with a large global user base.
% It is written in Scala, Java, R, and Python and gives programmers an Application Programming Interface (API)
% built on a fault tolerant, read-only multiset of distributed data items.
% In two years since its initial release (May 2014), it has seen wide acceptability for real-time,
% in-memory, advanced analytics — owing to its speed, ease of use, and the ability to handle
%  sophisticated analytical requirement
% https://dzone.com/articles/streaming-in-spark-flink-and-kafka-1

%  \input{02_1.spark.tex

% # Apache Spark e pyspark

% Apache Spark is an open-source distributed general-purpose cluster-computing
% framework. Spark provides an interface for programming entire clusters with
% implicit data parallelism and fault tolerance. Originally developed at the
% University of California, Berkeley's AMPLab, the Spark codebase was later
% donated to the Apache Software Foundation, which has maintained it since.

Apache Spark é um framework de código fonte aberto para computação
distribuída.[1] Foi desenvolvido no AMPLab da Universidade da Califórnia[2] e
posteriormente repassado para a Apache Software Foundation[3] que o mantém desde
então. Spark provê uma interface para programação de clusters com paralelismo e
tolerância a falhas.

% Apache Spark is a fast and general-purpose cluster computing system. It provides
% high-level APIs in Java, Scala, Python and R, and an optimized engine that
% supports general execution graphs. It also supports a rich set of higher-level
% tools including Spark SQL for SQL and structured data processing, MLlib for
% machine learning, GraphX for graph processing, and Spark Streaming.

-----------------------

Apache Spark \ cite{Zaharia} é um \emph{framework} para construção de sistemas
de computação distribuída em \emph{cluster} com garantias de tolerância a falhas
(execução em computadores não confiáveis) utilizando como premissas: paralelização
e localidade de dados, como 

api em Python (dataframe de pandas)
% ------------------------------------------------------------------------------------------------------


\section{Apache Flink}

Breve descrição do Flink (como esse vai ser usado, precisa explicar um pouco melhor - 2 paginas pelo menos):\\
- arquitetura\\
- modelo de programacao\\
- 1 pequeno exemplo de codigo explicando

\cite{Lopez2018}

% [80] ANDREONI LOPEZ, M., LOBATO, A. G. P., MATTOS, D. M. F., et al. “Um
% Algoritmo Não Supervisionado e Rápido para Seleção de Caracterı́sticas
% em Classificação de Tráfego”. In: XXXV SBRC’2017, Belém- Pará, PA,,
% 2017.
% [81] ANDREONI LOPEZ, M., LOBATO, A. G. P., DUARTE, O. C. M. B., et al.
% “An evaluation of a virtual network function for real-time threat detection
% using stream processing”. In: IEEE Fourth International Conference on
% Mobile and Secure Services (MobiSecServ), pp. 1–5, 2018. doi: 10.1109/
% MOBISECSERV.2018.8311440.
% [82] CHENG, Z., CAVERLEE, J., LEE, K. “You Are Where You Tweet: A Content-
% based Approach to Geo-locating Twitter Users”. In: Proceedings of the
% 19th ACM International Conference on Information and Knowledge Man-
% agement, CIKM ’10, pp. 759–768. ACM, 2010. ISBN: 978-1-4503-0099-5.
% [83] LOBATO, A. G. P., ANDREONI LOPEZ, M., DUARTE, O. C. M. B. “Um
% Sistema Acurado de Detecção de Ameaças em Tempo Real por Processa-
% mento de Fluxos”. In: SBRC’2016, pp. 572–585, Salvador, Bahia, 2016.


% \{Messa queueing systems - Sistemas de filas de mensagens}
% O que são, para que servem e citar alguns exemplos
% \{Kafka}
% 1 pag

% Apache Flink % Apache Flink is an open-source platform for distributed stream
% and batch data processing. Flink’s core is a streaming data flow engine that
% provides data distribution, communication, and fault tolerance for distributed
% computations over data streams. Flink also builds batch processing on top of the
% streaming engine, overlaying native iteration support, managed memory, and
% program optimization.

% Advantages of Flink:

% Flink streaming processes data streams as true streams, i.e. data elements are
% immediately “pipelined” through a streaming program as soon as they arrive. This
% allows performing flexible window operations on streams.

% Better memory management: Explicit memory management gets rid of the occasional spikes found in Spark framework.

% Speed: It manages faster speeds by allowing iterative processing to take place
% on the same node rather than having the cluster run them independently. Its
% performance can be further tuned by tweaking it to re-process only that part of
% data that has changed rather than the entire set. It offers up to a five-fold
% boost in speed when compared to the standard processing algorithm.

% Apache Spark is considered a replacement for the batch-oriented Hadoop system.
% But it includes a component called Apache Spark Streaming, as well. Contrast
% this with Apache Flink, which is a Big Data processing tool and it is known to
% process big data quickly with low data latency and high fault tolerance on
% distributed systems on a large scale. Its defining feature is its ability to
% process streaming data in real time.

\section{Detecção de Novidade}

Novelty Detection

breve descrição do que sao algoritmos para DN

ver se tem algum survey e citar

\section{O algoritmo MINAS}

breve descrição do MINAS \cite{deFaria2016}

ver paper da profa. Elaine

% discussão de 2020-02-01
Detecção de intrusão em redes
    - riscos de segurança
    % pontos de coleta de dados b´asicos para a maioria das estruturas de IoT sao Wireless Sensor Networks (WSN) e WSN baseadas em IP,
    % as quais s˜ao vulner´aveis e geram uma ameac¸a de seguranc¸a de alto n´ıvel (ADAT; GUPTA, 2018) Adat2018
    % (KASINATHAN et al., 2013), a detecc¸ ˜ao de assinaturas
    % (RAZA; WALLGREN; VOIGT, 2013; SHEIKHAN; BOSTANI, 2016) s˜ao propostos IDSs h´ıbridos com foco 
                % espec´ıfico em ataques de roteamento como sink-hole e redireciona- mento seletivo
    - técnicas de intrusão e tipos de ataques
    - mecanismo de detecção (análise de fluxo de rede -> detecção de anomalia)
    % Guilherme: A tarefa de detecção de intrusão consiste em descobrir, 
    %           determinar e identificar a utilização, duplicação, alteração ou destruição
    %           não autorizada de sistemas de informação (MUKKAMALA; SUNG; ABRAHAM, 2005) Mukkamala2005
    % deteccao por assinaturas (tambem chamada de misuse-detection), deteccao comportamental (tambem chamada de anomaly-detection)e deteccao hıbrida.(MODI et al., 2013).
    % implementacao usualmente ´e feita por meio de t´ecnicas de AM e MD (BUCZAK; GUVEN, 2016).
    % existem poucos trabalhos [..] online e deteccao de novidade ao problema [..] observado nas surveys (BUCZAK; GUVEN, 2016; MITCHELL; CHEN, 2014; MODI et al., 2013).
    % (FURQUIM et al., 2018), os autores implementam uma arquitetura de 3 camadas (WSN, Fog e Cloud)
    % (MIDI et al., 2017), os autores propuseram um IDS h´ıbrido [...] ativa apenas os m´odulos [...] especializado em um ataque espec´ıfic
    % (FAISAL et al., 2015)externo ou interno ao Smart Meter.dados do KDD99 [...] precis˜ao, Kappa, consumo de mem´oria, tempo e FAR
    % extensivamente em (Sommer; Paxson, 2010) e (MCHUGH, 2000), ´e dif´ıcil encontrar boas me- didas de avaliac¸ ˜ao para IDSs
    % (GAMA, 2010) afirma que no contexto de processamento de fluxo, as medidas tradicionais s˜ao impreci- sas.
Detecção de novidades
    % (PERNER, 2007)(GAMA, 2010). A
    - técnicas de Detecção de novidades
    - MINAS (incluir métricas) 
    % (FARIA et al., 2016)
    % ECSMiner (MASUD et al., 2011)
    % AnyNovel (ABDALLAH et al., 2016) s˜ao
    % medidas de Qualidade da Deteccao utilizadas foram Fnew, Mnew, Erro e a quantidade de exemplos rotulados por especialistas que cada t´ecnica requisitou (MASUD et al., 2011)
    - BigFlow (incluir métricas)
Processamento de Streams (big data)
    - cloud?
    % A arquitetura Lambda (MARZ; WARREN, 2015) de duas camadas de CPU (stream e batch) e camada de serviço
    % Kappa (KREPS, 2014) possui apenas um m´odulo de processamento on-line (apenas as camadas de processamento e servic¸o)
    - redes como stream
    - Atraso
    - Kafka/Spark/Flink
Redes IoT
    - Restrição hardware (Energia, CPU, Mem, Rede)
    - Consideração FOG vs Cloud
%/discussão

\chapter{Trabalhos Relacionados}\label{cha:related}

% % discussão de 2020-02-01
% Aqueles que contenham:
%     - detecção de anomalia em streams
%     - detecção de intrusão em rede com processamento de streams
%     - BigFlow
% %/discussão

% cap 3: Trabalhos relacionados
%     - Artigos sobre o Minas
%     - outros que paralelizaram algoritmos de mineração de 
%        dados/streams alguns online (5-10 refs)
%     - implementação paralelas/distribuídas em dispositivos pequenos
% % 

Esse Capítulo trata dos trabalhos relacionados e estabelece o estado da arte
dos tópicos Detecção de Novidades em Fluxos de Dados, e 
Processamento Distribuído de Fluxos de Dados.

\section{Algoritmo MINAS e Algoritmos Derivados}

\newcommand{\cluster}{\emph{cluster}\xspace}
\newcommand{\clusters}{\emph{clusters}\xspace}
\newcommand{\dataset}{\emph{data set}\xspace}
\newcommand{\datasets}{\emph{data sets}\xspace}
% \clusters é um conjunto de \cluster feito de um \dataset.

O algoritmo MINAS, como já foi discutido, classifica exemplos e detecta
novidades em DS e considera em sua composição \emph{concept drift} e
\emph{concept evolution}, sendo capaz de classificar como extensão de classe
conhecida e reconhecer novas classes sem intervenção de especialista
\cite{Faria2016minas}. Neste trabalho, consideram-se algoritmos derivados
aqueles apresentados em trabalhos publicados após 2016 que estendem a
implementação original seguindo sua estrutura básica.

\subsection*{Algoritmo FuzzyND}

% FuzzyND
% $(n, \mathit{M}, \overline{CF1^x}, SSD^e, t, l)$
% $(n, LS, SS, t, l)$
% A nova estrutura contrapõem a estrutura original
% substituindo a soma linear dos elementos ($LS$) por  e $SS$ por $M$ e $$

O algoritmo FuzzyND, derivado do MINAS é proposto por \citeonline{DaSilva2018}.
FuzzyND incrementa o algoritmo inicial aplicando à ele teorias de
conjuntos \emph{fuzzy} pela modificação da representação dos \clusters.
A modificação afeta os métodos de construção de \clusters, classificação
de exemplos e detecção de novidades de acordo com a nova representação.

\acronym{F1M}{\emph{Macro F-Score}, acurácia }

A avaliação do algoritmo FuzzyND é feita por meio de experimentos usando 3 
\datasets sintéticos (\emph{MOA3}, \emph{RBF}, \emph{SynEDC})
e comparação com o MINAS.
O método de avaliação utilizado baseia-se na matriz de confusão incremental
descrita por \citeonline{Faria2016nd} extraindo dessa matriz duas métricas:
acurácia (\emph{Macro F-Score}) \cite{Sokolova2009} e
taxa de desconhecidos (\emph{UnkR}) \cite{Faria2016minas}.
Em geral o algoritmo FuzzyND detecta melhor novidades e, consequentemente,
é mais robusto à valores atípicos (\emph{outlier}) porém perde a capaciade
de reconhecer padrões recorrentes.


% Experiments were evaluated using the incremental confusion-matrix proposed by [27],
% recently been proposed [5]–[9]
% [5] T. Al-Khateeb, M. M. Masud, L. Khan, C. Aggarwal, J. Han, and B. Thuraisingham, “Stream classification with recurring and novel class detection using class-based ensemble,” in Data Mining (ICDM), 2012 IEEE 12th International Conference on. IEEE, 2012, pp. 31–40.
% [6] E. R. de Faria, A. C. P. de Leon Ferreira, J. Gama et al., “Minas: multiclass learning algorithm for novelty detection in data streams,” Data Mining and Knowledge Discovery, vol. 30, no. 3, pp. 640–680, 2016.
% [7] M. Masud, J. Gao, L. Khan, J. Han, and B. M. Thuraisingham, “Classification and novel class detection in concept-drifting data streams under time constraints,” IEEE Transactions on Knowledge and Data Engineering, vol. 23, no. 6, pp. 859–874, 2011.
% [8] M. M. Masud, Q. Chen, L. Khan, C. Aggarwal, J. Gao, J. Han, and B. Thuraisingham, “Addressing concept-evolution in concept-drifting data streams,” in Data Mining (ICDM), 2010 IEEE 10th International Conference on. IEEE, 2010, pp. 929–934.
% [9] Z. S. Abdallah, M. M. Gaber, B. Srinivasan, and S. Krishnaswamy, “Anynovel: detection of novel concepts in evolving data streams,” Evolving Systems, vol. 7, no. 2, pp. 73–93, 2016.
% [27] E. R. de Faria, I. R. Goncalves, J. Gama, A. C. P. de Leon Ferreira et al., “Evaluation of multiclass novelty detection algorithms for data streams,” IEEE Transactions on Knowledge and Data Engineering, vol. 27, no. 11, pp. 2961–2973, 2015.
% [28] M. Sokolova and G. Lapalme, “A systematic analysis of performance measures for classification tasks,” Information Processing & Manage- ment, vol. 45, no. 4, pp. 427–437, 2009.

\subsection*{Algoritmos MINAS-LC e MINAS-BR}

O algoritmo MINAS-LC é proposto por \citeonline{Costa2019thesis} e trata classificação
multi-rótulo porém não trata evoluções de conceito (novas classes).
AS alterações fundamentais são:
a representação de \cluster onde MINAS-LC troca a etiqueta, que era única, por uma multi-rótulo;
a transformação de problema aplicada ao conjunto de treinamento para transforma-lo de um
conjunto multi-rotulo para um conjunto multi-classe (simplificação)
em duas variações \emph{Label Powerset} e \emph{Pruned Sets} com
mineração de conjunto de itens frequentes.

% Este capítulo apresentou o método MultI-label learNing Algorithm for data Streams
% with Label Combination-based methods (MINAS-LC) e o MultI-label learNing Algorithm for data Streams with Binary Relevance transformation (MINAS-BR) para CMFCD com latência extrema de rótulos.
% O MINAS-LC lida com problemas apenas com mudanças de conceito. O seu modelo
% de decisão e composto por microgrupos multirrotulados sendo capaz de classificar exemplos em várias classes simultaneamente e evoluir ao longo do fluxo de dados. Foram propostas duas variações do método: utilizando o método de transformação de problema Label Powerset (LP) e, utilizando o método Pruned Sets (PS) com mineração de conjunto de itens frequentes. O MINAS-BR lida com problemas tanto com mudanças de conceito, como com evo-
% luções de conceito. Ele possui um conjunto de modelos de decisão, um para cada classe do problema. Esses modelos de decisão podem ser entendidos adaptando-se às mudanças de con- ceito, ou novos modelos de decisão podem ser criados, adaptando-se às evoluções de conceito. O próximo capítulo apresenta os experimentos realizados envolvendo os dois métodos
% propostos neste trabalho.

Já o trabalho de \citeonline{Costa2019}, estende o algoritmo original para que
classifique um exemplo com uma ou mais etiquetas usando a transformação
\emph{Binary Relevance} propondo o algoritmo MINAS-BR.
O algoritmo modifica a representação do modelo, originalmente conjunto de \clusters, para
um grupo de \clusters por classe (etiqueta).
Também modifica o método de agrupamento substituindo a inicialização do 
algoritmo \emph{K-means}, originalmente aleatória, pelo algoritmo 
\emph{Leader Incremental Clustering} \cite{Vijaya2004505}.

% as 4CRE-V13, 4CRE-V24 e 5CVT5 6 foram geradas originalmente em Souza et al. (2015b)
% SOUZA, V. M. A.; SILVA, D. F.; GAMA, J.; BATISTA, G. E. A. P. A. Data stream classification guided by clustering on nonstationary environments and extreme verification latency. In: Procee- dings ofSIAM International Conference on Data Mining (SDM). [S.l.: s.n.], 2015. p. 873–881. Citado 4 vezes nas páginas 17, 65, 87 e 89.

O algoritmo MINAS-BR também é experimentalmente avaliado com 4 \emph{data sets} sintéticos:
\emph{MOA-3C-5C-2D}, \emph{MOA-5C-7C-2D}, \emph{MOA-5C-7C-3} da ferramenta  MOA \cite{MOA}
e \emph{4CRE-V2}
\footnote{A versão original do \dataset 4CRE-V2 está disponível em https://sites.google.com/site/nonstationaryarchive/home}
gerados pelo método \emph{Radial Basis Function} \cite{souza2015}.
MINAS-BR é comparado com 7 algoritmos da literatura também disponíveis na ferramenta
MOA \cite{MOA},
diferente da avaliação do FuzzyND que compara diretamente com MINAS.
Os 7 algoritmos são divididos em dois grupos: 3 com acesso às etiquetas corretas para
atualização do modelo e com a técnica ADWIN (\emph{ADaptive WINdowing}) para detectar
mudanças de conceito (\emph{Concept Drift}); 4 algoritmos sem acesso às etiquetas corretas,
ou seja, sem \emph{feedback} externo, mesma condição do MINAS-BR.

% Esse trecho parece mais fundamentação.

A avaliação elencada por \citeonline{Costa2019} leva em consideração que as classes
contidas no conjunto de testes podem não ter correlação direta com os padrões identificados
pelos algoritmos.
Para tratar a divergência, uma estratégia baseada em proposta anterior por
\citeonline{Faria2016nd} é apresentada com alterações para exemplos multi-rótulo.
A estratégia é executada na fase de classificação seguindo as regras:
\begin{enumerate*}
    \item após o consumo do exemplo $X_n$;
    \item para todo padrão $P_i$ (etiqueta atribuída) identificado sem associação até o momento;
    \item com classes novidade $y_j$ (etiqueta real) presentes em exemplos antes $X_n$;
    \item preenche-se a tabela de contingência $\mathbf{T}_{(i,j)}$ relacionando padrão $P_i$ e classe $y_j$;
    \item calcula-se o grau de dependência $\mathit{F1}$ derivado da tabela de contingência
    $\mathit{F1}_{(i,j)} = f(\mathbf{T}_{(i,j)})$;
    \item valores $\mathit{F1}_{(i,j)} = 0$ são descartados;
    \item dentre os valores restantes: o padrão $P_i$ é associado à classe $y_j$
    se $\mathit{F1}_{(i,j)}$ é máximo.
\end{enumerate*}
Após associação entre padrões de novidade e classes novidade é possível calcular métricas tradicionais.

As métricas utilizadas por \citeonline{Costa2019} após a associação de classes e padrões são
as tradicionais taxa de desconhecidos (\emph{UnkRM}) e \emph{F1M}.
Os resultados apresentados indicam que MINAS-BR capturou todas as novidades dos \datasets sintéticos de teste
e mostrou, como esperado, melhores métricas que os 4 algoritmos equivalentes da literatura ficando abaixo
dos 3 com \emph{feedback} externo.

Os trabalhos relacionados nessa Seção tem em comum muito além do algoritmo base,
tem também métricas de avaliação acurácia (\emph{Macro F-Score} e \emph{Macro F-Measure} F1M)
e taxa de desconhecidos, aplicadas com devido tratamento.
Também é comum entre eles o uso de \datasets sintéticos.
Outro potencial não explorado do MINAS é em aplicações de reais, ou seja,
consumindo além de \datasets reais, fluxos realistas em ambientes simulados ou reais porém
considerando uso de recursos computacionais.

Observando a arquitetura dos algoritmos abordados, todas são extremamente semelhantes:
a fase offline centrada no processo de agrupamento e criação de modelo;
a fase online dividida em classificação (com atualização das estatísticas do modelo)
e detecção de padrões, onde novamente o processo de agrupamento é central.
Portanto, apesar de outros trabalhos expandirem o algoritmo com diferentes técnicas, seu
núcleo continua relevante \cite{DaSilva2018,DaSilva2018thesis,Costa2019}\footnote{
Propostas de modificação do algoritmo MINAS estão longe de serem exauridas.
Não cabe ao presente trabalho expandir e validar conceitos de aprendizagem de maquina
porém alguns exemplos mencionados ainda não abordados são: \begin{enumerate*}[label={\alph*)}]
    \item diferentes métodos de cálculo de distância entre pontos além da distância euclidiana; 
    \item a mudança de representação de \clusters, atualmente hiper-esferas, para hiper-cubos
    para tratar \datasets onde as características representadas
    pelas dimensões são completamente independentes;
    \item um modo interativo onde o \cluster é formado, mostrado ao especialista
    que o classifica como inválido (ruido ou não representativo) ou válido, podendo conter
    uma ou mais classes e, se conter mais que uma classe corte em grupos menores até conter somente
    uma classe;
    \item ainda considerando interação com especialista, a possibilidade dele injetar 
    um exemplo não pertencer à uma classe, ou seja, marcar o exemplo como não
    pertencente à uma classe para manter ele na memória de desconhecidos e, eventualmente forçar
    criação de um \cluster que represente uma classe geometricamente próxima mas semanticamente distinta;
    \item na fase \emph{offline} a verificação de sobreposição de \clusters pertencentes à
    classes distintas e tratamento adequado.
\end{enumerate*} 
}.

% >>>>>>> master
% \citeonline{DaSilva2018}

% \citeonline{Costa2019} estende o algoritmo original na sua capacidade de
% classificar um exemplo com uma ou mais etiquetas usando a transformação
% \emph{Binary Relevance}. Essa versão modificada é testada e comparada com 7
% algoritmos por meio de experimentos com 4 \emph{data sets} sintéticos gerados
% pelo método \emph{Radial Basis Function}. Os 7 algoritmos e o método estão
% disponíveis na ferramenta MOA \cite{MOA}.
% w

% Apesar de outros trabalhos expandirem o algoritmo com diferentes técnicas, seu
% núcleo continua relevante \cite{DaSilva2018,DaSilva2018thesis,Costa2019}.
% <<<<<<

\section{AnyNovel}
\textbf{Incompleto}

\textit{também é da mesma classe do minas porém \citeonline{Cassales2019a} destaca um
desempenho inferior para o \emph{data set} testado.}
%  a atividade de detecção de intrusão

\citeonline{abdallah2016anynovel}

\section{Catraca Lopez2018}
\textbf{Incompleto}

\cite{Lopez2018}

% A monitoring and threat detection system using stream processing as a virtual function for Big Data

% A detecção tardia de ameaças de segurança causa um significante aumento no
% risco de danos irreparáveis, impossibilitando qualquer tentativa de defesa.
% Como consequência, a detecção rápida de ameaças em tempo real é essencial
% para a ad- ministração de segurança. Além disso, A tecnologia de
% virtualização de funções de rede (Network Function Virtualization - NFV)
% oferece novas oportunidades para soluções de segurança eficazes e de baixo
% custo. Propomos um sistema de detecção de ameaças rápido e eficiente,
% baseado em algoritmos de processamento de fluxo e de aprendizado de máquina. As
% principais contribuições deste trabalho são: 
% i) um novo sistema de monitoramento e detecção de ameaças baseado no processamento de fluxo; 
% ii) dois conjuntos de dados, o primeiro é um conjunto de dados sintético de
% segurança contendo tráfego suspeito e malicioso, e o segundo corresponde a uma
% semana de tráfego real de um operador de telecomunicações no Rio de Janeiro,
% Brasil; 
% iii) um algoritmo de pré-processamento de dados composto por um algoritmo
% de normalização e um algoritmo para seleção rápida de
% características com base na correlação entre variáveis;
% iv) uma função de
% rede virtualizada em uma plataforma de código aberto para fornecer um serviço
% de detecção de ameaças em tempo real;
% v) posicionamento quase perfeito de
% sensores através de uma heurística proposta para posicionamento estratégico
% de sensores na infraestrutura de rede, com um número mínimo de sensores; e,
% finalmente, 
% vi) um algoritmo guloso que aloca sob demanda uma sequência de 
% funções de rede virtual.

\section{BigFlow}
\textbf{Incompleto}

%  notas de bigflow
% Table 1
% Network-level feature set used in the experiments throughout this work [18].
% Types: Host-based (Host to All), Flow-based (Source to Destination, Destination to Source, Both)
% Features:\\
%     - Number of Packets,\\
%     - Number of Bytes,\\
%     - Average Packet Size,\\
%     - Percentage of Packets (PSH Flag),\\
%     - Percentage of Packets (SYN and FIN Flags),\\
%     - Percentage of Packets (FIN Flag),\\
%     - Percentage of Packets (SYN Flag),\\
%     - Percentage of Packets (ACK Flag),\\
%     - Percentage of Packets (RST Flag),\\
%     - Percentage of Packets (ICMP Redirect Flag),\\
%     - Percentage of Packets (ICMP Redirect Flag),\\
%     - Percentage of Packets (ICMP Time Exceeded Flag),\\
%     - Percentage of Packets (ICMP Unreachable Flag),\\
%     - Percentage of Packets (ICMP Other Types Flag),\\
%     - Throughput in Bytes,\\
%     - Protocol\\

BigFlow destaca em sua secção 2 (backgroud) o processamento de streams [18, 19],
a preferencia de NIDS por anomalia em contraste aos NIDS por assinatura [30, 31, 32],
a variabilidade e evolução dos padrões de tráfego em redes de propósito geral [9, 11, 20],
a necessidade de atualização regular do modelo classificador [8, 9, 10, 20] e
o tratamento de eventos onde a confiança resultante da classificação é baixa [9, 12, 13].

Também destaca em sua secção 3 (MAWIFlow) 
que data sets adequados para NIDS são poucos devido o conjunto de qualidades que os mesmos
devem atender como realismo, validade, etiquetamento, grande variabilidade
e reprodutividade (disponibilidade pública) [8, 9, 10, 17, 38].

% discutir somente técnicas e estratégias
% dão dê aval aos dados, especialmente os de 'venda' do artigo.

Para avaliar o desempenho de NIDS o data set MAWIFlow é proposto. Originário do 
'Packet traces from WIDE backbone, samplepoint-F' composto por seções de captura de pacotes
diárias de 15 minutos de um link de 1Gbps entre Japão e EUA, com inicio em 2006 continuamente até hoje,
anonimizados [22], etiquetados por MAWILab [8].
Desse data set original apenas os eventos de 2016 são utilizados e desses 158 atributos são extraídas
resultando em 7.9 TB de captura de pacotes. Além disso, os dados são estratificados [24] para redução
de seu tamanho a um centésimo mantendo as proporções de etiquetas (Ataque e Normal)
facilitando o compartilhamento e avaliação de NIDS além de atender as qualidades anteriormente mencionadas.

Com o data set MAWIFlow original e reduzido foram avaliados quatro classificadores [42, 43, 44, 45]
da literatura em dois modos de operação quanto seus dados de treinamento
(ambos contendo uma semana de captura) o primeiro usando somente a primeira semana do ano e as demais
como teste e o segundo modo usando a semana anterior como treinamento e a seguinte como teste.
Demostrando, com 62 atributos, que a qualidade da classificação retrai com o tempo quando não há
atualização frequente do modelo classificador.

Falar do modelo de distribuição.

\textit{concluir com um gap}

\textit{Discutir que o data set não é de borda de uma rede, portanto não tem
relevância para fog. Também discutir a classificação dos fluxos por endpoint
exarcerbando assim a distinção na fog com o efeito de particionamento dos dados.
Ou seja, um nó só vê e classifica os próprios dados.}

\chapter{Implementação e testes}
% cap 4: Implementação e testes
%     4.1. descrição da implementação
%         - offline, online, ND, Clustering
%         - observação de paralelização
%         - complexidade bigO (?)
%     4.2. cenário de teste 
%         - detecção de intrusão
%         - Arquitetura guilherme (dispositivos pequenos vs cloud)
%     4.3. Resultados de experimentos
%         - gráficos, tempos, tabelas...
%         - análises e comentários


\section{Descrição da Implementação}
% \section{Objetivos a serem alcançados}
% O que de fato vc irá fazer 

- offline, online, ND, Clustering

- observação/Considerações de paralelização

Notas sobre implementação Python/Kafka/Minas (não escala como esperado)

Dificuldade no processamento distribuido em Flink.

- complexidade bigO (?)

\section{Cenário de Teste}

Para testar e demonstrar essa implementação um cenário de aplicação é construido
onde seria vantajoso distribuir o processamento segundo o modelo \emph{fog}. Alguns
cenários de exemplo são
casos onde deve-se tomar ação caso uma classe ou anomalia seja detectada

- detecção de intrusão
- Arquitetura guilherme (dispositivos pequenos vs cloud)
\cite{Cassales2019a}
% Descrever a arquitetura IDS-IoT do paper do Guilherme

- BigFlow com dataset atual e maior
dataset kdd99
% dataset kyoto não está disponível http://www.takakura.com/Kyoto_data/

\section{Experimentos e Resultados}
    - gráficos, tempos, tabelas...
    - análises e comentários

Mostrar alguma implementação já feita e que esteja funcionando minimamente

Mostrar resultados mesmo que sejam bem simples e básicos,
apenas para demonstrar que vc domina o ambiente e as ferramentas e
que está apto a avançar no trabalho 

% discussão de 2020-02-01
passos feitos/a fazer
1. Entender Minas
2. Analisar/descrever dataset KDD
3. Notas sobre implementação Python/Kafka/Minas (não escala como esperado)
4. BigFlow (dataset mais novo, usa flink)
5. Plataforma Flink (processamento distribuído)
Proposta
6. Implementar minas em Scala/Flink
7. Testar com datasets KDD e BigFlow
8. Validar/Comparar métricas com seus trabalhos correspondentes
%/discussão

% Notas lendo Quali Casssales:
- Descrição do hardware utilizado pode conter:
    - Arch, OS, Kernel,
    - CPU (core, thread, freq),
    - RAM (total/free size, freq),
    - Disk (total/free size, seq RW, rand RW),
    - Net IO between nodes (direct crossover, switched, wireless, to cloud) (bandwidth, latency).
essas métricas permitem relacionar trade-offs para as questões de fog: Processar em node, edge ou cloud?

Provavelmente vou retirar o kafka da jogada em node/edge, deixando apenas em cloud.

\chapter{Cronograma}
\label{cha:crono}

% cap 5: Cronograma até a defesa
% (quando)

Neste capítulo apresentam-se as etapas previstas e sua distribuição temporal.

% \begin{itemize}
%   \item Enumerar métricas de qualidade de classificação e métricas de
%     escalabilidade encontradas na literatura;
%   \item Avaliar plataformas de processamento distribuído de fluxos como:
%     \begin{itemize}
%         \item \emph{Apache Kafka} com \emph{Python};
%         \item \emph{Apache Kafka Streams};
%         \item \emph{Apache Spark Streaming};
%         \item \emph{Apache Storm};
%         \item \emph{Apache Flink};
%     \end{itemize}
%   \item Implementar algoritmo MINAS sobre \emph{Apache Flink};
%   \item Executar a implementação com \emph{data sets} públicos relevantes;
%   \item Validar, por meio de comparação com o algoritmo original, se a
%     implementação gera os mesmos resultados quanto a classificação;
%   \item Extrair métricas de escalabilidade por meio de experimentação;
%   \item Avaliar estratégias de distribuição do algoritmo em \emph{fog} como:
%   \begin{itemize}
%     \item Detecção de novidade nas pontas;
%     \item Detecção de novidade na nuvem;
%     \item Detecção de novidade nas pontas e em nuvem;
%   \end{itemize}
%     e seu o impacto na qualidade de classificação e volume computado.
% \end{itemize}

\begin{enumerate}[label=\Alph*)]
  \item \label{task:A} Desenvolvimento da aplicação;
  \item \label{B} Validação da aplicação em contraste com a implementação
  MINAS original:
    \begin{itemize}
      \item preparação e, se necessário, adaptação da implementação
      original e \emph{data sets};
      \item comparação e, se necessário, ajustes à implementação.
    \end{itemize}
  \item \label{C} Experimentos com \emph{data sets} e estratégias de 
  distribuição em \emph{fog};
  \item \label{D} Comparação com outros trabalhos, envolvendo:
    \begin{itemize}
      \item preparação e, se necessário, adaptação dos programas e \emph{data sets};
      \item comparação e publicação dos resultados.
    \end{itemize}
\end{enumerate}

\begin{ganttchart}
  [vgrid, expand chart=\textwidth]
  {1}{12}
  \gantttitle{Cronograma 2020}{12} \\
  \gantttitlelist{1,...,12}{1} \\
  \ganttbar{A}{1}{4} \\
  \ganttbar{B}{5}{11}
\end{ganttchart}


\bibliography{99.referencias}

%% glossário

\end{document}
