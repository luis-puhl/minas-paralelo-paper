\chapter{Trabalhos Relacionados}

cap 3: Trabalhos relacionados
    - Artigos sobre o Minas
    - outros que paralelizaram algoritmos de mineração de dados/streams alguns online (5-10 refs)
    - implementação paralelas/distribuídas em dispositivos pequenos
% 

% A monitoring and threat detection system using stream processing as a virtual function for Big Data

% A detecção tardia de ameaças de segurança causa um significante aumento no
% risco de danos irreparáveis, impossibilitando qualquer tentativa de defesa.
% Como consequência, a detecção rápida de ameaças em tempo real é essencial
% para a ad- ministração de segurança. Além disso, A tecnologia de
% virtualização de funções de rede (Network Function Virtualization - NFV)
% oferece novas oportunidades para soluções de segurança eficazes e de baixo
% custo. Propomos um sistema de detecção de ameaças rápido e eficiente,
% baseado em algoritmos de processamento de fluxo e de aprendizado de máquina. As
% principais contribuições deste trabalho são: 
% i) um novo sistema de monitoramento e detecção de ameaças baseado no processamento de fluxo; 
% ii) dois conjuntos de dados, o primeiro é um conjunto de dados sintético de
% segurança contendo tráfego suspeito e malicioso, e o segundo corresponde a uma
% semana de tráfego real de um operador de telecomunicações no Rio de Janeiro,
% Brasil; 
% iii) um algoritmo de pré-processamento de dados composto por um algoritmo
% de normalização e um algoritmo para seleção rápida de
% características com base na correlação entre variáveis;
% iv) uma função de
% rede virtualizada em uma plataforma de código aberto para fornecer um serviço
% de detecção de ameaças em tempo real;
% v) posicionamento quase perfeito de
% sensores através de uma heurística proposta para posicionamento estratégico
% de sensores na infraestrutura de rede, com um número mínimo de sensores; e,
% finalmente, 
% vi) um algoritmo guloso que aloca sob demanda uma sequência de 
% funções de rede virtual.

% FuzzyND Minas

% hélio
% notas sobre a distribuição do modelo
% cenário com o processamento de novidades local
% cenário com ND na núvem 

% hermes
% escalabilidade do algoritmo

\cite{Lopez2018}

% discussão de 2020-02-01
Aqueles que contenham:
    - detecção de anomalia em streams
    - detecção de intrusão em rede com processamento de streams
    - BigFlow
%/discussão


\section{BigFlow}

%  notas de bigflow
Table 1
Network-level feature set used in the experiments throughout this work [18].
Types: Host-based (Host to All), Flow-based (Source to Destination, Destination to Source, Both)

% \begin{itemize}
%     \item Number of Packets,
%     \item Number of Bytes,
%     \item Average Packet Size,
%     \item Percentage of Packets (PSH Flag),
%     \item Percentage of Packets (SYN and FIN Flags),
%     \item Percentage of Packets (FIN Flag),
%     \item Percentage of Packets (SYN Flag),
%     \item Percentage of Packets (ACK Flag),
%     \item Percentage of Packets (RST Flag),
%     \item Percentage of Packets (ICMP Redirect Flag),
%     \item Percentage of Packets (ICMP Redirect Flag),
%     \item Percentage of Packets (ICMP Time Exceeded Flag),
%     \item Percentage of Packets (ICMP Unreachable Flag),
%     \item Percentage of Packets (ICMP Other Types Flag),
%     \item Throughput in Bytes,
%     \item Protocol
% \end{itemize}

Features:\\
    - Number of Packets,\\
    - Number of Bytes,\\
    - Average Packet Size,\\
    - Percentage of Packets (PSH Flag),\\
    - Percentage of Packets (SYN and FIN Flags),\\
    - Percentage of Packets (FIN Flag),\\
    - Percentage of Packets (SYN Flag),\\
    - Percentage of Packets (ACK Flag),\\
    - Percentage of Packets (RST Flag),\\
    - Percentage of Packets (ICMP Redirect Flag),\\
    - Percentage of Packets (ICMP Redirect Flag),\\
    - Percentage of Packets (ICMP Time Exceeded Flag),\\
    - Percentage of Packets (ICMP Unreachable Flag),\\
    - Percentage of Packets (ICMP Other Types Flag),\\
    - Throughput in Bytes,\\
    - Protocol\\


BigFlow destaca em sua secção 2 (backgroud) o processamento de streams [18, 19],
a preferencia de NIDS por anomalia em contraste aos NIDS por assinatura [30, 31, 32],
a variabilidade e evolução dos padrões de tráfego em redes de propósito geral [9, 11, 20],
a necessidade de atualização regular do modelo classificador [8, 9, 10, 20] e
o tratamento de eventos onde a confiança resultante da classificação é baixa [9, 12, 13].

Também destaca em sua secção 3 (MAWIFlow) 
que datasets adequados para NIDS são poucos devido o conjunto de qualidades que os mesmos
devem atender como realismo, validade, etiquetamento, grande variabilidade
e reprodutividade (disponibilidade pública) [8, 9, 10, 17, 38].

Para avaliar o desempenho de NIDS o dataset MAWIFlow é proposto. Originário do 
'Packet traces from WIDE backbone, samplepoint-F' composto por seções de captura de pacotes
diárias de 15 minutos de um link de 1Gbps entre Japão e EUA, com inicio em 2006 continuamente até hoje,
anonimizados [22], etiquetados por MAWILab [8].
Desse dataset original apenas os eventos de 2016 são utilizados e desses 158 atributos são extraídas
resultando em 7.9 TB de captura de pacotes. Além disso, os dados são estratificados [24] para redução
de seu tamanho a um centésimo mantendo as proporções de etiquetas (Ataque e Normal)
facilitando o compartilhamento e avaliação de NIDS além de atender as qualidades anteriormente mencionadas.

Com o dataset MAWIFlow original e reduzido foram avaliados quatro classificadores [42, 43, 44, 45]
da literatura em dois modos de operação quanto seus dados de treinamento
(ambos contendo uma semana de captura) o primeiro usando somente a primeira semana do ano e as demais
como teste e o segundo modo usando a semana anterior como treinamento e a seguinte como teste.
Demostrando, com 62 atributos, que a qualidade da classificação retrai com o tempo quando não há
atualização frequente do modelo classificador.
