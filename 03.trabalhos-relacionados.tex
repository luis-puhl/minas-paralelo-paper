\chapter{Trabalhos Relacionados}\label{cha:related}

% % discussão de 2020-02-01
% Aqueles que contenham:
%     - detecção de anomalia em streams
%     - detecção de intrusão em rede com processamento de streams
%     - BigFlow
% %/discussão

% cap 3: Trabalhos relacionados
%     - Artigos sobre o Minas
%     - outros que paralelizaram algoritmos de mineração de 
%        dados/streams alguns online (5-10 refs)
%     - implementação paralelas/distribuídas em dispositivos pequenos
% % 

Esse Capítulo trata dos trabalhos relacionados e estabelece o estado da arte
dos tópicos Detecção de Novidades em Fluxos de Dados, e 
Processamento Distribuído de Fluxos de Dados.

\section{Algoritmo MINAS e Algoritmos Derivados}

\newcommand{\cluster}{\emph{cluster}\xspace}
\newcommand{\clusters}{\emph{clusters}\xspace}
\newcommand{\dataset}{\emph{data set}\xspace}
\newcommand{\datasets}{\emph{data sets}\xspace}
% \clusters é um conjunto de \cluster feito de um \dataset.

O algoritmo MINAS, como já foi discutido, classifica exemplos e detecta
novidades em DS e considera em sua composição \emph{concept drift} e
\emph{concept evolution}, sendo capaz de classificar como extensão de classe
conhecida e reconhecer novas classes sem intervenção de especialista
\cite{Faria2016minas}. Neste trabalho, consideram-se algoritmos derivados
aqueles apresentados em trabalhos publicados após 2016 que estendem a
implementação original seguindo sua estrutura básica.

\subsection*{Algoritmo FuzzyND}

% FuzzyND
% $(n, \mathit{M}, \overline{CF1^x}, SSD^e, t, l)$
% $(n, LS, SS, t, l)$
% A nova estrutura contrapõem a estrutura original
% substituindo a soma linear dos elementos ($LS$) por  e $SS$ por $M$ e $$

O algoritmo FuzzyND, derivado do MINAS é proposto por \citeonline{DaSilva2018}.
FuzzyND incrementa o algoritmo inicial aplicando à ele teorias de
conjuntos \emph{fuzzy} pela modificação da representação dos \clusters.
A modificação afeta os métodos de construção de \clusters, classificação
de exemplos e detecção de novidades de acordo com a nova representação.

\acronym{F1M}{\emph{Macro F-Score}, acurácia }

A avaliação do algoritmo FuzzyND é feita por meio de experimentos usando 3 
\datasets sintéticos (\emph{MOA3}, \emph{RBF}, \emph{SynEDC})
e comparação com o MINAS.
O método de avaliação utilizado baseia-se na matriz de confusão incremental
descrita por \citeonline{Faria2016nd} extraindo dessa matriz duas métricas:
acurácia (\emph{Macro F-Score}) \cite{Sokolova2009} e
taxa de desconhecidos (\emph{UnkR}) \cite{Faria2016minas}.
Em geral o algoritmo FuzzyND detecta melhor novidades e, consequentemente,
é mais robusto à valores atípicos (\emph{outlier}) porém perde a capaciade
de reconhecer padrões recorrentes.


% Experiments were evaluated using the incremental confusion-matrix proposed by [27],
% recently been proposed [5]–[9]
% [5] T. Al-Khateeb, M. M. Masud, L. Khan, C. Aggarwal, J. Han, and B. Thuraisingham, “Stream classification with recurring and novel class detection using class-based ensemble,” in Data Mining (ICDM), 2012 IEEE 12th International Conference on. IEEE, 2012, pp. 31–40.
% [6] E. R. de Faria, A. C. P. de Leon Ferreira, J. Gama et al., “Minas: multiclass learning algorithm for novelty detection in data streams,” Data Mining and Knowledge Discovery, vol. 30, no. 3, pp. 640–680, 2016.
% [7] M. Masud, J. Gao, L. Khan, J. Han, and B. M. Thuraisingham, “Classification and novel class detection in concept-drifting data streams under time constraints,” IEEE Transactions on Knowledge and Data Engineering, vol. 23, no. 6, pp. 859–874, 2011.
% [8] M. M. Masud, Q. Chen, L. Khan, C. Aggarwal, J. Gao, J. Han, and B. Thuraisingham, “Addressing concept-evolution in concept-drifting data streams,” in Data Mining (ICDM), 2010 IEEE 10th International Conference on. IEEE, 2010, pp. 929–934.
% [9] Z. S. Abdallah, M. M. Gaber, B. Srinivasan, and S. Krishnaswamy, “Anynovel: detection of novel concepts in evolving data streams,” Evolving Systems, vol. 7, no. 2, pp. 73–93, 2016.
% [27] E. R. de Faria, I. R. Goncalves, J. Gama, A. C. P. de Leon Ferreira et al., “Evaluation of multiclass novelty detection algorithms for data streams,” IEEE Transactions on Knowledge and Data Engineering, vol. 27, no. 11, pp. 2961–2973, 2015.
% [28] M. Sokolova and G. Lapalme, “A systematic analysis of performance measures for classification tasks,” Information Processing & Manage- ment, vol. 45, no. 4, pp. 427–437, 2009.

\subsection*{Algoritmos MINAS-LC e MINAS-BR}

O algoritmo MINAS-LC é proposto por \citeonline{Costa2019thesis} e trata classificação
multi-rótulo porém não trata evoluções de conceito (novas classes).
AS alterações fundamentais são:
a representação de \cluster onde MINAS-LC troca a etiqueta, que era única, por uma multi-rótulo;
a transformação de problema aplicada ao conjunto de treinamento para transforma-lo de um
conjunto multi-rotulo para um conjunto multi-classe (simplificação)
em duas variações \emph{Label Powerset} e \emph{Pruned Sets} com
mineração de conjunto de itens frequentes.

% Este capítulo apresentou o método MultI-label learNing Algorithm for data Streams
% with Label Combination-based methods (MINAS-LC) e o MultI-label learNing Algorithm for data Streams with Binary Relevance transformation (MINAS-BR) para CMFCD com latência extrema de rótulos.
% O MINAS-LC lida com problemas apenas com mudanças de conceito. O seu modelo
% de decisão e composto por microgrupos multirrotulados sendo capaz de classificar exemplos em várias classes simultaneamente e evoluir ao longo do fluxo de dados. Foram propostas duas variações do método: utilizando o método de transformação de problema Label Powerset (LP) e, utilizando o método Pruned Sets (PS) com mineração de conjunto de itens frequentes. O MINAS-BR lida com problemas tanto com mudanças de conceito, como com evo-
% luções de conceito. Ele possui um conjunto de modelos de decisão, um para cada classe do problema. Esses modelos de decisão podem ser entendidos adaptando-se às mudanças de con- ceito, ou novos modelos de decisão podem ser criados, adaptando-se às evoluções de conceito. O próximo capítulo apresenta os experimentos realizados envolvendo os dois métodos
% propostos neste trabalho.

Já o trabalho de \citeonline{Costa2019}, estende o algoritmo original para que
classifique um exemplo com uma ou mais etiquetas usando a transformação
\emph{Binary Relevance} propondo o algoritmo MINAS-BR.
O algoritmo modifica a representação do modelo, originalmente conjunto de \clusters, para
um grupo de \clusters por classe (etiqueta).
Também modifica o método de agrupamento substituindo a inicialização do 
algoritmo \emph{K-means}, originalmente aleatória, pelo algoritmo 
\emph{Leader Incremental Clustering} \cite{Vijaya2004505}.

% as 4CRE-V13, 4CRE-V24 e 5CVT5 6 foram geradas originalmente em Souza et al. (2015b)
% SOUZA, V. M. A.; SILVA, D. F.; GAMA, J.; BATISTA, G. E. A. P. A. Data stream classification guided by clustering on nonstationary environments and extreme verification latency. In: Procee- dings ofSIAM International Conference on Data Mining (SDM). [S.l.: s.n.], 2015. p. 873–881. Citado 4 vezes nas páginas 17, 65, 87 e 89.

O algoritmo MINAS-BR também é experimentalmente avaliado com 4 \emph{data sets} sintéticos:
\emph{MOA-3C-5C-2D}, \emph{MOA-5C-7C-2D}, \emph{MOA-5C-7C-3} da ferramenta  MOA \cite{MOA}
e \emph{4CRE-V2}
\footnote{A versão original do \dataset 4CRE-V2 está disponível em https://sites.google.com/site/nonstationaryarchive/home}
gerados pelo método \emph{Radial Basis Function} \cite{souza2015}.
MINAS-BR é comparado com 7 algoritmos da literatura também disponíveis na ferramenta
MOA \cite{MOA},
diferente da avaliação do FuzzyND que compara diretamente com MINAS.
Os 7 algoritmos são divididos em dois grupos: 3 com acesso às etiquetas corretas para
atualização do modelo e com a técnica ADWIN (\emph{ADaptive WINdowing}) para detectar
mudanças de conceito (\emph{Concept Drift}); 4 algoritmos sem acesso às etiquetas corretas,
ou seja, sem \emph{feedback} externo, mesma condição do MINAS-BR.

% Esse trecho parece mais fundamentação.

A avaliação elencada por \citeonline{Costa2019} leva em consideração que as classes
contidas no conjunto de testes podem não ter correlação direta com os padrões identificados
pelos algoritmos.
Para tratar a divergência, uma estratégia baseada em proposta anterior por
\citeonline{Faria2016nd} é apresentada com alterações para exemplos multi-rótulo.
A estratégia é executada na fase de classificação seguindo as regras:
\begin{enumerate*}
    \item após o consumo do exemplo $X_n$;
    \item para todo padrão $P_i$ (etiqueta atribuída) identificado sem associação até o momento;
    \item com classes novidade $y_j$ (etiqueta real) presentes em exemplos antes $X_n$;
    \item preenche-se a tabela de contingência $\mathbf{T}_{(i,j)}$ relacionando padrão $P_i$ e classe $y_j$;
    \item calcula-se o grau de dependência $\mathit{F1}$ derivado da tabela de contingência
    $\mathit{F1}_{(i,j)} = f(\mathbf{T}_{(i,j)})$;
    \item valores $\mathit{F1}_{(i,j)} = 0$ são descartados;
    \item dentre os valores restantes: o padrão $P_i$ é associado à classe $y_j$
    se $\mathit{F1}_{(i,j)}$ é máximo.
\end{enumerate*}
Após associação entre padrões de novidade e classes novidade é possível calcular métricas tradicionais.

As métricas utilizadas por \citeonline{Costa2019} após a associação de classes e padrões são
as tradicionais taxa de desconhecidos (\emph{UnkRM}) e \emph{F1M}.
Os resultados apresentados indicam que MINAS-BR capturou todas as novidades dos \datasets sintéticos de teste
e mostrou, como esperado, melhores métricas que os 4 algoritmos equivalentes da literatura ficando abaixo
dos 3 com \emph{feedback} externo.

Os trabalhos relacionados nessa Seção tem em comum muito além do algoritmo base,
tem também métricas de avaliação acurácia (\emph{Macro F-Score} e \emph{Macro F-Measure} F1M)
e taxa de desconhecidos, aplicadas com devido tratamento.
Também é comum entre eles o uso de \datasets sintéticos.
Outro potencial não explorado do MINAS é em aplicações de reais, ou seja,
consumindo além de \datasets reais, fluxos realistas em ambientes simulados ou reais porém
considerando uso de recursos computacionais.

Observando a arquitetura dos algoritmos abordados, todas são extremamente semelhantes:
a fase offline centrada no processo de agrupamento e criação de modelo;
a fase online dividida em classificação (com atualização das estatísticas do modelo)
e detecção de padrões, onde novamente o processo de agrupamento é central.
Portanto, apesar de outros trabalhos expandirem o algoritmo com diferentes técnicas, seu
núcleo continua relevante \cite{DaSilva2018,DaSilva2018thesis,Costa2019}\footnote{
Propostas de modificação do algoritmo MINAS estão longe de serem exauridas.
Não cabe ao presente trabalho expandir e validar conceitos de aprendizagem de maquina
porém alguns exemplos mencionados ainda não abordados são: \begin{enumerate*}[label={\alph*)}]
    \item diferentes métodos de cálculo de distância entre pontos além da distância euclidiana; 
    \item a mudança de representação de \clusters, atualmente hiper-esferas, para hiper-cubos
    para tratar \datasets onde as características representadas
    pelas dimensões são completamente independentes;
    \item um modo interativo onde o \cluster é formado, mostrado ao especialista
    que o classifica como inválido (ruido ou não representativo) ou válido, podendo conter
    uma ou mais classes e, se conter mais que uma classe corte em grupos menores até conter somente
    uma classe;
    \item ainda considerando interação com especialista, a possibilidade dele injetar 
    um exemplo não pertencer à uma classe, ou seja, marcar o exemplo como não
    pertencente à uma classe para manter ele na memória de desconhecidos e, eventualmente forçar
    criação de um \cluster que represente uma classe geometricamente próxima mas semanticamente distinta;
    \item na fase \emph{offline} a verificação de sobreposição de \clusters pertencentes à
    classes distintas e tratamento adequado.
\end{enumerate*} 
}.

% >>>>>>> master
% \citeonline{DaSilva2018}

% \citeonline{Costa2019} estende o algoritmo original na sua capacidade de
% classificar um exemplo com uma ou mais etiquetas usando a transformação
% \emph{Binary Relevance}. Essa versão modificada é testada e comparada com 7
% algoritmos por meio de experimentos com 4 \emph{data sets} sintéticos gerados
% pelo método \emph{Radial Basis Function}. Os 7 algoritmos e o método estão
% disponíveis na ferramenta MOA \cite{MOA}.
% w

% Apesar de outros trabalhos expandirem o algoritmo com diferentes técnicas, seu
% núcleo continua relevante \cite{DaSilva2018,DaSilva2018thesis,Costa2019}.
% <<<<<<

\section{AnyNovel}
\nota{Incompleto}

\nota{também é da mesma classe do minas porém \citeonline{Cassales2019a} destaca um
desempenho inferior para o \emph{data set} testado.}
%  a atividade de detecção de intrusão

\citeonline{abdallah2016anynovel}

\section{Catraca Lopez2018}
\nota{Incompleto}

\cite{Lopez2018}

% A monitoring and threat detection system using stream processing as a virtual function for Big Data

% A detecção tardia de ameaças de segurança causa um significante aumento no
% risco de danos irreparáveis, impossibilitando qualquer tentativa de defesa.
% Como consequência, a detecção rápida de ameaças em tempo real é essencial
% para a ad- ministração de segurança. Além disso, A tecnologia de
% virtualização de funções de rede (Network Function Virtualization - NFV)
% oferece novas oportunidades para soluções de segurança eficazes e de baixo
% custo. Propomos um sistema de detecção de ameaças rápido e eficiente,
% baseado em algoritmos de processamento de fluxo e de aprendizado de máquina. As
% principais contribuições deste trabalho são: 
% i) um novo sistema de monitoramento e detecção de ameaças baseado no processamento de fluxo; 
% ii) dois conjuntos de dados, o primeiro é um conjunto de dados sintético de
% segurança contendo tráfego suspeito e malicioso, e o segundo corresponde a uma
% semana de tráfego real de um operador de telecomunicações no Rio de Janeiro,
% Brasil; 
% iii) um algoritmo de pré-processamento de dados composto por um algoritmo
% de normalização e um algoritmo para seleção rápida de
% características com base na correlação entre variáveis;
% iv) uma função de
% rede virtualizada em uma plataforma de código aberto para fornecer um serviço
% de detecção de ameaças em tempo real;
% v) posicionamento quase perfeito de
% sensores através de uma heurística proposta para posicionamento estratégico
% de sensores na infraestrutura de rede, com um número mínimo de sensores; e,
% finalmente, 
% vi) um algoritmo guloso que aloca sob demanda uma sequência de 
% funções de rede virtual.

\section{BigFlow}

\nota{Incompleto}

%  notas de bigflow
% Table 1
% Network-level feature set used in the experiments throughout this work [18].
% Types: Host-based (Host to All), Flow-based (Source to Destination, Destination to Source, Both)
% Features:\\
%     - Number of Packets,\\
%     - Number of Bytes,\\
%     - Average Packet Size,\\
%     - Percentage of Packets (PSH Flag),\\
%     - Percentage of Packets (SYN and FIN Flags),\\
%     - Percentage of Packets (FIN Flag),\\
%     - Percentage of Packets (SYN Flag),\\
%     - Percentage of Packets (ACK Flag),\\
%     - Percentage of Packets (RST Flag),\\
%     - Percentage of Packets (ICMP Redirect Flag),\\
%     - Percentage of Packets (ICMP Redirect Flag),\\
%     - Percentage of Packets (ICMP Time Exceeded Flag),\\
%     - Percentage of Packets (ICMP Unreachable Flag),\\
%     - Percentage of Packets (ICMP Other Types Flag),\\
%     - Throughput in Bytes,\\
%     - Protocol\\

BigFlow destaca em sua secção 2 (backgroud) o processamento de streams [18, 19],
a preferencia de NIDS por anomalia em contraste aos NIDS por assinatura [30, 31, 32],
a variabilidade e evolução dos padrões de tráfego em redes de propósito geral [9, 11, 20],
a necessidade de atualização regular do modelo classificador [8, 9, 10, 20] e
o tratamento de eventos onde a confiança resultante da classificação é baixa [9, 12, 13].

Também destaca em sua secção 3 (MAWIFlow) 
que data sets adequados para NIDS são poucos devido o conjunto de qualidades que os mesmos
devem atender como realismo, validade, etiquetamento, grande variabilidade
e reprodutividade (disponibilidade pública) [8, 9, 10, 17, 38].

% discutir somente técnicas e estratégias
% dão dê aval aos dados, especialmente os de 'venda' do artigo.

Para avaliar o desempenho de NIDS o data set MAWIFlow é proposto. Originário do 
'Packet traces from WIDE backbone, samplepoint-F' composto por seções de captura de pacotes
diárias de 15 minutos de um link de 1Gbps entre Japão e EUA, com inicio em 2006 continuamente até hoje,
anonimizados [22], etiquetados por MAWILab [8].
Desse data set original apenas os eventos de 2016 são utilizados e desses 158 atributos são extraídas
resultando em 7.9 TB de captura de pacotes. Além disso, os dados são estratificados [24] para redução
de seu tamanho a um centésimo mantendo as proporções de etiquetas (Ataque e Normal)
facilitando o compartilhamento e avaliação de NIDS além de atender as qualidades anteriormente mencionadas.

Com o data set MAWIFlow original e reduzido foram avaliados quatro classificadores [42, 43, 44, 45]
da literatura em dois modos de operação quanto seus dados de treinamento
(ambos contendo uma semana de captura) o primeiro usando somente a primeira semana do ano e as demais
como teste e o segundo modo usando a semana anterior como treinamento e a seguinte como teste.
Demostrando, com 62 atributos, que a qualidade da classificação retrai com o tempo quando não há
atualização frequente do modelo classificador.

\nota{Falar do modelo de distribuição.}

\section{Arquitetura IDSA-IOT}

\section{Conjuntos de Dados e Referência de Desempenho para Detecção de Anomalia}

The Numenta Anomaly Benchmark

% - Airline, approximately 116 million flight arrival and departure records (cleaned and sorted) compiled by E. Ikonomovska. Reference: Data Expo 2009 Competition [1]. Access
% - Chess.com (online games) and Luxembourg (social survey) datasets compiled by I. Zliobaite. Access
% - ECUE spam 2 datasets each consisting of more than 10,000 emails collected over a period of approximately 2 years by an individual. Access from S.J.Delany webpage
% - Elec2, electricity demand, 2 classes, 45,312 instances. Reference: M. Harries, Splice-2 comparative evaluation: Electricity pricing, Technical report, The University of South Wales, 1999. Access from J.Gama webpage. Comment on applicability.
% - PAKDD'09 competition data represents the credit evaluation task. It is collected over a five-year period. Unfortunately, the true labels are released only for the first part of the data. Access
% - Sensor stream and Power supply stream datasets are available from X. Zhu's Stream Data Mining Repository. Access
% - SMEAR is a benchmark data stream with a lot of missing values. Environment observation data over 7 years. Predict cloudiness. Access
% - Text mining, a collection of text mining datasets with concept drift, maintained by I. Katakis. Access
% - Gas Sensor Array Drift Dataset, a collection of 13,910 measurements from 16 chemical sensors utilized for drift compensation in a discrimination task of 6 gases at various levels of concentrations. Access

\nota{concluir com um gap}

\nota{Discutir que o data set não é de borda de uma rede, portanto não tem
relevância para fog. Também discutir a classificação dos fluxos por endpoint
exarcerbando assim a distinção na fog com o efeito de particionamento dos dados.
Ou seja, um nó só vê e classifica os próprios dados.}
