% !TeX root = ./00.ppgcc-2020.tex

\chapter{Introdução}\label{cha:intro}

% \todo[inline]{IoT} 
A \iot conecta globalmente variados dispositivos, incluindo dispositivos móveis,
\emph{wearables}, eletrônicos domésticos, automóveis e sensores industriais.
Estes dispositivos podem, através da Internet, ser acessados, conectar-se a
outros dispositivos, servidores ou aplicações, tudo com mínima intervenção ou
supervisão humana
\cite{Tahsien2020,abane2019,haddadpajouh2019survey,Shanbhag2015}.
Outra característica de dispositivos \iot são os recursos computacionais
dimensionados, para propósitos específicos, que limitam a capacidade de computar
outras funções muito além da função original do dispositivo.

% O número de dispositivos categorizados como \iot na última década teve
% crescimento sem precedentes e, proporcionalmente, cresceu o volume de dados
% gerados por esses dispositivos.
% % 
% A análise desses dados pode trazer novos conhecimentos e tem sido um tema
% frequentemente abordado por trabalhos de pesquisa.

% \todo[inline]{Segurança} 
Segurança e privacidade é uma grande preocupação em \iot, especialmente em
relação aos dados pessoais como localização e saúde aos quais dispositivos tem
acesso \cite{sengupta2020comprehensive}.
Contudo, além dos dados de sensores e atuadores, esses dispositivos se
subvertidos, podem gerar tráfego maligno, como o gerado pela \emph{mirai botnet}
em 2016 \cite{Kambourakis2017,Kolias2017mirai}.
Nesse cenário, fatores que podem favorecer a subversão dos dispositivos incluem
a falta de controle sobre a origem do hardware e software embarcado nos
dispositivos bem como a menor frequência de atualizações deste software.
Além disso, estes dispositivos tem longa vida e, após implantação, convivem com
ampla diversidade de outros dispositivos, complicando a manutenção da rede
que os hospeda, aumentando sua superfície de ataque.

% \todo[inline]{NIDS} 
No contexto de segurança de redes \iot, ferramentas que facilitem a detecção e
resposta à ataques são necessárias.
Como a maioria dos dispositivos \iot tem recursos limitados (como bateria,
processamento, memória e comunicação), técnicas de segurança tradicionais
baseadas em algoritmos configuráveis não são usuais, com isso restam as
técnicas de observação de rede \cite{Zhou2017}.
Ferramentas como \nids observam o comportamento da rede e de seus dispositivos
e detectam possíveis ataques.
% 
% Analogamente, as mesmas técnicas de classificação podem ser aplicadas para os
% metadados gerados pela comunicação entre esses nós e a Internet, detectando
% alterações nos padrões de comunicação num \nids.

% \todo[inline]{ML} 
Para implementação de \nids, técnicas de \ml tem sido empregadas na detecção de
ataques a partir de ataques conhecidos ou descobrir novos ataques o mais cedo
possível \cite{buczak2016survey,mitchell2014survey}.
Apesar do uso promissor de \ml para segurança para sistemas \iot, muitos estudos
na literatura \cite{buczak2016survey,mitchell2014survey,Tahsien2020} são
limitados à métodos tradicionais de \ml.
Estes métodos tradicionais utilizam modelos estáticos para descrever e prever o
comportamento da rede que, por serem estáticos, não mantém confiabilidade quando
confrontados com a evolução de ataques \cite{Viegas2019,AndreoniLopez2019}.

% \todo[inline]{DS} 
Além das complicações de confiabilidade, a grande quantidade de dispositivos em
redes distantes gerando dados em volumes e velocidades consideráveis,
diretamente ligados às suas funções originais ou metadados produzidos como
subproduto, técnicas tradicionais que tratam grandes lotes não são aplicáveis.
Considerando fluxos contínuos de dados (\emph{Data Stream}), técnicas de
mineração de fluxos de dados (\emph{Data Stream Mining}) são amplamente
necessárias.
% 
Nesses cenários, essas técnicas são aplicadas, por exemplo, em problemas de
monitoramento e classificação de valores originários de sensores para tomada de
decisão tanto em nível micro, como na modificação de atuadores remotos, ou
macro, na otimização de processos industriais.
% 
% Quando aplicado a \nids, técnicas de mineração de fluxo de dados possuem
% vantagens como processamento em uma única leitura, reduzindo uso de memória e
% armazenamento possibilitando sua implementação em

% % processing traffic data with a single read;
% % working with limited memory (allowing the implementation in small devices
% % commonly employed in edge services);
% Produção de resultados 
% producing real-time response; and
% detecting novelty and changes in concepts already learned.

% \todo[inline]{ND} 
Dentre as técnicas de mineração de fluxo de dados, classificadores podem ser
utilizados para detectar padrões conhecidos e, em conjunto com algoritmos de
\nd, detectar novos padrões.
Essa capacidade de detectar novos padrões é relevante para \nids, onde novidades
na rede podem representar novas funcionalidades ou ataques por agentes
maliciosos, sem assinaturas existentes em bancos de dados de ataques conhecidos.
Outras características que fazem \nd atraente para \nids são a produção de
respostas imediatas e a detecção de novidades e mudança de conceitos já
conhecidos.

% \todo[inline]{Fog} 
Análises como \emph{Data Stream Mining} e \nd são geralmente implementadas sobre
o paradigma de computação na nuvem (\emph{Cloud Computing}) e, recentemente,
sobre paradigmas como computação em névoa (\emph{Fog Computing}).
Para \emph{fog}, além dos recursos em \emph{cloud}, são explorados os recursos
distribuídos pela rede desde o nó remoto até a \emph{cloud}.
Processos que dependem desses recursos são distribuídos de acordo com
características como sensibilidade à latência, privacidade, consumo
computacional ou energético.

Como elaborado, a aplicação de \nd para detecção de ameaças em fluxos de dados
originários de redes \iot dentro de \nids tem sido um ponto de interesse
\cite{Viegas2019,AndreoniLopez2019,DaCosta2019a}.
Este trabalho foca explora as características de implementação destas técnicas
em conjunto, focando serviços localizados na borda da rede, de maneira
distribuída, para uso em ambientes \iot.

% Given the recent \cite{Viegas2019,AndreoniLopez2019,DaCosta2019a} use of Data Stream Novelty
% Detection (\nd) in network data streams, this paper shows the effects of
% adapting these mechanisms to edge services for use in \iot environments.

\section{Motivação}\label{sec:motivo}

% A pesquisa recente de \citeonline{Tahsien2020} mostra que técnicas de \ml são
% uma alternativa promissora que pode prover ferramentas de segurança para redes
% \iot, fazendo elas mais confiáveis e acessíveis.
% A recent survey \cite{Tahsien2020} shows that ML based methods are a
% promising alternative which can provide potential security tools for the \iot
% network making them more reliable and accessible than before.

Um problema recente que une, em um único contexto, os métodos de computação em
névoa, processamento de fluxo de dados e detecção de novidades nesses fluxos é a
detecção de intrusão em redes de dispositivos \iot.
Para tratar esse problema, a arquitetura \arch, recentemente proposta por
\citeonline{Cassales2019a}, aplica ao problema algoritmos atuais de detecção de
novidades em fluxos, executando esses algoritmos em ambiente próximo aos
dispositivos e avaliando-os quanto à detecção de intrusão.

Na arquitetura proposta, \citeonline{Cassales2019a} avaliou os algoritmos
ECSMiner \cite{Masud2010ECSMiner}, AnyNovel \cite{Abdallah2016anynovel} e MINAS
\cite{Faria2016minas}, sendo que o último mostrou resultados promissores.
% A arquitetura proposta foi avaliada com o conjunto de dados (\emph{data set})
% \emph{Kyoto 2006+}, composto de dados coletados de 348 \emph{Honeypots}
% (máquinas isoladas equipadas com diversos softwares com vulnerabilidades
% conhecidas expostas à Internet com propósito de atrair ataques) de 2006 até
% dezembro 2015.
% O \emph{data set} \emph{Kyoto 2006+} contém 24 atributos, 3 etiquetas atribuídas por
% detectores de intrusão comerciais e uma etiqueta
% distinguindo o tráfego entre normal, ataque conhecido e ataque desconhecido
% \cite{Cassales2019a}.

% \notahl{ Por que a necessidade paralelismo? O que falta que justifica essa
% avaliação? Tempo de resposta?\\
% Necessidade de tratar os dados próximo a onde são produzidos? Uso da cloud? ….}
Contudo, o algoritmo MINAS ainda não foi implementado e avaliado com paralelismo,
multi-processamento ou distribuição computacional, que são necessários para
tratar fluxos de dados com grandes volumes e velocidades.
% \notafa{Por que isso é importante. Acho que convém ressaltar a importancia da sua
% proposta. O que o MINAS original não trata, em quais cenários ele apresenta um
% gargalo? como ele está dividido?}
% O tratamento de distribuição em ambiente \fog é essencial para aplicação deste
O tratamento de distribuição em ambiente \fog é essencial para aplicação deste
algoritmo ao problema de detecção de intrusão em redes \iot, pois esta aplicação
requer tempo de resposta mínimo e mínima comunicação entre nós distantes, como
aqueles na borda e na nuvem.
Ainda observando o algoritmo MINAS, destaca-se a possível divisão em três partes
semi-independentes, sendo elas treinamento, classificação e detecção de
novidade; a classificação é o elemento central cujos resultados são utilizados
para a identificação de intrusões.

Ainda no contexto de \nd como método de detecção de intrusão, outras propostas
tratam do caso de fluxos com grandes volumes e velocidades, como é o caso de
\citeonline{Viegas2019}, que apresenta o \emph{BigFlow} no intuito de detectar
intrusão em redes do tipo \emph{10 Gigabit Ethernet}, que podem produzir um
volume considerável.
% atualmente impossível de ser processado em um único núcleo de processador
% (\emph{single-threaded}).
Essa implementação foi feita sobre uma plataforma distribuída processadora de
fluxos (\emph{Apache Flink}) executada em um cluster com até $10$ nós de trabalho,
cada um com $4$ núcleos de processamento, totalizando $40$ núcleos, para atingir
taxas de até $10,72 \ Gbps$.

% Além de apresentar uma implementação, \citeonline{Viegas2019} também apresenta o
% \emph{data set} \emph{MAWIFlow}. Esse conjunto é derivado do ponto de coleta
% F, localizado em um elo de comunicação entre o Japão e os EUA (\emph{Backbone})
% com capacidade de $1\ Gbps$, diariamente 15 minutos são capturados desde 2006
% sob supervisão de \citeonline{mawiSamplepointF} \cite{Fontugne2010}. O conjunto
% \emph{MAWIFlow} limita-se às coletas de 2016 ($7.9\ TB$) e estratificado para
% $1\%$ desse tamanho facilitando compartilhamento e avaliação por outros
% softwares. Esse conjunto contempla 158 atributos de nós e fluxos e etiquetado
% por \citeonline{Fontugne2010}.

% \acronym{Drift}{\emph{Concept Drift}, Deriva conceitual: variação temporal de um conceito conhecido}
% \acronym{Evolution}{\emph{Concept Evolution}, Conceitos emergentes: conceitos não  }

% O sistema \emph{BigFlow} é composto de dois estágios: extração de
% características (estatísticas de tráfego da rede) e aprendizado confiável de
% fluxo. O segundo estágio implementa um algoritmo de detecção de novidade
% utilizando classificadores já estabelecidos na biblioteca \emph{Massive Online Analysis framework} (MOA) \cite{MOA} com
% adição de um módulo de verificação que armazena valores classificados com baixa
% confiança para serem manualmente avaliados por um especialista \cite{Viegas2019}.
% A escolha dessa
% abordagem não é nova e visa tratar nuances do problema abordado como
% variação temporal de conceitos conhecidos (\emph{Concept Drift}) e
% conceitos emergentes (\emph{Concept Evolution})
% \cite{Faria2016ndds}.
% Essas nuances causam redução de acurácia durante a
% avaliação inicial dos algoritmos tradicionais e devidamente mitigada com a
% atualização constante do modelo. Esses problemas são amplamente abordados e
% tratados em outros algoritmos como o MINAS \cite{Faria2016minas}.

Os trabalhos de \citeonline{Cassales2019a} e \citeonline{Viegas2019} abordam
detecção de intrusão em redes utilizando algoritmos de ND em DS, porém com
perspectivas diferentes.
O primeiro investiga \emph{IoT} e processamento em \emph{fog} e baseia-se em um
algoritmo genérico de detecção de novidade.
O segundo trabalho trata de \emph{backbones} e processamento em \emph{cloud} e
implementa o próprio algoritmo de detecção de novidade.
Essas diferenças deixam uma lacuna onde, de um lado, tem-se uma
arquitetura mais adequada para o ambiente \emph{fog} com um algoritmo estado da arte de
detecção de novidades, porém sem paralelismo e.
Do outro lado da lacuna, tem-se um sistema
escalável de alto desempenho porém almejando outro ambiente (\emph{cloud}) e
com um algoritmo menos preparado para os desafios de detecção de 
novidades.
% 
% novidades.

% \nota{
% \\ deixar mais claro o contraste entre cassales e bigflow
% \\ abordar o **gap** no qual a minha pesquisa entra
% \\ mostrar que modelo bigflow não considera fog
% \\ arquiteutra distribuida em fog com minas e alto desempenho bigflow
% }

A proposta, aqui chamada \mfog, adapta a arquitetura \arch \cite{Cassales2019a}
empregando o algoritmo de \nd \minas \cite{Faria2016minas}, tornando-o capaz
de ser executado em um sistema distribuído composto de pequenos computadores com
recursos limitados, alocados na borda da rede próximos dos dispositivos \iot.
Utilizando a nova implementação do algoritmo \minas, avalia-se experimentalmente
como a distribuição afeta a capacidade do sistema de detectar mudanças
(novidades) nos padrões de tráfego e o impacto na eficiência computacional.
Por fim, algumas estratégias e políticas para configuração do sistema de
detecção de novidades em fluxo de dados são discutidas.

% Our proposal, called \mfog, adapted the \arch architecture \cite{Cassales2019a}
% using the \nd algorithm \minas \cite{Faria2013Minas,Faria2016minas}, making it
% suitable to run on a distributed system composed of small devices with limited
% resources on the edge of the network.
% Using our newer version of the \minas algorithm, we have experimentally
% evaluated how the distribution affects the capability to detect changes
% (novelty) in traffic patterns and its impact on the computational efficiency.
% Finally, some distribution strategies and policies for the data stream novelty
% detection system are discussed.

\section{Objetivos}\label{sec:objetivos}

Como estabelecido na \refsec{motivo}, a lacuna no estado da arte observada é
a ausência de uma implementação de algoritmo de detecção de
novidades que trate adequadamente os desafios de fluxo de dados contínuos
(como volume e velocidade do fluxo, evolução e mudança de conceito)
e considere o ambiente de computação em névoa aplicada à detecção de
intrusão.
Seguindo a comparação entre algoritmos desse gênero realizada por
\citeonline{Cassales2019a}, esta pesquisa escolheu investigar o algoritmo MINAS \cite{Faria2016minas}
para receber o tratamento necessário para adequá-lo ao ambiente de névoa e para
fluxos de grandes volumes e velocidades.

Portanto, seguindo os trabalhos do Grupo de Sistemas Distribuídos e Redes
(GSDR) da Universidade Federal de São Carlos (UFSCar), propõem-se a construção
de uma 

aplicação que implemente o algoritmo MINAS
de maneira escalável e distribuível para ambientes de computação em névoa e a avaliação
dessa implementação com experimentos baseados na literatura usando conjunto de dados
públicos relevantes.
O resultado esperado é uma implementação compatível em qualidade de
classificação ao algoritmo MINAS e passível de ser distribuída em um ambiente
de computação em névoa aplicado à detecção de intrusão.

Com foco no objetivo geral, alguns objetivos específicos são propostos:

\begin{itemize}

    \item Implementar o algoritmo MINAS de maneira distribuída sobre uma
    plataforma de processamento distribuída de fluxos de dados;

    \item Avaliar a qualidade de detecção de intrusão em ambiente distribuído 
    conforme a arquitetura \arch;
    
    \item Avaliar o desempenho da implementação em ambiente de computação em névoa.

\end{itemize}

% There is a need for real-time stream processing, as data is arriving as
% continuous flows of events; for example, cars in motion emitting GPS signals;
% financial transactions; the interchange of signals between cellphone towers; web
% traffic including things like session tracking and understanding user behavior
% on websites; and measurements from industrial sensors.
% https://dzone.com/articles/streaming-in-spark-flink-and-kafka-1

\section{Proposta Metodológica}

Para cumprir os objetivos citados na \refsec{objetivos}, foi identificada a necessidade
de um processo exploratório seguido de experimentação. Tal processo inclui a
revisão da literatura, tanto acadêmica quanto técnica, seguida da experimentação
através de implementação de aplicação e testes.

O foco da revisão da literatura acadêmica é em trabalhos que abordem
processamento de fluxos de dados, classificação de fluxo de dados, detecção de
novidades em fluxo de dados e processamento distribuído de fluxo de dados.
O objetivo da revisão é o estabelecimento do estado da arte desses assuntos,
de forma que alguns desses trabalhos sirvam para comparações e relacionamentos.
Além disso, desses trabalhos buscam-se métricas de qualidade de classificação
(por exemplo, taxa de falso positivo e matriz de confusão) e métricas de
escalabilidade (como taxa de mensagens por segundo e escalabilidade vertical ou
horizontal).

A revisão da literatura técnica será focada em plataformas, ferramentas e técnicas
para realizar a implementação proposta.
Portanto, são selecionadas plataformas de processamento distribuído de DS
e técnicas de aprendizado de máquina associadas a elas.
Dessa revisão também serão obtidas técnicas ou ferramentas necessárias
para extração das métricas de avaliação, bem como \emph{data sets}
públicos relevantes para detecção de novidades em DS.

Uma vez definidos o estado da arte, as ferramentas técnicas e os
\emph{data sets}, o passo seguinte é a experimentação.
Nesse passo, será desenvolvida uma aplicação na plataforma escolhida que, com base no
algoritmo MINAS \cite{Faria2016minas}, irá classificar e detectar novidades em DS.
Também nesse passo, a implementação será validada comparando os resultados de
classificação obtidos com os resultados de classificação do algoritmo original
MINAS.
Posteriormente, serão realizados experimentos com a implementação e variações em \emph{data sets} e
cenários de distribuição em \emph{fog}, coletando as métricas de classificação e escalabilidade.

Ao final, a aplicação, resultados, comparações e discussões serão publicados
nos meios e formatos adequados, como repositórios técnicos, eventos ou revistas
acadêmicas.

\section{Organização do trabalho}

O restante desse trabalho segue a estrutura:
Capítulo \ref{cha:fundamentos} aborda conceitos teóricos e técnicos que embasam
esse trabalho;
Capítulo \ref{cha:related} enumera e discute trabalhos relacionados e estabelece
o estado da arte do tema detecção de novidade em fluxos de dados e seu
processamento;
Capítulo \ref{cha:proposta} descreve a proposta de implementação, discute as
escolhas de plataformas e resultados esperados.
Também são discutidos no Capítulo \ref{cha:implementacao} os desafios e
resultados preliminares encontrados durante o desenvolvimento do trabalho.
Capítulo \ref{cha:results} 
Capítulo \ref{cha:final} adiciona considerações gerais sobre o trabalho e seus
resultados.
