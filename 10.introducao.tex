% !TeX root = ./00.main.tex
\chapter{Introdução}\label{cha:intro}

\acronym{IoT}{Internet of Things, Internet das Coisas}
\newcommand{\iot}{IoT\xspace}

A Internet das Coisas (\emph{Internet of Things} - IoT) é um sistema global de
dispositivos (máquinas, objetos físicos ou virtuais, sensores, atuadores e
pessoas) com capacidade de comunicação pela Internet, sem depender de
interação com interface humano-computador tradicional.
Outra característica de dispositivos \iot são os 
\notahl{?}\hlhl{recursos computacionais dimensionados},
% por vezes omitindo sistema operacional e outros sistemas,
para
propósitos específicos que limitam a capacidade de computar outras funções além
da função original do dispositivo.
O número de dispositivos categorizados como IoT na última década teve
crescimento sem precedentes e, proporcionalmente, cresceu o volume de dados
gerados por esses dispositivos.
A análise desses dados pode trazer novos conhecimentos e tem sido um tema frequentemente
abordado por trabalhos de pesquisa.
Contudo, além dos dados de sensores e atuadores, esses dispositivos se subvertidos,
podem gerar tráfego maligno
% à sociedade
, como o gerado pela
\emph{botnet} mirai em 2016 \cite{Kambourakis2017}.
Nesse cenário, fatores que podem favorecer a subversão dos dispositivos incluem a
falta de controle sobre a origem do hardware e software embarcado nos
dispositivos, além da falta das cruciais atualizações de segurança.

% \notahl{não está clara qual é a implicação de serem dispositivos pouco robustos… \\
% … contendo implementações comumente apenas dedicada a propósitos específicos?\\
% … comumente sem mecanismos de segurança robustos?}

\acronym{DS}{\emph{Data Stream}, Fluxo de Dados}
\acronym{NIDS}{\emph{Network Intrusion Detection System}, sistema de detecção de intrusão em redes}
\newcommand{\ds}{DS\xspace}

Com milhares de dispositivos em redes distantes gerando dados (diretamente
ligados às suas funções originais ou metadados produzidos como subproduto) em
volumes e velocidades consideráveis, formando fluxos contínuos de dados (\emph{Data
Stream} - DS), técnicas de mineração de fluxos de dados
(\emph{Data Stream Mining}) são amplamente necessárias.
% parágrafo anterior marcado com '?', muito longo será? ou fora de contexto?
Nesses cenários, essas técnicas são
aplicadas, por exemplo, em problemas de monitoramento e classificação de valores
originários de sensores para tomada de decisão tanto em nível micro, como na
modificação de atuadores remotos, ou macro, na otimização de processos
industriais.
Analogamente, as mesmas técnicas de classificação podem ser aplicadas para os
metadados gerados pela comunicação entre esses nós e a Internet, detectando
alterações nos padrões de comunicação num serviço de detecção de intrusão
(\emph{Network Intrusion Detection System}, NIDS).

% \nota{ou de alterações nos padrões de comunicação}

% \nota{essa Introdução deixa claro que é IDS?}

\acronym{ND}{\emph{Novelty Detection}, Detecção de Novidade}
\newcommand{\nd}{ND\xspace}

\hlke{Técnicas} de \emph{Data Stream Mining} envolvem mineração de dados
\notake{parágrafo marcado}
(\emph{Data Mining}), aprendizado de
máquina (\emph{Machine Learning}) e, recentemente, detecção de novidades
(\emph{Novelty Detection}, \nd).
\nd, além de classificar em modelos conhecidos, permite classificar novos padrões,
já que trata problemas como evolução e deriva de conceito, que são característicos
de fluxos contínuos de dados e, consequentemente, permite que ações sejam tomadas corretamente mesmo
em face a padrões nunca vistos.
Essa capacidade é relevante em especial para o
exemplo de detecção de intrusão, onde novidades na rede podem distinguir novas
funcionalidades (entregues aos dispositivos após sua implantação em campo) de
ataques por agentes externos, sem assinaturas existentes em bancos de
dados de ataques conhecidos.

% finalizo com "por exemplo"?

Análises como \emph{Data Stream Mining} e \nd são \hlhl{tradicionalmente} implementadas
sobre o paradigma de computação na nuvem
(\emph{Cloud Computing}) e, recentemente, sobre paradigmas
\notahl{O que (referências) reslpalda essa afirmação?}
como \hlhl{computação em névoa}
(\emph{Fog Computing}). Para \emph{fog}, além dos recursos em \emph{cloud}, são
explorados os recursos distribuídos pela rede desde o nó remoto até a
\emph{cloud}. Processos que dependem desses recursos são distribuídos de acordo
com características como sensibilidade à latência, privacidade,
consumo computacional ou energético.

\section{Motivação}\label{sec:motivo}

\newcommand{\idsiot}{IDSA-IoT\xspace}
\newcommand{\fog}{\emph{fog computing}\xspace}

Um problema recente que une, em um único contexto, os métodos de computação em
névoa, processamento de fluxo de dados e detecção de novidades nesses fluxos é a
detecção de intrusão em redes de dispositivos \iot.
Para tratar esse problema, a arquitetura \idsiot, recentemente proposta por
\citeonline{Cassales2019a}, aplica ao problema algoritmos atuais de detecção de
novidades em fluxos, executando esses algoritmos em ambiente próximo aos
dispositivos e avaliando-os quanto à detecção de intrusão.

Na arquitetura proposta, \citeonline{Cassales2019a} avaliou os algoritmos
ECSMiner \cite{Masud2010ECSMiner}, AnyNovel \cite{Abdallah2016anynovel} e MINAS
\cite{Faria2016minas}, sendo que o último mostrou resultados promissores.
A arquitetura proposta foi avaliada com o conjunto de dados (\emph{data set})
\emph{Kyoto 2006+}, composto de dados coletados de 348 \emph{Honeypots}
(máquinas isoladas equipadas com diversos softwares com vulnerabilidades
conhecidas expostas à Internet com propósito de atrair ataques) de 2006 até
dezembro 2015.
O \emph{data set} \emph{Kyoto 2006+} contém 24 atributos, 3 etiquetas atribuídas por
detectores de intrusão comerciais e uma etiqueta
distinguindo o tráfego entre normal, ataque conhecido e ataque desconhecido
\cite{Cassales2019a}.

% \notahl{ Por que a necessidade paralelismo? O que falta que justifica essa
% avaliação? Tempo de resposta?\\
% Necessidade de tratar os dados próximo a onde são produzidos? Uso da cloud? ….}
Contudo, o algoritmo MINAS ainda não foi implementado e avaliado com paralelismo,
multi-processamento ou distribuição computacional, que são necessários para
tratar fluxos de dados com grandes volumes e velocidades.
O tratamento de distribuição em ambiente \fog é essencial para aplicação deste
algoritmo ao problema de detecção de intrusão em redes \iot, pois esta aplicação
requer tempo de resposta mínimo e mínima comunicação entre nós distantes, como
aqueles na borda e na nuvem.
Ainda observando o algoritmo MINAS, destaca-se a possível divisão em três partes
semi-independentes, sendo elas treinamento, classificação e detecção de
novidade; a classificação é o elemento central cujos resultados são utilizados
para a identificação de intrusões.

Ainda no contexto de \nd como método de detecção de intrusão,
outras propostas tratam do caso de fluxos com grandes volumes e velocidades, como é o caso
de \citeonline{Viegas2019}, que apresenta o \emph{BigFlow} no intuito de detectar
intrusão em redes do tipo \emph{10 Gigabit Ethernet}, que podem produzir um volume considerável,
atualmente impossível de ser processado em um único núcleo de processador
(\emph{single-threaded}).
Essa implementação foi feita sobre uma plataforma
distribuída processadora de fluxos (\emph{Apache Flink}) executada em um cluster
com até 10 nós de trabalho, cada um com 4 núcleos de processamento, totalizando
40 núcleos, para atingir taxas de até $10,72 \ Gbps$.

% Além de apresentar uma implementação, \citeonline{Viegas2019} também apresenta o
% \emph{data set} \emph{MAWIFlow}. Esse conjunto é derivado do ponto de coleta
% F, localizado em um elo de comunicação entre o Japão e os EUA (\emph{Backbone})
% com capacidade de $1\ Gbps$, diariamente 15 minutos são capturados desde 2006
% sob supervisão de \citeonline{mawiSamplepointF} \cite{Fontugne2010}. O conjunto
% \emph{MAWIFlow} limita-se às coletas de 2016 ($7.9\ TB$) e estratificado para
% $1\%$ desse tamanho facilitando compartilhamento e avaliação por outros
% softwares. Esse conjunto contempla 158 atributos de nós e fluxos e etiquetado
% por \citeonline{Fontugne2010}.

% \acronym{Drift}{\emph{Concept Drift}, Deriva conceitual: variação temporal de um conceito conhecido}
% \acronym{Evolution}{\emph{Concept Evolution}, Conceitos emergentes: conceitos não  }

% O sistema \emph{BigFlow} é composto de dois estágios: extração de
% características (estatísticas de tráfego da rede) e aprendizado confiável de
% fluxo. O segundo estágio implementa um algoritmo de detecção de novidade
% utilizando classificadores já estabelecidos na biblioteca \emph{Massive Online Analysis framework} (MOA) \cite{MOA} com
% adição de um módulo de verificação que armazena valores classificados com baixa
% confiança para serem manualmente avaliados por um especialista \cite{Viegas2019}.
% A escolha dessa
% abordagem não é nova e visa tratar nuances do problema abordado como
% variação temporal de conceitos conhecidos (\emph{Concept Drift}) e
% conceitos emergentes (\emph{Concept Evolution})
% \cite{Faria2016nd}.
% Essas nuances causam redução de acurácia durante a
% avaliação inicial dos algoritmos tradicionais e devidamente mitigada com a
% atualização constante do modelo. Esses problemas são amplamente abordados e
% tratados em outros algoritmos como o MINAS \cite{Faria2016minas}.

Os trabalhos de \citeonline{Cassales2019a} e \citeonline{Viegas2019} abordam
detecção de intrusão em redes utilizando algoritmos de ND em DS, porém com
perspectivas diferentes.
O primeiro investiga \emph{IoT} e processamento em \emph{fog} e baseia-se em um
algoritmo genérico de detecção de novidade.
O segundo trabalho trata de \emph{backbones} e processamento em \emph{cloud} e
implementa o próprio algoritmo de detecção de novidade.
Essas diferenças deixam uma lacuna onde, de um lado, tem-se uma
arquitetura mais adequada para o ambiente \emph{fog} com um algoritmo estado da arte de
detecção de novidades, porém sem paralelismo e, de outro, tem-se um sistema
escalável de alto desempenho porém almejando outro ambiente (\emph{cloud}) e
com um algoritmo menos preparado para os desafios de detecção de novidades.

% \nota{
% \\ deixar mais claro o contraste entre cassales e bigflow
% \\ abordar o **gap** no qual a minha pesquisa entra
% \\ mostrar que modelo bigflow não considera fog
% \\ arquiteutra distribuida em fog com minas e alto desempenho bigflow
% }

\section{Objetivos}\label{sec:objetivos}

% \nota{
%     a ausencia nao é de um algoritmo de deteccao de novidade com paralelismo?\\
%     é, mas tem o contexto de ids agora
% }

Como estabelecido na \refsec{motivo}, a lacuna no estado da arte observada é
a \hlhl{ausência de uma} implementação de algoritmo de detecção de
novidades que trate adequadamente os desafios de fluxo de dados contínuos
(como volume e velocidade do fluxo, evolução e deriva de conceito)
e considere o ambiente de computação em névoa aplicada à detecção de
intrusão.
Seguindo a comparação entre algoritmos desse gênero realizada por
\citeonline{Cassales2019a}, esta pesquisa escolheu investigar o algoritmo MINAS \cite{Faria2016minas}
para receber o tratamento necessário para adequá-lo ao ambiente de névoa e para
fluxos de grandes volumes e velocidades.

% \nota{GSDR da universidade federal de sao carlos...}

Portanto, seguindo os trabalhos do Grupo de Sistemas Distribuídos e Redes
(GSDR) da Universidade Federal de São Carlos (UFSCar), propõem-se a construção
de uma aplicação que implemente o algoritmo MINAS
de maneira escalável e distribuível para ambientes de computação em névoa e a avaliação
dessa implementação com experimentos baseados na literatura usando conjunto de dados
públicos relevantes.
O resultado esperado é uma implementação compatível em qualidade de
classificação ao algoritmo MINAS e passível de ser distribuída em um ambiente
de computação em névoa aplicado à detecção de intrusão.

% Hermes
% \nota{Citar que o MINAS já foi testado para detecção de intrusão [Guilherme2019].}
% \nota{Falar do resultado bem resumuidamente}
% \nota{Porem a implementação não distribuída.}

% Com a avaliação inicial formulou-se a questão \emph{``quais resultados podem ser
% esperados de um sistema que implementa um algoritmo de detecção de novidades em
% fluxo de dados?"} que engloba sucintamente os temas abordados nesse trabalho.

% c. Proposta: executar em nós multi-core de maneira escalável,
% d. Resultado Esperado: minas com mesma qualidade porém escalável.
% 
% Linha de argumentação: discussão de localidade de dados e paralelização
% - implementar
% - paralelizar
% - avaliar
% - avaliar aprendizado local ou global trade-offs

Com foco no objetivo geral, alguns objetivos \hlke{secundários} são propostos:

% \nota{citar ferramentas e a escolha só depois do python e kafka}

% \nota{entre flink e spark, outro grupo de pesquisa já está explorando spark}

% \begin{itemize}
% \item
% \end{itemize}

% \nota{
%     estudar o algoritmo\\
%     fazer uma implementação\\
%     pegar datasets relevantes\\
%     comparar a corretude  com o sequencial\\
%     avaliar desempenho e escalabilidade
% }
\begin{itemize}

    \item Implementar o algoritmo MINAS de maneira distribuída sobre uma
    plataforma de processamento distribuída de fluxos de dados;

    \item Avaliar a qualidade de detecção de intrusão em ambiente distribuído 
    conforme a arquitetura \idsiot;
    
    \item Avaliar o desempenho da implementação em ambiente de computação em névoa.

\end{itemize}

% There is a need for real-time stream processing, as data is arriving as
% continuous flows of events; for example, cars in motion emitting GPS signals;
% financial transactions; the interchange of signals between cellphone towers; web
% traffic including things like session tracking and understanding user behavior
% on websites; and measurements from industrial sensors.
% https://dzone.com/articles/streaming-in-spark-flink-and-kafka-1

\section{Proposta Metodológica}

% (Metodologia) (como)

Para cumprir os objetivos citados na \refsec{objetivos}, foi identificada a necessidade
de um processo exploratório seguido de experimentação. Tal processo inclui a
revisão da literatura, tanto acadêmica quanto técnica, seguida da experimentação
através de implementação de aplicação e testes.

O foco da \notahl{estão nas referências?}\hlhl{revisão da literatura acadêmica} é em trabalhos que abordem
processamento de fluxos de dados, classificação de fluxo de dados, detecção de
novidades em fluxo de dados e processamento distribuído de fluxo de dados.
O objetivo da revisão é o estabelecimento do estado da arte desses assuntos,
de forma que alguns desses trabalhos sirvam para comparações e relacionamentos.
Além disso, desses trabalhos buscam-se métricas de qualidade de classificação
(por exemplo, taxa de falso positivo e matriz de confusão) e métricas de
escalabilidade (como taxa de mensagens por segundo e escalabilidade vertical ou
horizontal).

A revisão da literatura técnica será focada em plataformas, ferramentas e técnicas
para realizar a implementação proposta.
Portanto, são selecionadas plataformas de processamento distribuído de DS
e técnicas de aprendizado de máquina associadas a elas.
Dessa revisão também serão obtidas técnicas ou ferramentas necessárias
para extração das métricas de avaliação, bem como \emph{data sets}
públicos relevantes para detecção de novidades em DS.

Uma vez definidos o estado da arte, as ferramentas técnicas e os
\emph{data sets}, o passo seguinte é a experimentação.
Nesse passo, será desenvolvida uma aplicação na plataforma escolhida que, com base no
algoritmo MINAS \cite{Faria2016minas}, irá classificar e detectar novidades em DS.
Também nesse passo, a implementação será validada comparando os resultados de
classificação obtidos com os resultados de classificação do algoritmo original
MINAS.
Posteriormente, serão realizados experimentos com a implementação e variações em \emph{data sets} e
cenários de distribuição em \emph{fog}, coletando as métricas de classificação e escalabilidade.

Ao final, a aplicação, resultados, comparações e discussões serão publicados
nos meios e formatos adequados, como repositórios técnicos, eventos ou revistas
acadêmicas.

\section{Organização do trabalho}

O restante desse trabalho segue a estrutura:
\refcap{fundamentos} aborda conceitos teóricos e técnicos que embasam
esse trabalho;
\refcap{related} enumera e discute trabalhos relacionados e estabelece
o estado da arte do tema detecção de novidade em fluxos de dados e seu processamento;
\refcap{proposta} descreve a proposta de implementação, discute
as escolhas de plataformas e resultados esperados.
Também são discutidos no \refcap{proposta} os desafios e resultados preliminares encontrados
durante o desenvolvimento do trabalho.
\refcap{final} adiciona considerações gerais e apresenta o plano de trabalho
e cronograma até a defesa do mestrado.
