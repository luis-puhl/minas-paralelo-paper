% !TeX root = ./00.ppgcc-2020.tex

\chapter{Introdução}\label{cha:intro}

% \todo[inline]{IoT} 
A \iot conecta globalmente variados dispositivos, incluindo dispositivos móveis,
\emph{wearables}, eletrônicos domésticos, automóveis e sensores industriais.
Estes dispositivos podem, através da Internet, ser acessados, conectar-se a
outros dispositivos, servidores ou aplicações, tudo com mínima intervenção ou
supervisão humana
\cite{Tahsien2020,abane2019,haddadpajouh2019survey,Shanbhag2015}.
Outra característica de dispositivos \iot são os recursos computacionais
dimensionados para propósitos específicos, que limitam a capacidade de computar
outras funções muito além da função original do dispositivo.

% O número de dispositivos categorizados como \iot na última década teve
% crescimento sem precedentes e, proporcionalmente, cresceu o volume de dados
% gerados por esses dispositivos.
% % 
% A análise desses dados pode trazer novos conhecimentos e tem sido um tema
% frequentemente abordado por trabalhos de pesquisa.

% \todo[inline]{Segurança} 
Segurança e privacidade são uma grande preocupação em \iot, especialmente em
relação aos dados pessoais como localização e saúde aos quais dispositivos podem ter
acesso \cite{sengupta2020comprehensive}.
Além dos dados de sensores e atuadores que esses dispositivos gerenciam, se
esses dispositivos forem subvertidos podem gerar tráfego maligno, como o
% \notaPA{Explicar brevemente o que foi. Poderia ser até uma nota de rodapé. 
% Não se deve assumir que o leitor conheça ou tenha que buscar.}
% produzido pela \hlpa{\emph{mirai botnet} em 2016}
produzido pela \emph{mirai botnet} em 2016, onde em um dos ataques de \acf{DDoS}
$50\,000$ endereços únicos, de 164 países, formaram um pico de tráfego de $280$ Gbps
% 280 Gbps attributed to about 50 thousand unique IPs dispersed in 164 countries
% \notaPA{Comentar desta e outras situações semelhantes no texto.
%  (author: ???)
% }
\cite{Kambourakis2017,Kolias2017mirai}.
Nesse cenário, fatores que podem favorecer a subversão dos dispositivos incluem
a falta de controle sobre a origem do hardware e software embarcado nos
dispositivos, bem como a menor frequência de atualizações deste software.
Além disso, estes dispositivos têm longa vida e, após implantação, convivem com
ampla diversidade de outros dispositivos, o que torna complexa a manutenção da rede
que os hospeda, aumentando sua superfície de ataque.

% \todo[inline]{NIDS} 
No contexto de segurança de redes \iot, ferramentas que facilitem a detecção e
resposta a ataques são necessárias.
Como a maioria dos dispositivos \iot tem recursos limitados (como energia,
processamento, memória e comunicação), técnicas de segurança tradicionais
baseadas em algoritmos configuráveis não são usuais, restando as
técnicas de observação de rede \cite{Zhou2017}.
Ferramentas como \nids observam o comportamento da rede e de seus dispositivos
e detectam possíveis ataques.
% 
% Analogamente, as mesmas técnicas de classificação podem ser aplicadas para os
% metadados gerados pela comunicação entre esses nós e a Internet, detectando
% alterações nos padrões de comunicação num \nids.

% \todo[inline]{ML} 
Para implementação de \nids, técnicas de \ml têm sido empregadas na detecção de
ataques a partir de características de ataques conhecidos ou na
descoberta de novos ataques o mais cedo possível
\cite{buczak2016survey,mitchell2014survey}.
Apesar do uso promissor de \ml para segurança para sistemas \iot, muitos estudos
na literatura \cite{buczak2016survey,mitchell2014survey,Tahsien2020} são
limitados a métodos tradicionais de \ml.
% \notaPA{Tricorder ``Estes métodos [...] evolução de ataques''}
Estes métodos comumente utilizam modelos estáticos, ou com atualização manual,
para descrever e prever o comportamento da rede, que não mantêm a confiabilidade
frente à evolução de ataques \cite{Viegas2019,AndreoniLopez2019}.

% confiabilidade ou eficiência ??
% \todo[inline]{DS} 
% \notaPA{Bem difícil de ler essa frase.}
Além das complicações de confiabilidade, grande quantidade de dispositivos,
redes distantes, geração de dados em volumes e velocidades elevadas, as técnicas
tradicionais, que tratam grandes lotes em \emph{datacenters}, não são aplicáveis.
Para esses fluxos contínuos de dados (\emph{Data Stream}), técnicas de
mineração de fluxos de dados (\emph{Data Stream Mining}) entre outras que tratam
são promissoras \cite{AriyaluranHabeeb2019,Faria2016ndds,Akbar2017}.
% \notaPA{Faltam referências para embasar a afirmação.}
% 
Nesses cenários, essas técnicas são aplicadas, por exemplo, em problemas de
monitoramento e classificação de valores originários de sensores para tomada de
decisão tanto em nível micro, como na modificação de atuadores remotos, ou
macro, na otimização de processos industriais.
% 
% Quando aplicado a \nids, técnicas de mineração de fluxo de dados possuem
% vantagens como processamento em uma única leitura, reduzindo uso de memória e
% armazenamento possibilitando sua implementação em

% % processing traffic data with a single read;
% % working with limited memory (allowing the implementation in small devices
% % commonly employed in edge services);
% Produção de resultados 
% producing real-time response; and
% detecting novelty and changes in concepts already learned.

% \todo[inline]{ND} 
% \notaPA{Damicore - Tricorder}
Dentre as técnicas de mineração de fluxo de dados, classificadores podem ser
utilizados para detectar padrões conhecidos e, em conjunto com algoritmos de
\nd (\emph{Novelty Detection in Data Streams}), detectar novos padrões.
Um destes algoritmos de \nd é o \minas \cite{Faria2013Minas}.
Essa capacidade de detectar novos padrões é relevante para \nids, onde novidades
na rede podem representar novas funcionalidades ou ataques por agentes
maliciosos, sem assinaturas existentes em bancos de dados de ataques conhecidos.
Outras características que fazem \nd atraente para \nids são a produção de
respostas imediatas e a detecção de novidades e mudança de conceitos já
conhecidos.
Neste sentido, uma avaliação do algoritmo \minas como \nids foi feita por
\citeonline{Cassales2019} utilizando o conjunto de dados \emph{Kyoto 2006+},
composto de dados coletados de 348 \emph{Honeypots} pela Universidade de Kyoto
\cite{KyotoDataset}.

% \todo[inline]{Fog} 
Análises como mineração de fluxos de dados e \nd têm sido implementadas sobre
o paradigma de computação na nuvem (\emph{Cloud Computing}) e, recentemente,
também sobre paradigmas como computação em névoa (\emph{Fog Computing}).
Para névoa, além dos recursos em nuvem, são explorados os recursos
espalhados de nós remotos até a nuvem.
% \notaPA{Tricorder ``consumo computacional''}
Processos que implementam este tipo de análise em névoa fazem uso desses
recursos de acordo com características como sensibilidade à latência,
privacidade, consumo de recursos computacionais ou consumo energético.

De maneira geral, a aplicação de \nd para detecção de ameaças em fluxos de dados
originários de redes \iot dentro de \nids tem sido um ponto de interesse
\cite{Viegas2019,AndreoniLopez2019,DaCosta2019a}.
Este trabalho explora as características de implementação destas técnicas
em conjunto, concentrando-se em serviços localizados na borda da rede, de maneira
distribuída, para uso em ambientes \iot.

% Given the recent \cite{Viegas2019,AndreoniLopez2019,DaCosta2019a} use of Data Stream Novelty
% Detection (\nd) in network data streams, this paper shows the effects of
% adapting these mechanisms to edge services for use in \iot environments.

% \todo[inline]{FALTA na INTRODUCAO 1 OU 2 PARAGRAFOS EXPLICANDO O QUE VC FEZ/ E
% QUAL É UM RESUMO DOS RESULTADOS, SO PRA SETAR EXPECTATIVA DO LEITO}

% \notaPA{Introduzir antes de falar que vai usar.}
Este trabalho apresenta a construção e avaliação do \mfog\footnote{Disponível em
\url{https://github.com/luis-puhl/minas-flink}.}, uma implementação paralela e
distribuída em névoa de dispositivos \iot do algoritmo \minas.
% \notaPA{Explicar antes o que é MiNAS!}
Esta implementação foi construída com o padrão \mpi buscando escalabilidade na
tarefa de processamento de fluxo de dados e economia dos recursos limitados
comumente encontrados em sistemas \iot, seguindo a arquitetura \arch
\cite{Cassales2019}.

A avaliação do \mfog é constituída de métricas de qualidade de classificação e
% \notake{?}
% métricas de escalabilidade, extraídas com \hlke{auxilio} do conjunto de dados
métricas de escalabilidade, ambas extraídas experimentalmente com o conjunto de dados
% \notaPA{Introduzir "o que é" o conjunto de dados Kyoto 2015.}
\emph{Kyoto dez 2015} relevante para \nids.
% 
% \notaPA{... métricas de qualidade de classificação e métricas de escalabilidade,
% extraídas com auxilio do conjunto de dados Kyoto 2015...
% O que tem nesse conjunto de dados?}
% 
As métricas de qualidade de classificação obtidas dos resultados mostraram
valores equivalentes à implementação de referência do algoritmo \minas e
% as métricas de \hlke{escalabilidade mostraram melhora em relação à implementação de
as métricas de escalabilidade mostraram melhora em relação à implementação de
% \notaPA{Mas o paralelismo é uma das motivações...}
% referência} porém com eficiência de \hlke{paralelismo abaixo do esperado}.
referência porém com eficiência de paralelismo abaixo do esperado.
% \notaPA{Acho que este deveria ser o foco de investigação do trabalho, a ser destacado desde o título.}
Este trabalho contribui com uma análise do algoritmo \minas com a ótica de
distribuição em névoa, mostrando benefícios e desafios deste tipo de aplicação,
iluminando os detalhes do problema abordado e apontando algumas soluções para
trabalhos futuros.

\section{Motivação}\label{sec:motivo}

% A pesquisa recente de \citeonline{Tahsien2020} mostra que técnicas de \ml são
% uma alternativa promissora que pode prover ferramentas de segurança para redes
% \iot, fazendo elas mais confiáveis e acessíveis.
% A recent survey \cite{Tahsien2020} shows that ML based methods are a
% promising alternative which can provide potential security tools for the \iot
% network making them more reliable and accessible than before.

Um problema recente que une, em um único contexto, os métodos de computação em
névoa, processamento de fluxo de dados e detecção de novidades nesses fluxos é a
detecção de intrusão em redes de dispositivos \iot.
Para tratar esse problema, a arquitetura \arch, recentemente proposta por
\citeonline{Cassales2019}, aplica ao problema algoritmos relevantes
% atuais     % isso pode ser afirmado? Hélio
do tema de detecção de
novidades em fluxos, executando esses algoritmos em ambiente próximo aos
dispositivos e avaliando-os quanto à detecção de intrusão.

Na arquitetura proposta, \citeonline{Cassales2019} avaliou os algoritmos
ECSMiner \cite{Masud2010ECSMiner}, AnyNovel \cite{Abdallah2016anynovel} e MINAS
\cite{Faria2016minas}, sendo que o último mostrou resultados promissores.
% A arquitetura proposta foi avaliada com o conjunto de dados (\emph{data set})
% \emph{Kyoto 2006+}, composto de dados coletados de 348 \emph{Honeypots}
% (máquinas isoladas equipadas com diversos softwares com vulnerabilidades
% conhecidas expostas à Internet com propósito de atrair ataques) de 2006 até
% dezembro 2015.
% O \emph{data set} \emph{Kyoto 2006+} contém 24 atributos, 3 etiquetas atribuídas por
% detectores de intrusão comerciais e uma etiqueta
% distinguindo o tráfego entre normal, ataque conhecido e ataque desconhecido
% \cite{Cassales2019}.

% \notahl{ Por que a necessidade paralelismo? O que falta que justifica essa
% avaliação? Tempo de resposta?\\
% Necessidade de tratar os dados próximo a onde são produzidos? Uso da cloud? ….}
% \notaPA{Mas o resultado do paralelismo ficou abaixo do esperado...  Como fica essa motivação?}
% Author: Nem toda motivação é escrita após um bom resultado.
% Author: Geralmente motiva-se à fazer algo antes de te-lo feito.
Contudo, o algoritmo \minas ainda não foi implementado e avaliado com paralelismo,
multi-processamento ou distribuição computacional, que são necessários para
tratar fluxos de dados 
% com grandes volumes e velocidades
em ambientes distribuídos, como em cenários \iot e névoa.

% \notafa{Por que isso é importante. Acho que convém ressaltar a importancia da sua
% proposta. O que o MINAS original não trata, em quais cenários ele apresenta um
% gargalo? como ele está dividido?}
% O tratamento de distribuição em ambiente \fog é essencial para aplicação deste
O tratamento de distribuição em ambiente névoa é essencial para aplicação deste
algoritmo ao problema de detecção de intrusão em redes \iot, pois esta aplicação
requer tempo de resposta mínimo e pequena comunicação entre nós distantes, como
aquelas comunicações entre borda e a nuvem.
% \notaPA{Por que a divisão em 03 partes pertence ao escopo da não implementação
% do multi-processamento e da distribuição?}
Ainda observando o algoritmo \minas, destacam-se suas três partes: treinamento,
classificação e detecção de novidades.
A classificação é o elemento central cujos resultados são utilizados
para a identificação de intrusões, enquanto a detecção de novidades fornece
atualização automática do modelo de classificação.

Ainda no contexto de \nd como método de detecção de intrusão, outras propostas
tratam do caso de fluxos com grandes volumes e velocidades, como é o caso de
\citeonline{Viegas2019}, que apresenta o \emph{BigFlow} no intuito de detectar
intrusão em redes do tipo \emph{10 Gigabit Ethernet}, que podem produzir um
volume considerável.
% atualmente impossível de ser processado em um único núcleo de processador
% (\emph{single-threaded}).
Essa implementação foi feita sobre uma plataforma distribuída processadora de
fluxos (\emph{Apache Flink}) executada em um cluster com até $10$ nós de trabalho,
cada um com $4$ núcleos de processamento, totalizando $40$ núcleos, para atingir
taxas de até $10.72$ Gbps.

% Além de apresentar uma implementação, \citeonline{Viegas2019} também apresenta o
% \emph{data set} \emph{MAWIFlow}. Esse conjunto é derivado do ponto de coleta
% F, localizado em um elo de comunicação entre o Japão e os EUA (\emph{Backbone})
% com capacidade de $1\ Gbps$, diariamente 15 minutos são capturados desde 2006
% sob supervisão de \citeonline{mawiSamplepointF} \cite{Fontugne2010}. O conjunto
% \emph{MAWIFlow} limita-se às coletas de 2016 ($7.9\ TB$) e estratificado para
% $1\%$ desse tamanho facilitando compartilhamento e avaliação por outros
% softwares. Esse conjunto contempla 158 atributos de nós e fluxos e etiquetado
% por \citeonline{Fontugne2010}.

% \acronym{Drift}{\emph{Concept Drift}, Deriva conceitual: variação temporal de um conceito conhecido}
% \acronym{Evolution}{\emph{Concept Evolution}, Conceitos emergentes: conceitos não  }

% O sistema \emph{BigFlow} é composto de dois estágios: extração de
% características (estatísticas de tráfego da rede) e aprendizado confiável de
% fluxo. O segundo estágio implementa um algoritmo de detecção de novidade
% utilizando classificadores já estabelecidos na biblioteca \emph{Massive Online Analysis framework} (MOA) \cite{MOA} com
% adição de um módulo de verificação que armazena valores classificados com baixa
% confiança para serem manualmente avaliados por um especialista \cite{Viegas2019}.
% A escolha dessa
% abordagem não é nova e visa tratar nuances do problema abordado como
% variação temporal de conceitos conhecidos (\emph{Concept Drift}) e
% conceitos emergentes (\emph{Concept Evolution})
% \cite{Faria2016ndds}.
% Essas nuances causam redução de acurácia durante a
% avaliação inicial dos algoritmos tradicionais e devidamente mitigada com a
% atualização constante do modelo. Esses problemas são amplamente abordados e
% tratados em outros algoritmos como o MINAS \cite{Faria2016minas}.

Os trabalhos de \citeonline{Cassales2019} e \citeonline{Viegas2019} abordam
detecção de intrusão em redes utilizando algoritmos de \nd, porém com
perspectivas diferentes.
% \label{lbl:paulo-draw-an-arrow}
O primeiro investiga \iot e processamento em névoa e baseia-se em um algoritmo
genérico de detecção de novidade, sem modificações que o adaptem para o ambiente
de névoa (recursos limitados, distribuídos e alta velocidade).
O segundo trabalho trata de \emph{backbones} e processamento em nuvem e
implementa o próprio algoritmo de detecção de novidade.
Essas diferenças deixam uma lacuna onde, de um lado, tem-se uma
arquitetura mais adequada para o ambiente de névoa com um algoritmo estado da arte de
detecção de novidades, porém sem paralelismo.
Do outro lado da lacuna, tem-se um sistema escalável de alto desempenho porém
limitado ao ambiente nuvem e com um algoritmo que não foi projetado
para os desafios de detecção de novidades.
% \todo[inline]{``menos adequado'' -> que não foi projetado ... }

% \nota{
% \\ deixar mais claro o contraste entre cassales e bigflow
% \\ abordar o **gap** no qual a minha pesquisa entra
% \\ mostrar que modelo bigflow não considera fog
% \\ arquiteutra distribuida em fog com minas e alto desempenho bigflow
% }

% \notaPA{Introduzir antes.}
% A proposta deste trabalho, aqui chamada \mfog, adapta a arquitetura \hlpa{\arch} \cite{Cassales2019}
A proposta deste trabalho, aqui chamada \mfog, adapta a arquitetura \arch
\cite{Cassales2019} empregando o algoritmo de \nd \minas \cite{Faria2016minas},
tornando-o capaz de ser executado em um sistema distribuído composto de pequenos
computadores com recursos limitados, alocados na borda da rede próximos dos
dispositivos \iot.
% \notaPA{O foco do trabalho deveria ser na avaliação de uma versão distribuída
% /paralela, não na implementação em si.}\hlpa{
Utilizando a nova implementação do algoritmo \minas, avalia-se experimentalmente
como a distribuição afeta a capacidade do sistema de detectar mudanças
(novidades) nos padrões de tráfego e o impacto na eficiência computacional.
% }
Por fim, algumas estratégias e políticas para configuração do sistema de
detecção de novidades em fluxo de dados são discutidas.

% \todo[inline]{Ler aqui --> sera que motivacao tá clara, vc tá implementando isso pq o
% MINAS é um algoritmo promissor, IoT e subversao de dispositivos pode ser
% perigoso, e tem lacuna de duas implementacoes ali que nao fazem distribuicao ou
% paralelismo, o que ajudaria na resposta mais rapida do algortimo, certo?} uhum.

% Our proposal, called \mfog, adapted the \arch architecture \cite{Cassales2019}
% using the \nd algorithm \minas \cite{Faria2013Minas,Faria2016minas}, making it
% suitable to run on a distributed system composed of small devices with limited
% resources on the edge of the network.
% Using our newer version of the \minas algorithm, we have experimentally
% evaluated how the distribution affects the capability to detect changes
% (novelty) in traffic patterns and its impact on the computational efficiency.
% Finally, some distribution strategies and policies for the data stream novelty
% detection system are discussed.

\section{Objetivos}\label{sec:objetivos}

Como estabelecido na \refsec{motivo}, a lacuna no estado da arte observada é a
% \notaPA{``a lacuna [...] contexto de \underline{computação distribuída em névoa}''
% \ref{lbl:paulo-draw-an-arrow}}
% ausência de \hlke{uma implementação de algoritmo de detecção de novidades no contexto
% de computação distribuída em névoa}.
ausência de uma implementação de algoritmo de detecção de novidades no contexto
de computação distribuída em névoa.
% \notaPA{Fiquei confuso. Entendi que tinha essa implementação... \ref{lbl:paulo-draw-an-arrow}}
Neste sentido, um algoritmo que trate adequadamente os desafios de fluxo de
dados contínuos (como volume e velocidade do fluxo, evolução e mudança de
conceito) e considere o ambiente de computação em névoa aplicada à detecção de
intrusão.
% Seguindo a comparação entre algoritmos desse gênero realizada por
% \citeonline{Cassales2019}, esta pesquisa escolheu investigar o algoritmo MINAS
% \cite{Faria2016minas} para receber o tratamento necessário para adequá-lo ao
% ambiente de névoa e para fluxos de grandes volumes e velocidades.

Portanto, seguindo os trabalhos do Grupo de Sistemas Distribuídos e Redes (GSDR)
% da Universidade Federal de São Carlos (UFSCar), \hlke{propõem-se a construção de}\notake{*} uma
da Universidade Federal de São Carlos (UFSCar), propõem-se a construção de uma
% \notaPA{A proposta é ter escalabilidade. Afirmou que não teve bom desempenho. Como fica isso?}
% Author: Como anotado antes, nem sempre tem-se os resultados ótimos.
% Author: Não acho legal mudar a narrativa para acomodar meus resultados.
aplicação que implemente o algoritmo \minas \cite{Faria2016minas} de maneira
escalável e distribuível para ambientes de computação em névoa, seguindo a
arquitetura \arch \cite{Cassales2019}.

% \notaPA{Como posto no trabalho, com o objetivo maior de implementar um versão
% distribuída, eu não vejo essa avaliação aqui como um objetivo, mas como
% metodologia.
% Ela, como escrito, serve para validar a proposta, ou não?}
% Author: Com a mudança do título acho que deixa claro que a avaliação é um objetivo principal.
% \notake{verificar e inserir no trabalho.*}
% Além disso, \hlke{propõem-se} a avaliação dessa implementação com experimentos
Além disso, propõem-se a avaliação dessa implementação com experimentos
baseados na literatura usando conjunto de dados públicos relevantes.
O resultado esperado é uma implementação compatível em qualidade de
classificação ao algoritmo \minas e passível de ser distribuída em um ambiente de
computação em névoa aplicado à detecção de intrusão.

Com foco no objetivo geral, alguns objetivos específicos são propostos:
% \notaPA{Não está igual ao objetivo geral?}
% Author: Enquanto a geral fala de arquitetura teórica, a específica fala de tecnologia aplicada.
implementar o algoritmo \minas de maneira distribuída sobre uma plataforma de
processamento distribuída de fluxos de dados;
arquitetar e implementar um mecanismo de execução e avaliação de qualidade e
desempenho para os ambientes escolhidos;
avaliar e comparar a qualidade de detecção de intrusão da nova implementação;
avaliar a qualidade de detecção de intrusão em ambiente distribuído conforme a
arquitetura \arch em ambiente de computação em névoa.

% There is a need for real-time stream processing, as data is arriving as
% continuous flows of events; for example, cars in motion emitting GPS signals;
% financial transactions; the interchange of signals between cellphone towers; web
% traffic including things like session tracking and understanding user behavior
% on websites; and measurements from industrial sensors.
% https://dzone.com/articles/streaming-in-spark-flink-and-kafka-1

% \todo[inline]{revisar para ver se os objetivos gerais e especificos estao
% alinhados com o que vc esperava. Vc consegue avaliar a qualidade? Nao tem
% objetivo especifico se repetindo?}

% \notaPA{Gostei muito desta seção explicando como a pesquisa foi feita.}
\section{Proposta Metodológica}

Para cumprir os objetivos citados na \refsec{objetivos}, foi identificada a necessidade
de um processo exploratório seguido de experimentação. Tal processo inclui a
revisão da literatura, tanto acadêmica quanto técnica, seguida da experimentação
através de implementação de aplicação e testes.

O foco da revisão da literatura acadêmica é em trabalhos que abordem
processamento de fluxos de dados, classificação de fluxo de dados, detecção de
% \notaPA{
%     % ``detecção de novidades em fluxo de dados''
%     Esse termo ficaria bom na Tricorder
%     % Author: Os comentários de tricorder do Paulo SLS sempre me confundem.
%     % Author: Mas acho que ele quer dizer que eu inventei o termo (maldade da parte dele, nada se inventa).
% }
novidades em fluxo de dados e processamento distribuído de fluxo de dados.
O objetivo da revisão foi o estabelecimento do estado da arte desses assuntos,
de forma que alguns desses trabalhos servissem para comparações e relacionamentos.
Além disso, desses trabalhos buscaram-se métricas de qualidade de classificação
(por exemplo, taxa de falso positivo e matriz de confusão) e métricas de
escalabilidade (como taxa de mensagens, número de nós e número de processadores).
% \notaPA{O que é escalabilidade vertical e horizontal?
% Vertical: aumentar recursos de um nó.
% Horizontal: aumentar os nós disponíveis.}

A revisão da literatura técnica foi focada em plataformas, ferramentas e técnicas
para realizar a implementação proposta.
Portanto, foram selecionadas plataformas de processamento distribuído de fluxos
contínuos de dados
e técnicas de aprendizado de máquina associadas a elas.
Dessa revisão também foram identificadas técnicas ou ferramentas necessárias
para extração das métricas de avaliação, bem como conjunto de dados
públicos relevantes para \nids.

Uma vez definidos o estado da arte, as ferramentas técnicas e os
conjunto de dados, o passo seguinte foi a experimentação.
Nesse passo, foi desenvolvida uma aplicação na plataforma escolhida que, com base no
algoritmo \minas \cite{Faria2016minas}, classifica e detecta novidades em fluxos
contínuos de dados.
Também nesse passo, a implementação foi validada comparando os resultados de
classificação obtidos com os resultados de classificação do algoritmo original
\minas.
Posteriormente, foram realizados experimentos com a implementação e variações em
cenários de distribuição em névoa, coletando as métricas de classificação e
escalabilidade.

Ao final, a aplicação, resultados, comparações e discussões foram organizados
para publicação em meios e formatos adequados, como repositórios técnicos,
eventos ou revistas acadêmicas.

\section{Organização do trabalho}

% revisar ao final...
O restante desse trabalho segue a estrutura:
Capítulo \ref{cha:related} enumera e discute trabalhos relacionados e estabelece
o estado da arte do tema detecção de novidade em fluxos de dados e seu
processamento;
Capítulo \ref{cha:fundamentos} aborda conceitos teóricos e técnicos que embasam
esse trabalho;
Capítulo \ref{cha:proposta} descreve a proposta de implementação, discute
resultados preliminares durante a escolha da plataforma e apresenta os desafios
encontrados durante o desenvolvimento do trabalho;
Capítulo \ref{cha:results} aborda os ambientes de teste, descreve experimentos
realizados e discute os resultados obtidos.
Capítulo \ref{cha:final} adiciona considerações gerais sobre o trabalho, seus
resultados e comentários para trabalhos futuros.

% \todo[inline]{finalizar organizacao do trabalho versao final}
