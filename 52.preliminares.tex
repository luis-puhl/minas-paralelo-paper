% !TeX root = ./00.ppgcc-2020.tex

\section{Resultados preliminares}\label{sec:resultados}

No desenvolvimento parcial desta pesquisa, algumas experimentações e algumas
ferramentas de teste já foram desenvolvidas.
Aspectos desses desenvolvimentos são descritos a seguir.
% obrigado hélio.

\subsection{Implementação com \python e \kafka}

A primeira implementação e avaliação do \mfog realizada foi construída sobre a
linguagem \python com o sistema de fila de mensagens \kafka e a respectiva
biblioteca de conexão.
A escolha desse conjunto para a implementação ocorreu \hlhl{devido à ampla}
disponibilidade de bibliotecas de aprendizagem de máquina no ecossistema
\python e, à simplicidade geral da linguagem.
Na implementação desenvolvida, o sistema \kafka recebe mensagens e as armazena
em tópicos distribuídos em partições replicadas em nós de um \cluster,
gerenciados por um nó mestre e suportados pelo serviço de gerenciamento de
configuração distribuída \emph{Apache ZooKeeper}.
A aplicação \emph{Python} consome eventos através da interface \emph{Consumer API},
que expõe a distribuição através da associação de um consumidor às partições
mantidas pelo \kafka.

Para essa implementação, havia a hipótese de que a distribuição de
mensagens gerenciada pelo \kafka
se estenderia a processos consumidores, efetivamente distribuindo o volume de
mensagens entre eles igualmente.
No entanto, a hipótese foi refutada nos experimentos realizados.
Os experimentos em questão foram compostos de 8 processos consumidores, um
processo produtor, uma instância \kafka com 8 partições em seu tópico principal
e uma instância \emph{Apache ZooKeeper} associada à instância \kafka.
% A hipótese era que, como o número de partições igualava o número de consumidores,
% cada consumidor associaria-se a uma partição, distribuindo os dados igualmente
% entre os consumidores para a paralelização a execução.
A hipótese foi refutada quando observou-se que o número de
mensagens consumidas por um dos 8 processos representava a maioria (mais de
80\%) do volume introduzido no sistema, o restante sendo distribuído entre
outros 3 processos e o restante dos processos não recebia nenhuma mensagem.
Portanto, a iniciativa de implementar o algoritmo MINAS em \python com \kafka e
atingir os objetivos de distribuição falhou, o que levou à reconsideração das
plataformas escolhidas.

\subsection{Implementação com \flink}

% \nota{citar ferramentas e a escolha só depois do python e kafka}
% \nota{entre flink e spark, outro grupo de pesquisa já está explorando spark}

A segunda alternativa explorada teve por inspiração o trabalho de
\citeonline{Viegas2019} e, como outro grupo de pesquisa já estava explorando
o algoritmo na plataforma \emph{Apache Spark}, a segunda implementação
foi baseada na plataforma \flink.

A plataforma \flink tem modelos de processamento tanto de fluxos como em lotes.
O modelo em lotes é implementado como extensão do modelo de fluxos e, apesar
de não ser foco desse trabalho, mostrou-se útil para a construção do \offline,
já que o conjunto consumido por esse módulo é limitado.

Um desafio encontrado durante o desenvolvimento da implementação do \mfog foi a falta
de bibliotecas na plataforma \flink que disponibilizem versões adaptadas
à plataforma de algoritmos base para o algoritmo MINAS.
Em especial, a ausência dos algoritmos \emph{K-means} e \emph{CluStream}
gerou carga imprevista sobre o processo de desenvolvimento
resultando no atraso do processo de desenvolvimento.

Esta implementação segue a arquitetura descrita na \refsec{descricao} e as
avaliações e resultados esperados descritos neste \refcap{proposta}
referem-se à implementação do \mfog na plataforma \flink.
