\documentclass[aspectratio=43,10pt]{beamer}

\usetheme[progressbar=frametitle]{metropolis}
\usepackage{appendixnumberbeamer}
\usepackage{booktabs}
% \usepackage[scale=2]{ccicons}
\usepackage{pgfplots}
\usepgfplotslibrary{dateplot}
\usepackage{xspace}
\usepackage[english,main=brazilian]{babel}
\usepackage[utf8x]{inputenc}
\usepackage[alf]{abntex2cite}
\usepackage{multirow}
\usepackage{ragged2e}

\title[Experimental Design]{Experimentos - \textit{Experimental Design}}
\subtitle{Seminários de Metodologia Científica}
\author{Luís Henrique Puhl de Souza}
\institute{
Universidade Federal de São Carlos \\
Centro de Ciências Exatas e de Tecnologia \\
Departamento de Computação \\
Programa de Pós-Graduação em Ciência da Computação}
% \date{\today}
\date{Outubro 2018}
% \titlegraphic{\hfill\includegraphics[height=1.5cm]{logo.pdf}}

\begin{document}

\maketitle

\begin{frame}{Índice}
  \setbeamertemplate{section in toc}[sections numbered]
  \tableofcontents[hideallsubsections]
\end{frame}

\section{Introdução}
\begin{frame}[fragile]{Introdução}
    Dentro da rotina de experimentos, onde preocupa-se em relacionar fatores (variáveis),
    o \textbf{planejamento de experimentos} (\textit{Experimental Design} ou \textit{Design of Experiments -- DoE})
    é o momento dedicado a formular métricas e organizar a sua extração de maneira a
    estatisticamente garantir a modificação das variáveis de interesse e minimizar a interferência 
    de todas as outras variáveis.
\end{frame}

% sugestão da carol para layout
% contexto
% definição
% componentes
% aplicação (artigo)

\begin{frame}[fragile]{Histórico}
    \citeonline{montgomery2008} descreve as 4 eras do desenho experimental estatístico:

    \begin{alertblock}{Agrícola}
        \citeonline{fisher1935} apresenta os princípios de \textbf{aleatoriedade}, \textbf{blocos} e \textbf{replicação}
        e introduz entre outros  \textbf{desenho de conceito fatorial} e análise de variança (\textbf{ANOVA}).
    \end{alertblock}
    \begin{alertblock}{Industrial}
        \citeonline{box1951} aponta as características de experimentos industriais e propõe: 
        \textbf{imediatismo} e \textbf{sequencialidade}.
    \end{alertblock}
    \begin{alertblock}{Design de experimento ótimo}
        \citeonline{KieferJ;WolfowitzJ1959} apresentam esse método \textbf{formal} de selecionar um desenho ótimo.
        Não difundidos por limitações computacionais.
    \end{alertblock}
    \begin{alertblock}{Design robusto de parâmetros}
        Introduzido por \citeonline{taguchi1979introduction} e extremamente popular (aplicado \textbf{industrialmente}) 
        e controverso (inicialmente não referendado).
    \end{alertblock}
    
\end{frame}

\begin{frame}{Por que adotar o design de experimentos?}
   Segundo \citeonline{Coleman1993} não adotar métodos estatísticos em experimentos pode causar:
   \begin{itemize}
       \item Pressupostos injustificáveis;
       \item Combinações e controles indesejáveis de variáveis;
       \item Design excessivamente grandes ou pequenos;
       \item Precisão de medição inadequada;
       \item Erros de predição inaceitáveis;
       \item Ordem de execução indesejada;
       \item Não entendimento dos efeitos de uma interação;
       \item Identificação inadequada de fatores;
       \item Propagação de erros para futuros experimentos.
   \end{itemize}
\end{frame}

\section{Referencial Teórico}

\begin{frame}[allowframebreaks]{Terminologia de experimentos}
    \metroset{block=fill}
    \begin{exampleblock}{Fatores}
        ou entradas, podem ser mensuráveis, controláveis e/ou influenciar as variáveis de resposta.
    \end{exampleblock}
    \begin{exampleblock}{Níveis}
        configurações de cada fator no estudo, ou seja, diferentes
        valores que um determinado fator pode assumir.
    \end{exampleblock}
    \begin{exampleblock}{Variável de Resposta}
        saída do experimento.
    \end{exampleblock}
    \begin{exampleblock}{Placebo}
        Tratamento falso, frequentemente usado com o grupo de controle.
    \end{exampleblock}
    \begin{exampleblock}{Efeito Placebo}
        Quando a unidade experimental apresenta resultado porque acredita estar recebendo 
        o tratamento.
    \end{exampleblock}
    \begin{exampleblock}{Experimento cego:}
        É o experimento em que somente o pesquisador conhece quais 
        os tratamentos foram alocados às unidades experimentais
        ou parcelas. O avaliador desconhece essa informação.
    \end{exampleblock}
    \begin{exampleblock}{Experimento duplo-cego:}
        É o experimento em que o pesquisador e o avaliador desconhecem quais
        os tratamentos foram alocados às unidades experimentais ou parcelas.
    \end{exampleblock}
\end{frame}

\begin{frame}{Categorias de fatores em experimentos}
    \begin{figure}
      \centering
      \includegraphics[width=0.9\textwidth]{fatores.png}
      \caption{Diferentes categorias de fatores que afetam as variáveis de resposta. Fonte: \citeonline{Coleman1993}}
    \end{figure}
\end{frame}

\begin{frame}{Métodos de Fisher}
    Segundo \citeonline{fisher1935}, há três métodos que objetivam o isolamento de fatores
    não controlados:
    \metroset{block=fill}
    \begin{exampleblock}{Aleatoriedade}
        da ordem de aplicação do experimento (fatores desconhecidos).
    \end{exampleblock}
    \begin{exampleblock}{Blocos}
        homogêneos de sujeitos (fatores conhecidos).
    \end{exampleblock}
    \begin{exampleblock}{Replicação}
        em escala suficientemente grande (previne erros de medição).
    \end{exampleblock}
\end{frame}

\begin{frame}{Exemplo não científico}
    \begin{figure}
      \centering
      \includegraphics[width=0.8\textwidth]{exemplo.png}
      \caption{Exemplo ilustrando fatores, níveis e respostas. Fonte: \citeonline{Skrivanek2011}}
    \end{figure}
\end{frame}
\begin{frame}[fragile]{Processo de Experimento}
    \begin{figure}
      \centering
      \includegraphics[width=\textwidth]{refs/montgomery-1993-table.PNG}
      \caption{Orientações para planejamento de um experimento. Fonte: \citeonline{Coleman1993}.}
    \end{figure}
\end{frame}

%%%%%%%%%%%%%%%%%%%%%%%%%%%%%%%%%%%%%%%%%%%%%%%%%%%%%%%%%%%%%%%%%
\section{Modelos de design}

\begin{frame}{Completamente aleatorizado}
    Os tratamentos são distribuídos em \textbf{ordem aleatória}.
    O ambiente deve ser \textbf{homogêneo}. Segue os princípios da \textbf{repetição e aleatorização}.
    
    \begin{exampleblock}{Exemplo}
    Um engenheiro tem como objetivo verificar o desgaste de pneus de carros de diferentes marcas.
    Quatro marcas diferentes são testadas em quatro carros diferentes.
    
    Fichas de 1 a 16 são distribuídas para cada um dos pneus.
    A análise de variância é feita, então determina-se os desgastes de cada marca de pneu.
    \end{exampleblock}
      
\end{frame}

\begin{frame}{Conclusão do Artigo}
  \justifying Essa otimização pode reduzir o custo total do armazenamento sem adição ou troca de hardware.
  \vspace{0.5cm}

  Os resultados mostram que o DoE pode ser aplicado como método heurístico para determinar um aumento de desempenho apenas realizando ajustes em configurações.
\end{frame}

\section{Conclusão}

\begin{frame}{Conclusão}
    Esse tema é de extrema importância para muitos tipos de pesquisa por ser uma ferramenta
    formalizada. No entanto, o domínio das ferramentas estatísticas são uma barreira
    para maior adoção dessa metodologia.
    
    Em destaque a clara diferença que encontramos em nossas pesquisas entre material didático ou empírico (comum)
    e material referendado (artigos), onde o comum cita muito o DoE clássico de Fisher e os artigos 
    se utilizam de técnicas modernas altamente fundamentadas em estatística.
\end{frame}

\begin{frame}{Sumário}
    DoE foca em preparar o experimento considerado fatores e resultados de modo à minimizar fatores não controlados.
    
    Com isso define diferentes desenhos para tratar diferentes fatores e volumes de fatores.
\end{frame}

{\setbeamercolor{palette primary}{fg=black, bg=yellow}\begin{frame}[standout]
  Dúvidas?
\end{frame}}

\appendix

\begin{frame}[fragile]{Recomendações de Leitura}
   \begin{figure}
      \centering
      \includegraphics[height=0.75\textheight]{refs/Montgomery-cover.jpg}
      \caption{\textit{Design and Analysis of Experiments -- Eighth edition} de \citeonline{montgomery2008}.}
    \end{figure}
\end{frame}

\begin{frame}[allowframebreaks]{Referências}
  \bibliography{99.referencias.bib}
\end{frame}

\end{document}