\documentclass[aspectratio=43,10pt]{beamer}

\usetheme[progressbar=frametitle]{metropolis}
\usepackage{appendixnumberbeamer}
\usepackage{booktabs}
% \usepackage[scale=2]{ccicons}
\usepackage{pgfplots}
\usepgfplotslibrary{dateplot}
\usepackage{xspace}
\usepackage[english,main=brazilian]{babel}
\usepackage[utf8x]{inputenc}
\usepackage[alf]{abntex2cite}
\usepackage{multirow}
\usepackage{ragged2e}

\title[]{Uma Implementação Distribuída em Névoa do Algoritmo de Detecção de Novidade em Fluxos de Dados MINAS}
% \subtitle{Seminários de Metodologia Científica}
\author{Luís Henrique Puhl de Souza\\
Orientador: Prof. Dr. Hermes Senger}
\institute{
Universidade Federal de São Carlos \\
Centro de Ciências Exatas e de Tecnologia \\
Departamento de Computação \\
Programa de Pós-Graduação em Ciência da Computação}
% \date{\today}
\date{Fevereiro 2020}
% \titlegraphic{\hfill\includegraphics[height=1.5cm]{logo.pdf}}

\begin{document}

\maketitle

\begin{frame}{Índice}
  \setbeamertemplate{section in toc}[sections numbered]
  \tableofcontents[hideallsubsections]
\end{frame}

\section{Introdução}

\begin{frame}[fragile]{Introdução}
    
    fire running nearest trap hello it atmosphere area at no neighbor hand
    typical cream distant parallel difficulty touch safe fully suppose
    impossible guide lucky
    
\end{frame}


\section{Fundamentos}
\section{Trabalhos Relacionados}
\section{Proposta}
\section{Resultados Preliminares}
\section{Considerações Finais}

\begin{frame}{Conclusão}
    
    fire running nearest trap hello it atmosphere area at no neighbor hand
    typical cream distant parallel difficulty touch safe fully suppose
    impossible guide lucky
    
\end{frame}

{\setbeamercolor{palette primary}{fg=black, bg=yellow}\begin{frame}[standout]
  Obrigado!
\end{frame}}

\appendix

\begin{frame}[fragile]{Recomendações de Leitura}
   lorem ipsum
\end{frame}

\begin{frame}[allowframebreaks]{Referências}
  \bibliography{99.referencias.bib}
\end{frame}

\end{document}