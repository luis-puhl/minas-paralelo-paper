% !TeX root = ./00.main.tex
\chapter{Considerações Finais}\label{cha:final}

\begin{resumocap}
  Este Capítulo resume o trabalho realizado até agora e estabelece
  os próximos passos até sua completude.
\end{resumocap}

Este trabalho reúne conceitos de aprendizado de máquina com ênfase em detecção
de novidades em fluxos contínuos de dados e conceitos de processamento
distribuído de fluxos contínuos, com o objetivo de unir a lacuna no estado da
arte desses conceitos à luz de uma implementação e avaliação no cenário de
detecção de intrusão em redes de dispositivos da Internet das Coisas (\iot) em
ambiente de computação em névoa (\fog).

O objeto central desse trabalho (\mfog) trata da implementação do algoritmo MINAS na
plataforma de processamento de fluxos \flink, em três módulos que podem ser
distribuídos em um ambiente de \fog.
Sua distribuição permite selecionar o nó que tem os recursos computacionais mais
adequados para cada tarefa.
\hlke{A avaliação do \mfog será feita por}
\hlke{meio de métricas de qualidade}
\hlke{de classificação e métricas de escalabilidade.}

% \nota{deixar claro que é futuro, voz singular, 3ª pessoa \\ metodologia do restante}

Dando continuidade a este trabalho, segue-se com o desenvolvimento da implementação objeto
(\mfog) bem como a contínua avaliação comparativa dos resultados
produzidos pelo \mfog com seu algoritmo base, MINAS.
Também será dada continuidade nos experimentos com os conjuntos de dados (\datasets)
diversos e configurações variadas de distribuição de processamento em \fog
extraindo desses experimentos as métricas previamente discutidas.

% \nota{configurações variadas de distribuição..}

% \nota{falar/resumir os resultados esperados}

% As métricas extraídas servirão, além de resultado final para este trabalho,
% para a construção de artigos usados para compra

% O resultado esperado ́e uma implementac ̧ ̃aocompat ́ıvel em qualidade de
% classificac ̧ ̃ao ao algoritmo MINAS e pass ́ıvel de ser distribu ́ıda emum
% ambiente de computac ̧ ̃ao em n ́evoa aplicado`a detecc ̧ ̃ao de intrus ̃ao

Dessa forma, o \mfog pode contribuir com adição de uma ferramenta para os interessados
em sistemas de detecção de intrusão de redes de \hlke{dispositivos \iot}
ou outros sistemas que tratam de fluxos contínuos que tradicionalmente sofrem
com os ônus de latência e largura de banda na comunicação entre borda e nuvem.
Além disso, o \mfog objetiva contribuir com a adição de uma implementação
distribuída de um algoritmo cujo modelo é estado da arte em detecção de
novidades em fluxos contínuos de dados.


% \section{Cronograma}\label{sec:crono}
% % cap 5: Cronograma até a defesa
% % (quando)

% Nesta Seção apresentam-se as etapas previstas e sua distribuição temporal até o
% final deste trabalho de pesquisa.

% % \begin{itemize}
% %   \item Enumerar métricas de qualidade de classificação e métricas de
% %     escalabilidade encontradas na literatura;
% %   \item Avaliar plataformas de processamento distribuído de fluxos como:
% %     \begin{itemize}
% %         \item \emph{Apache Kafka} com \emph{Python};
% %         \item \emph{Apache Kafka Streams};
% %         \item \emph{Apache Spark Streaming};
% %         \item \emph{Apache Storm};
% %         \item \emph{Apache Flink};
% %     \end{itemize}
% %   \item Implementar algoritmo MINAS sobre \emph{Apache Flink};
% %   \item Executar a implementação com \emph{data sets} públicos relevantes;
% %   \item Validar, por meio de comparação com o algoritmo original, se a
% %     implementação gera os mesmos resultados quanto a classificação;
% %   \item Extrair métricas de escalabilidade por meio de experimentação;
% %   \item Avaliar estratégias de distribuição do algoritmo em \emph{fog} como:
% %   \begin{itemize}
% %     \item Detecção de novidade nas pontas;
% %     \item Detecção de novidade na nuvem;
% %     \item Detecção de novidade nas pontas e em nuvem;
% %   \end{itemize}
% %     e seu o impacto na qualidade de classificação e volume computado.
% % \end{itemize}

% \begin{enumerate}[label=\Alph*)]
%   \item \label{task:Z} Exame de Qualificação;
%   \item \label{task:A} Desenvolvimento da aplicação;
%   \item \label{task:B} Validação da aplicação em contraste com a implementação
%   MINAS original:
%     \begin{itemize}
%       \item preparação e, se necessário, adaptação da implementação
%       original e \emph{data sets};
%       \item comparação e, se necessário, ajustes à implementação.
%     \end{itemize}
%   \item \label{task:C} Experimentos com \hlke{\datasets} e estratégias de 
%   \notake{Testes com Wilcoxon
%   outras técnicas inteligentes que poderiam ser aplicadas para comparar com a sua}
%   distribuição em \emph{fog};
%   \item \label{task:D} Submissão de artigos com resultados de (\ref{task:C}
%   \item \label{task:E} Defesa da Dissertação.
%   % \item \label{task:E} Comparação com outros trabalhos, envolvendo:
%   %   \begin{itemize}
%   %     \item preparação e, se necessário, adaptação dos programas e \emph{data sets};
%   %     \item comparação e publicação dos resultados.
%   %   \end{itemize}
% \end{enumerate}

% \noindent\begin{ganttchart}[
%   vgrid,
%   time slot format=isodate-yearmonth,
%   time slot unit=month,
%   expand chart=\textwidth,
%   inline,
%   title height = 1,
%   y unit title = 0.6cm,
%   y unit chart = 0.7cm,
%   bar height = .8,
%   bar left shift=.05,
%   bar right shift=-.05,
%   bar/.style={fill=blue!55, rounded corners=3pt}
% ]{2020-03}{2020-08}
%   \gantttitlecalendar{year, month} \\
%   % \ganttmilestone{(\ref{task:Z}}{2020-03-02} \\
%   \ganttbar{(\ref{task:Z}}{2020-03}{2020-03} \\
%   \ganttbar{(\ref{task:A}}{2020-03}{2020-04} \\
%   \ganttbar{(\ref{task:B}}{2020-03}{2020-04} \\
%   \ganttbar{(\ref{task:C}}{2020-03}{2020-06} \\
%   \ganttbar{(\ref{task:D}}{2020-05}{2020-08} \\
%   \ganttbar{(\ref{task:E}}{2020-06}{2020-08} \\
% \end{ganttchart}
