% !TeX root = ./00.ppgcc-2020.tex

\chapter{Conclusão}\label{cha:final}

% \begin{resumocap}
%   Este Capítulo resume o trabalho realizado até agora e estabelece
%   os próximos passos até sua completude.
% \end{resumocap}

% retomar motivacao aqui
Segurança e privacidade são uma grande preocupação em \acf{IoT}, seja para a
preservação do sigilo de dados pessoais ou para evitar a subversão do
dispositivo para uso em uma \emph{botnet}. Nesses  cenários,
o monitoramento da rede para detecção
de possíveis intrusões é uma tarefa necessária.
Atendendo a essa necessidade, um \acf{NIDS} é uma ferramenta que observa o
tráfego da rede e alerta para possíveis ataques.
Este alerta é gerado através de detecção de anomalias, e uma técnica para
detectar anomalias são algoritmos de \acf{ND-DS}. O algoritmo \minas é um
algoritmo de detecção de novidade que já foi avaliado e considerado promissor para aplicação em \nids.

Para um bom funcionamento de um \nids, é necessário que este responda rapidamente
aos padrões do fluxo da rede. Portanto, para aplicações em redes \iot o envio das
observações de uma rede remota para processamento em nuvem (\cloud) é inviável.
Uma alternativa viável à computação em nuvem neste caso é a computação em névoa
(\fog), que via a alocação de recursos computacionais para um local mais próximo
à fonte de dados, mesmo que de menor capacidade, oferece menor latência.
Porém, a necessidade de resposta rápida de um \nids e a natureza distribuída da
computação em névoa requer que as técnicas utilizadas estejam preparadas para
um cenário paralelo e distribuído, o que não acontece com o algoritmo \minas.

% cumpriu objetivo 1?? - propõem-se a construção de uma aplicação que implemente o
% algoritmo MINAS (FARIA; CARVALHO; GAMA, 2016) de maneira escalável e distribuível
% para ambientes de computação em névoa, seguindo a arquitetura
Este trabalho apresenta a construção e avaliação do \mfog, uma implementação
paralela e distribuída com foco na execução em névoa de dispositivos \iot do
algoritmo \minas, pautado em um caso de uso para \nids.
Esta implementação foi construída com \mpi buscando a escalabilidade na tarefa de
processamento de fluxo de dados e economia dos recursos limitados comumente
encontrados em sistemas \iot, seguindo a arquitetura \arch \cite{Cassales2019a}.

% cumpriu obj 2 ??- Além disso, propõem-se também a avaliação dessa implementação
% com experimentos baseados na literatura usando conjunto de dados públicos
% relevantes.
A avaliação do sistema construído utilizou o conjunto de dados (\dataset)
\emph{Kyoto 2015}, relevante para \nids por conter descritores de fluxo de uma
rede de \emph{Honeypots}.
A manipulação do \dataset utilizado, apesar de não focada em \nids, foi
suficiente para avaliar a implementação e comparar resultados.

% cumpriu obj? - implemen-tar o algoritmo MINAS de maneira distribuída sobre uma
% plataforma de processamento distribuída de fluxos de dados

% cumpriu obj?? - Arquitetar e implementar um mecanismo de execução e avaliação de
% qualidade e desempenho para os ambientes escolhidos
As métricas de qualidade de classificação escolhidas mostraram resultado
equivalente entre a nova implementação e a implementação de referência do
algoritmo \minas, seja na versão sequencial, paralela com 4 processadores ou
distribuída com 3 nós.
As métricas de escalabilidade no entanto não mostraram os resultados esperados,
mesmo com melhora significante comparada à implementação de referência executada
no ambiente de névoa \iot, não foi observada melhoria no tempo e latência
compatível com o aumento no número de nós no cluster, ou seja, a escalabilidade
esperada não se realizou.

% cumpriu?? Avaliar e comparar a qualidade de detecção de intrusão da nova
% implementação;

% cumpriu?? - Avaliar a qualidade de detec-ção de intrusão em ambiente distribuído
% conforme a arquitetura IDSA-IoT em ambientede computação em névoa.

% se cumptri objetivo como cumpriu e qual foi o resultado????

% chegou-se ao resultao esoerado? - O resultado esperadoé uma implementação
% compatível em qualidade de classificação ao algoritmo MINAS epassível de ser
% distribuída em um ambiente de computação em névoa aplicado à detecçãode
% intrusão.


% qual a conclusao geeral sobre sua ilmplementacao? ta distribuida? é escalave? eh
% comparavel com estado da arte pra pior/melhor/igual?
Para o problema \nids em redes \iot (ou detecção de comportamentos anômalos em
fluxos de dados de sensores \iot em geral) com base em \nd paralela e
distribuída em névoa (\fog), o equilíbrio dos atributos de escalabilidade e
qualidade de classificação ainda não foi alcançado.
Para os trabalhos futuros, espera-se que esta implementação e análise sirva de
inspiração para busca de algoritmos e sistemas que maximizam o aproveitamento do
\emph{hardware} e energia dedicados à esta tarefa.

% o que vc  poderia melhorar/??
Em termos mais práticos, sugere-se a utilização do algoritmo \emph{CluStream}
\cite{Aggarwal2003} ou seus derivados e a ideia de um Modelo de tamanho grande
porém fixo, maximizando o aproveitamento de memória com um grande conjunto de
\mclusters.
Seguindo esta sugestão, estes \mclusters podem ser definidos como polígonos para
melhor cobertura do espaço, o Modelo pode ser checado quanto à sobreposição
evitando ambiguidade e o Modelo organizado em índices para que a busca espacial
seja rápida e eficiente.
Acredita-se que estas características junto com as propriedades de paralelismo e
distribuição em névoa podem levar a resultados brilhantes.


% -------------------------------------------------------------

% Este trabalho reúne conceitos de aprendizado de máquina com ênfase em detecção
% de novidades em fluxos contínuos de dados e conceitos de processamento
% distribuído de fluxos contínuos, com o objetivo de unir a lacuna no estado da
% arte desses conceitos à luz de uma implementação e avaliação no cenário de
% detecção de intrusão em redes de dispositivos da Internet das Coisas (\iot) em
% ambiente de computação em névoa (\fog).

% O objeto central desse trabalho (\mfog) trata da implementação do algoritmo MINAS na
% plataforma de processamento de fluxos \flink, em três módulos que podem ser
% distribuídos em um ambiente de \fog.
% Sua distribuição permite selecionar o nó que tem os recursos computacionais mais
% adequados para cada tarefa.
% A avaliação do \mfog será feita por
% meio de métricas de qualidade
% de classificação e métricas de escalabilidade.

% Dando continuidade a este trabalho, segue-se com o desenvolvimento da implementação objeto
% (\mfog) bem como a contínua avaliação comparativa dos resultados
% produzidos pelo \mfog com seu algoritmo base, MINAS.
% Também será dada continuidade nos experimentos com os conjuntos de dados (\datasets)
% diversos e configurações variadas de distribuição de processamento em \fog
% extraindo desses experimentos as métricas previamente discutidas.

% Dessa forma, o \mfog pode contribuir com adição de uma ferramenta para os interessados
% em sistemas de detecção de intrusão de redes de dispositivos \iot
% ou outros sistemas que tratam de fluxos contínuos que tradicionalmente sofrem
% com os ônus de latência e largura de banda na comunicação entre borda e nuvem.
% Além disso, o \mfog objetiva contribuir com a adição de uma implementação
% distribuída de um algoritmo cujo modelo é estado da arte em detecção de
% novidades em fluxos contínuos de dados.

% \subsection{Conclusion}
% \label{sec:conclusion}

% % A \acf{ND-DS} pode ser um mecanismo útil para \acf{NIDS} em ambientes \acf{IoT}.
% % \nd em ambientes \iot também pode servir a outras tipos de monitoramento
% % contínuo, análisando do comportamento do sistema.
% % Com relação à enorme quantidade de dados que devem ser processados na análise de
% % fluxo para \nd, é melhor que esse processamento ocorra próximo da fonte desses
% % dados, evitando tráfego e latência na comunicação com a nuvem.
% % No entanto, outra característica de redes \iot é a capacidade de processamento
% % reduzida de tais dispositivos de borda.

% Data Stream Novelty Detection (\nd) can be a useful mechanism for Network
% Intrusion Detection (\nids) in IoT environments.
% It can also serve other related applications of \nd using continuous network or
% system behavior monitoring and analysis.
% Regarding the tremendous amount of data that must be processed in the flow
% analysis for \nd, it is relevant that this processing takes place at the edge of
% the network.
% However, one relevant shortcoming of the IoT, in this case, is the reduced
% processing capacity of such edge devices.

% In this sense, we have put together and evaluated a distributed architecture for
% performing \nd in network flows at the edge.
% Our proposal, \mfog is a distributed \nd implementation based on the \nd
% algorithm \minas.

% The main goal of this work is to observe the effects of our approach to a
% previously serial only algorithm, especially in regards to time and quality
% metrics.

% While there is some impact on the predictive metrics, this is not reflected on
% overall classification quality metrics indicating that distribution of \minas
% shows a negligible loss of accuracy.
% In regards to time and scale, our distributed executions was faster than the
% previous sequential implementation of \minas, but efficient data distribution
% was not achieved as the observed time with each added node remained constant.

% Overall, \mfog and the idea of using distributed flow classification and novelty
% detection while minimizing memory usage to fit in smaller devices at the edge of
% the network is a viable and promising solution.
% Further work include the investigation of other \nd algorithms, other clustering
% algorithms in \minas and analysis of varying load balancing strategies.
