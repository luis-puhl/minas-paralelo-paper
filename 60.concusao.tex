% !TeX root = ./00.main.tex
\chapter{Considerações Finais}\label{cha:final}

% \begin{resumocap}
%   Este Capítulo resume o trabalho realizado até agora e estabelece
%   os próximos passos até sua completude.
% \end{resumocap}

Este trabalho reúne conceitos de aprendizado de máquina com ênfase em detecção
de novidades em fluxos contínuos de dados e conceitos de processamento
distribuído de fluxos contínuos, com o objetivo de unir a lacuna no estado da
arte desses conceitos à luz de uma implementação e avaliação no cenário de
detecção de intrusão em redes de dispositivos da Internet das Coisas (\iot) em
ambiente de computação em névoa (\fog).

O objeto central desse trabalho (\mfog) trata da implementação do algoritmo MINAS na
plataforma de processamento de fluxos \flink, em três módulos que podem ser
distribuídos em um ambiente de \fog.
Sua distribuição permite selecionar o nó que tem os recursos computacionais mais
adequados para cada tarefa.
A avaliação do \mfog será feita por
meio de métricas de qualidade
de classificação e métricas de escalabilidade.

Dando continuidade a este trabalho, segue-se com o desenvolvimento da implementação objeto
(\mfog) bem como a contínua avaliação comparativa dos resultados
produzidos pelo \mfog com seu algoritmo base, MINAS.
Também será dada continuidade nos experimentos com os conjuntos de dados (\datasets)
diversos e configurações variadas de distribuição de processamento em \fog
extraindo desses experimentos as métricas previamente discutidas.

Dessa forma, o \mfog pode contribuir com adição de uma ferramenta para os interessados
em sistemas de detecção de intrusão de redes de dispositivos \iot
ou outros sistemas que tratam de fluxos contínuos que tradicionalmente sofrem
com os ônus de latência e largura de banda na comunicação entre borda e nuvem.
Além disso, o \mfog objetiva contribuir com a adição de uma implementação
distribuída de um algoritmo cujo modelo é estado da arte em detecção de
novidades em fluxos contínuos de dados.
