% !TeX root = ./00.ppgcc-2020.tex

\chapter{Considerações Finais}\label{cha:final}

% \begin{resumocap}
%   Este Capítulo resume o trabalho realizado até agora e estabelece
%   os próximos passos até sua completude.
% \end{resumocap}

Este trabalho reúne conceitos de aprendizado de máquina com ênfase em detecção
de novidades em fluxos contínuos de dados e conceitos de processamento
distribuído de fluxos contínuos, com o objetivo de unir a lacuna no estado da
arte desses conceitos à luz de uma implementação e avaliação no cenário de
detecção de intrusão em redes de dispositivos da Internet das Coisas (\iot) em
ambiente de computação em névoa (\fog).

O objeto central desse trabalho (\mfog) trata da implementação do algoritmo MINAS na
plataforma de processamento de fluxos \flink, em três módulos que podem ser
distribuídos em um ambiente de \fog.
Sua distribuição permite selecionar o nó que tem os recursos computacionais mais
adequados para cada tarefa.
A avaliação do \mfog será feita por
meio de métricas de qualidade
de classificação e métricas de escalabilidade.

Dando continuidade a este trabalho, segue-se com o desenvolvimento da implementação objeto
(\mfog) bem como a contínua avaliação comparativa dos resultados
produzidos pelo \mfog com seu algoritmo base, MINAS.
Também será dada continuidade nos experimentos com os conjuntos de dados (\datasets)
diversos e configurações variadas de distribuição de processamento em \fog
extraindo desses experimentos as métricas previamente discutidas.

Dessa forma, o \mfog pode contribuir com adição de uma ferramenta para os interessados
em sistemas de detecção de intrusão de redes de dispositivos \iot
ou outros sistemas que tratam de fluxos contínuos que tradicionalmente sofrem
com os ônus de latência e largura de banda na comunicação entre borda e nuvem.
Além disso, o \mfog objetiva contribuir com a adição de uma implementação
distribuída de um algoritmo cujo modelo é estado da arte em detecção de
novidades em fluxos contínuos de dados.

% \subsection{Conclusion}
% \label{sec:conclusion}

Data Stream Novelty Detection (\nd) can be a useful mechanism for Network
Intrusion Detection (\nids) in IoT environments.
It can also serve other related applications of \nd using continuous network or
system behavior monitoring and analysis.
Regarding the tremendous amount of data that must be processed in the flow
analysis for \nd, it is relevant that this processing takes place at the edge of
the network.
However, one relevant shortcoming of the IoT, in this case, is the reduced
processing capacity of such edge devices.

In this sense, we have put together and evaluated a distributed architecture for
performing \nd in network flows at the edge.
Our proposal, \mfog is a distributed \nd implementation based on the \nd
algorithm \minas.

The main goal of this work is to observe the effects of our approach to a
previously serial only algorithm, especially in regards to time and quality
metrics.

While there is some impact on the predictive metrics, this is not reflected on
overall classification quality metrics indicating that distribution of \minas
shows a negligible loss of accuracy.
In regards to time and scale, our distributed executions was faster than the
previous sequential implementation of \minas, but efficient data distribution
was not achieved as the observed time with each added node remained constant.

Overall, \mfog and the idea of using distributed flow classification and novelty
detection while minimizing memory usage to fit in smaller devices at the edge of
the network is a viable and promising solution.
Further work include the investigation of other \nd algorithms, other clustering
algorithms in \minas and analysis of varying load balancing strategies.
