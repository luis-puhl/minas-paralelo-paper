\chapter{Implementação e testes}
\label{cha:imp}
% cap 4: Implementação e testes
%     4.1. descrição da implementação
%         - offline, online, ND, Clustering
%         - observação de paralelização
%         - complexidade bigO (?)
%     4.2. cenário de teste 
%         - detecção de intrusão
%         - Arquitetura guilherme (dispositivos pequenos vs cloud)
%     4.3. Resultados de experimentos
%         - gráficos, tempos, tabelas...
%         - análises e comentários


% % hélio
% notas sobre a distribuição do modelo
% - cenário com o processamento de novidades local
% - cenário com ND na núvem

% - minas totalmente active

% helio amanha:
% a. Cenário, b. Problema, c. Proposta e d. Resultado Esperado.
% hermes:
% a. Monitoramento, classificação,
% b. detecção de novidades,
% c. executar em nós multi-core de maneira escalável,
% d. minas com mesma qualidade porém escalável.

% helio:
% Técnicas de firewall tradicional: allow all, deny all.
% Firewall moderno usa modelos.
% Minas como modelo.

% hermes
% iot, nós expostos vitima de ataque, novas funcionalidades ->
% firewall rigido vs modelos -> streams ->
% novidade -> minas -> big data -> discussão de localidade de dados
% paralelização

% motivação: testes preliminares do guilherme que fez teste com tais bases
% teve resultados promissores

% - implementar
% - paralelizar
% - avaliar
% - avaliar aprendizado local ou global trade-offs


% % luis
% O minas abre espaço 

% % hermes
% escalabilidade do algoritmo

% Kafka particiona e expoem esse particionamento ao consumidor.
% Tentei usar Python + Kafka, mas não escalou.

% Detalhar a implementação

% % hermes, esqueça kafka, foque em arquivos


\section{Descrição da Implementação}
% \section{Objetivos a serem alcançados}
% O que de fato vc irá fazer 

- offline, online, ND, Clustering

- observação/Considerações de paralelização

Notas sobre implementação Python/Kafka/Minas (não escala como esperado)

Dificuldade no processamento distribuido em Flink.

- complexidade bigO (?)

\section{Cenário de Teste}

Para testar e demonstrar essa implementação um cenário de aplicação é construído
onde seria vantajoso distribuir o processamento segundo o modelo \emph{fog}. Alguns
cenários de exemplo são
casos onde deve-se tomar ação caso uma classe ou anomalia seja detectada


% In time, process control and automation of industrial facilities, ranging from
% oil refineries to corn flakes factories \cite{Stonebraker2005}

- detecção de intrusão
- Arquitetura guilherme (dispositivos pequenos vs cloud)
\cite{Cassales2019a}
% Descrever a arquitetura IDS-IoT do paper do Guilherme

- BigFlow com dataset atual e maior
dataset kdd99
% dataset kyoto não está disponível http://www.takakura.com/Kyoto_data/

\section{Experimentos e Resultados}
    - gráficos, tempos, tabelas...
    - análises e comentários


Mostrar alguma implementação já feita e que esteja funcionando minimamente

Mostrar resultados mesmo que sejam bem simples e básicos,
apenas para demonstrar que vc domina o ambiente e as ferramentas e
que está apto a avançar no trabalho 

% discussão de 2020-02-01
passos feitos/a fazer
1. Entender Minas
2. Analisar/descrever dataset KDD
3. Notas sobre implementação Python/Kafka/Minas (não escala como esperado)
4. BigFlow (dataset mais novo, usa flink)
5. Plataforma Flink (processamento distribuído)
Proposta
6. Implementar minas em Scala/Flink
7. Testar com datasets KDD e BigFlow
8. Validar/Comparar métricas com seus trabalhos correspondentes
%/discussão

% Notas lendo Quali Casssales:
- Descrição do hardware utilizado pode conter:
    - Arch, OS, Kernel,
    - CPU (core, thread, freq),
    - RAM (total/free size, freq),
    - Disk (total/free size, seq RW, rand RW),
    - Net IO between nodes (direct crossover, switched, wireless, to cloud) (bandwidth, latency).
essas métricas permitem relacionar trade-offs para as questões de fog: Processar em node, edge ou cloud?

Provavelmente vou retirar o kafka da jogada em node/edge, deixando apenas em cloud.
