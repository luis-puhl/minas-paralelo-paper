\chapter{Proposta de Pesquisa}

Descrever a arquitetura IDS-IoT do paper do Guilherme

\section{Objetivos a serem alcançados}

O que de fato vc irá fazer 

\section{Resultados preliminares}

Mostrar alguma implementação já feita e que esteja funcionando minimamente

Mostrar resultados mesmo que sejam bem simples e básicos, apenas para demonstrar que vc domina o ambiente e as ferramentas e que está apto a avançar no trabalho 

% discussão de 2020-02-01
passos feitos/a fazer
1. Entender Minas
2. Analisar/descrever dataset KDD
3. Notas sobre implementação Python/Kafka/Minas (não escala como esperado)
4. BigFlow (dataset mais novo, usa flink)
5. Plataforma Flink (processamento distribuído)
Proposta
6. Implementar minas em Scala/Flink
7. Testar com datasets KDD e BigFlow
8. Validar/Comparar métricas com seus trabalhos correspondentes

Considerações
- Notas sobre implementação Python/Kafka/Minas (não escala como esperado)
- BigFlow com dataset atual e maior
- Dificuldade no processamento distribuido em Flink.
%/discussão

% Notas lendo Quali Casssales:
- Descrição do hardware utilizado pode conter:
    - Arch, OS, Kernel,
    - CPU (core, thread, freq),
    - RAM (total/free size, freq),
    - Disk (total/free size, seq RW, rand RW),
    - Net IO between nodes (direct crossover, switched, wireless, to cloud) (bandwidth, latency).
essas métricas permitem relacionar trade-offs para as questões de fog: Processar em node, edge ou cloud?

Provavelmente vou retirar o kafka da jogada em node/edge, deixando apenas em cloud.
