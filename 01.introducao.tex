% cap 1:
%     - Objetivo paralelizar minas em plataforma de big-data capaz de consumir streams de forma eficiente
%     - Motivação: Minas é recente, com potencial em várias aplicações, por exemplo (NIDS, sensores, ...)
%         para isso deseja-se uma implementação eficiente (low power, ou usar todo hardware) e escalável (big-data)
% cap 2: fundamentos científicos e tecnológicos
%     (pege apenas os mais citados, siga a Elaine)
%     2.1. Computação em Núvem, Fog e Edge
%     2.2. Plataformas de processamento distribuído
%         - arq labmda, kappa, (vide guilherme)
%         - MapReduce, Haddop, Spark, Storm
%     2.3. Apache Flink
%     2.4. Mineração de Dados
%     2.5. Mineração de Stream
%         - quem são, o que consomem
%         (BigFlow apud Gaber2005) Mining data streams: A Review.
%     2.6. Novelty Detection
%     2.7. O algoritmo Minas
% cap 3: Trabalhos relacionados
%     - Artigos sobre o Minas
%     - outros que paralelizaram algoritmos de mineração de dados/streams alguns online (5-10 refs)
%     - implementação paralelas/distribuídas em dispositivos pequenos
% cap 4: Implementação e testes
%     4.1. descrição da implementação
%         - offline, online, ND, Clustering
%         - observação de paralelização
%         - complexidade bigO (?)
%     4.2. cenário de teste 
%         - detecção de intrusão
%         - Arquitetura guilherme (dispositivos pequenos vs cloud)
%     4.3. Resultados de experimentos
%         - gráficos, tempos, tabelas...
%         - análises e comentários
% cap 5: Cronograma até a defesa

\chapter{Introdução}

% Se

% ELEMENTOS QUE INTRO DEVE TER
% -contexto
% - problema
% - sua solucao
% - como vc vai fazer (metodologia/simples)
% - pq é importante (justificativa)
% - ja teve algum resultado preliminar (se sim, cita ele)
% - organizacao do trabalho( no capitulo 1 teremos... no capitulo 2 teremos ...)

% ------------------------------------------------------------------------------------------------------
% % hélio
% notas sobre a distribuição do modelo
% - cenário com o processamento de novidades local
% - cenário com ND na núvem

% - minas totalmente active

% helio amanha:
% a. Cenário, b. Problema, c. Proposta e d. Resultado Esperado.
% hermes:
% a. Monitoramento, classificação,
% b. detecção de novidades,
% c. executar em nós multi-core de maneira escalável,
% d. minas com mesma qualidade porém escalável.

% helio:
% Técnicas de firewall tradicional: allow all, deny all.
% Firewall moderno usa modelos.
% Minas como modelo.

% hermes
% iot, nós expostos vitima de ataque, novas funcionalidades ->
% firewall rigido vs modelos -> streams ->
% novidade -> minas -> big data -> discussão de localidade de dados
% paralelização

% motivação: testes preliminares do guilherme que fez teste com tais bases
% teve resultados promissores

% - implementar
% - paralelizar
% - avaliar
% - avaliar aprendizado local ou global trade-offs

% % luis
% O minas abre espaço para diferentes estratégias de paralelismo e distribuição.

% % hermes
% escalabilidade do algoritmo

% Kafka particiona e expõem esse particionamento ao consumidor.
% Tentei usar Python + Kafka, mas não escalou.

% Detalhar a implementação

% % hermes, esqueça kafka, foque em arquivos
% ------------------------------------------------------------------------------------------------------

% Hermes 2020-02-04
% Paralelize o minas no flink.
% (não se preocupe com o uso, seja ele NIDS ou qualquer outra coisa)
% Use a detecção de intrusão apenas como validação do algoritmo.

% cenário 2: NIDS é meu foco, esses são os desafios de NIDS, o minas resolve. mas esse cenário é tragico
% ------------------------------------------------------------------------------------------------------

% cap 1:
%   - Objetivo paralelizar minas em plataforma de big-data capaz de consumir streams de forma eficiente
%   - Motivação: Minas é recente, com potencial em várias aplicações, por exemplo (NIDS, sensores, ...)
%       para isso deseja-se uma implementação eficiente (low power, ou usar todo hardware) e escalável (big-data)
%/hermes
% ------------------------------------------------------------------------------------------------------

% a. Cenário: Monitoramento, classificação,
% b. Problema: detecção de novidades,
% c. Proposta: executar em nós multi-core de maneira escalável,
% d. Resultado Esperado: minas com mesma qualidade porém escalável.
% hermes
% iot, nós expostos vitima de ataque, novas funcionalidades ->
% firewall rigido vs modelos -> streams ->
% novidade -> minas -> big data -> discussão de localidade de dados
% paralelização
% motivação: testes preliminares do guilherme que fez teste com tais bases
% teve resultados promissores
% - implementar
% - paralelizar
% - avaliar
% - avaliar aprendizado local ou global trade-offs

A Internet das Coisas (\emph{Internet of Things} - IoT) é um tema frequentemente
abordado na última década claramente motivado crescimento sem precedentes do
número de dispositivos dessa categoria e respectivos dados gerados que podem
trazer novos conhecimentos através da análise desses dados. No entanto, além dos
dados de sensores e atuadores esses dispositivos, quando subvertidos, podem
gerar outro tipo de tráfego, maligno à sociedade, como o gerado pela
\emph{botnet} mirai em 2016 \cite{Kambourakis2017}. Nesse cenário são fatores
que possibilitaram esses ataques: falta de controle sobre a origem do hardware e
software embarcado nos dispositivos além das cruciais atualizações de segurança.

Com o cenário de milhares de dispositivos em redes distantes, complexa análise
sobre os dados gerados, tanto os diretamente ligados a sua função original como
os metadados produzidos como subproduto dessa função, o volume e velocidade com
o qual esses dados são gerados levam a ampla aplicação de técnicas de mineração
de fluxos de dados (\emph{Data Stream Mining}). Essas técnicas são aplicadas,
por exemplo, em problemas de monitoramento e classificação de valores
originários de sensores para tomada de decisão tanto em nível micro como
modificação de atuadores remotos ou macro na otimização de processos
industriais.

Essas técnicas envolvem mineração de dados (\emph{Data Mining}), aprendizado de
máquina (\emph{Machine Learning}) e recentemente Detecção de Novidades
(\emph{Novelty Detection} - ND) para, além de classificar em modelos conhecidos,
descobrir novos padrões e atuar coerentemente mesmo em face a valores nunca
vistos. Essa capacidade de reconhecer padrões nunca vistos é relevante em
especial para o processamento dos metadados para detecção de anomalias no
tráfego de rede podendo ser usado para distinguir novas funcionalidades
desenvolvidas e entregues aos dispositivos após sua implantação em campo de
ataques executados por agentes externos sem assinatura existente em bancos de
dados de ataques conhecidos.

Esse tipo de análise é tradicionalmente feita com o paradigma de
computação na nuvem (\emph{Cloud Computing} - \emph{cloud}), e recentemente
paradigmas como computação em névoa (\emph{Fog Computing}) tem proposto que, por
exemplo, partes do processo que são sensíveis à latência, sensíveis à
privacidade, de baixo consumo computacional (ou energético) ou que reduzem o
volume de dados enviados à \emph{cloud}.

Nesse contexto, uma arquitetura recente proposta por \citeonline{Cassales2019a}
baseada no algoritmo de detecção de novidades em fluxo de dados (ND em DS) MINAS
\cite{Faria2016minas} mostra resultados promissores. Contudo esse algoritmo
ainda não foi implementado e avaliado com paralelismo \emph{multi-core} ou
distribuído, seja no paradigma \emph{cloud} ou \emph{fog} deixando uma
oportunidade de contribuição técnica e acadêmica. Além disso outros algoritmos e
arquiteturas já receberam esse tratamento como é o caso de
\citeonline{Viegas2019} que apresenta \emph{BigFlow} no intuito de detectar
intrusão em redes \emph{10 Gigabit Ethernet} que é um volume considerável.

-----------------------------------------------

Os IDS foram tradicionalmente construídos à partir de técnicas de mineração de
dados (\emph{Data Mining} - DM), aprendizado de máquina (\emph{Machine Learning} - ML), mais
especificamente Detecção de Novidades (\emph{Novelty Detection} - ND) para detectar
% Guilherme: detecção de novidade (PERNER, 2007) (GAMA, 2010).
% PERNER2007 GAMA2010
ataques e descobrir novos padrões, porém ao analisar tráfego de rede a
velocidade da análise deve ser próxima ou superior à velocidade da rede
analisada além de não consumir mais recursos do que a própria rede analisada.
Mais restrições nesse sentido devem ser incorporadas quando trata-se de uma rede
IoT, diferente de uma rede operando em computação na nuvem (\emph{Cloud Computing} - cloud),
especialmente latência e banda são ainda mais restritos e ao mitigar
esses atributos movendo o IDS para o mais próximo da
rede IoT passando a processar na névoa computacional (\emph{Fog Computing} - fog)
armazenamento e processamento são também restringidos. Portanto uma única
leitura do conjunto analisado, rápida atualização e distribuição do modelo de
detecção e resultados em tempo real são características positivas encontradas
nas técnicas de mineração e processamento de fluxos de dados (\emph{Data Streams} -
DS).

Nesse contexto, foca-se na arquitetura de IDS proposta por [3] baseada no
algoritmo de ND em DS MINAS [4] e na implementação BigFlow [5] proposta para
redes 10 \emph{Gigabit Ethernet}. Acredita-se que a fusão dessas abordagens em uma nova
implementação seja capaz de tratar uma rede de maior fluxo com nível comparável
de precisão da análise com o mesmo hardware e maior capacidade de escalonamento
horizontal com distribuição de carga entre nós na Fog.

Ideias:

Necessidade de utilizar recursos da Fog para eliminar latência de comunicacao com a nuvem (enviar os dados para classificar na nuvem teria alta latencia)

Necessidade de processamento concorrente e distribuído para ter maior escalabilidade

\section{Objetivos}

% discussão de 2020-02-01
% nota og: Avaliar o apache flink e o Minas para detecção de intrusão em redes IoT.
Avaliar o algoritmo minas na plataforma Flink para detecção de intrusão em redes IoT.
% Avaliar = Testes com diferentes datasets (KDD e BigFlow) com as métricas Minas e BigFlow.

% O que o minas-flink tem de especial para detecção de intrusão?
% - conformidade com variação da rede

Objetivos secundários:
% visto como um passo-a-passo até o objetivo principal.
Identificar métricas na literatura:
    - Quanto a detecção de anomalias;
    - e desempenho na detecção de intrusão da literatura;
Implementar Minas sobre Apache Flink para detecção de intrusão em redes IoT;
% Demonstrar que Minas sobre Apache Flink (a implementação) é um alternativa 
% mudar verbo 'Mostrar' para algo mais concreto/fechado
Extrair as métricas de detecção de anomalias da implementação com datasets diferentes;
Validar a implementação comparando as métricas extraídas da implementação com as encontradas na literatura;
% teste ferramentas de stream (python/kafaka, scala/flink)
Extrair as métricas de desempenho da implementação com datasets diferentes;

\section{Motivação}
- flink não foi abordado para intrusões em redes iot;
- contribuição de técnica/método para a area de segurança da informação;
%/discussão

% (o que)
Implementar a arquitetura proposta em [3] (IDS-IoT) com foco no volume de fluxo
explorando aspectos de paralelismo e distribuição de recursos em fog e cloud visando
alta vazão e baixa latência.

Para redução da latência pretende-se utilizar recursos próximos à rede monitorada (fog) 
distribuindo o volume entre vários nós (com recursos muitas vezes limitados) para dar
vazão ao processamento. Além disso, para tarefas com maiores exigências de recursos de
processamento e armazenamento como treinamento e reconstrução do modelo pretende-se utilizar computação em nuvem.

Implementar o algoritmo MINAS [4] buscando máximo desempenho nos aspectos 
latência e vazão utilizando recursos de fog e nuvem além de avaliar seus resultados
qualitativos na detecção e classificação de novidades em \emph{datasets}
relevantes a detecção de intrusão em redes IoT.

% There is a need for real-time stream processing, as data is arriving as continuous flows of events;
% for example, cars in motion emitting GPS signals; financial transactions;
% the interchange of signals between cellphone towers; web traffic including things like session tracking
% and understanding user behavior on websites; and measurements from industrial sensors.
% https://dzone.com/articles/streaming-in-spark-flink-and-kafka-1

\section{Proposta Metodológica}
% (Metodologia) (como)

Implementar  -- e avaliar esta implementação -- o algoritmo MINAS [4] 
aplicando técnicas, como Apache Flink, usadas na solução BigFlow [5];
Aplicar essa implementação na arquitetura proposta por [3] e
comparar o desempenho com a implementação e conjunto de dados (dataset) original;
Avaliar o desempenho e escalabilidade da aplicação com dataset MAWIFlow.

Flink  - dizer que vai usar, para fazer o que e porque escolheu 

Kafka ?

Raspberry - para todos

A base - 

