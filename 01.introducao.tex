% !TeX root = ./00.main.tex
\chapter{Introdução}\label{cha:intro}

% ELEMENTOS QUE INTRO DEVE TER
% -contexto
% - problema
% - sua solucao
% - como vc vai fazer (metodologia/simples)
% - pq é importante (justificativa)
% - ja teve algum resultado preliminar (se sim, cita ele)
% - organizacao do trabalho( no capitulo 1 teremos... no capitulo 2 teremos ...)

% ------------------------------------------------------------------------------------------------------
% % hélio
% notas sobre a distribuição do modelo
% - cenário com o processamento de novidades local
% - cenário com ND na nuvem

% - minas totalmente active

% helio amanha:
% a. Cenário, b. Problema, c. Proposta e d. Resultado Esperado.
% hermes:
% a. Monitoramento, classificação,
% b. detecção de novidades,
% c. executar em nós multi-core de maneira escalável,
% d. minas com mesma qualidade porém escalável.

% helio:
% Técnicas de firewall tradicional: allow all, deny all.
% Firewall moderno usa modelos.
% Minas como modelo.

% hermes
% iot, nós expostos vitima de ataque, novas funcionalidades ->
% firewall rigido vs modelos -> streams ->
% novidade -> minas -> big data -> discussão de localidade de dados
% paralelização

% motivação: testes preliminares do guilherme que fez teste com tais bases
% teve resultados promissores

% - implementar
% - paralelizar
% - avaliar
% - avaliar aprendizado local ou global trade-offs

% % luis
% O minas abre espaço para diferentes estratégias de paralelismo e distribuição.

% % hermes
% escalabilidade do algoritmo

% Kafka particiona e expõem esse particionamento ao consumidor.
% Tentei usar Python + Kafka, mas não escalou.

% Detalhar a implementação

% % hermes, esqueça kafka, foque em arquivos
% ------------------------------------------------------------------------------------------------------

% Hermes 2020-02-04
% Paralelize o minas no flink.
% (não se preocupe com o uso, seja ele NIDS ou qualquer outra coisa)
% Use a detecção de intrusão apenas como validação do algoritmo.

% cenário 2: NIDS é meu foco, esses são os desafios de NIDS, o minas resolve. mas esse cenário é tragico
% ------------------------------------------------------------------------------------------------------

% cap 1:
%   - Objetivo paralelizar minas em plataforma de big-data capaz de consumir streams de forma eficiente
%   - Motivação: Minas é recente, com potencial em várias aplicações, por exemplo (NIDS, sensores, ...)
%       para isso deseja-se uma implementação eficiente (low power, ou usar todo hardware) e escalável (big-data)
%/hermes
% ------------------------------------------------------------------------------------------------------

% a. Cenário: Monitoramento, classificação,
% b. Problema: detecção de novidades,
% c. Proposta: executar em nós multi-core de maneira escalável,
% d. Resultado Esperado: minas com mesma qualidade porém escalável.
% 
% Linha de argumentação: iot, nós expostos vitima de ataque, novas funcionalidades ->
% firewall rigido vs modelos -> streams -> novidade -> minas ->
% big data -> discussão de localidade de dados e paralelização
% 
% motivação: testes preliminares do guilherme que fez teste com tais bases
% teve resultados promissores
% - implementar
% - paralelizar
% - avaliar
% - avaliar aprendizado local ou global trade-offs

% Se 3 palavras formam uma ideia atômica, tenha 2 ligações nas pontas
% 3 ou 4 linhas max, quebrar em sentenças
% muito Essa, Esse em começo de parágrafo

% \textit{A Internet das Coisas, tema frequentemente abordado nos últimos anos,
% atrelado ao crescimento sem precedentes do número de dispositivos conectados.
% Tais dispositivos captam informações através de sensores e outros meios de
% entrada e geram dados que podem trazer conhecimentos relevantes através de sua  
% análise.}
\acronym{IoT}{Internet of Things, Internet das Coisas}

A Internet das Coisas (\emph{Internet of Things} - IoT) é um sistema global de
dispositivos (máquinas, objetos físicos ou virtuais, sensores, atuadores e
pessoas) com capacidade de intercomunicação pela Internet sem depender de
interação com interface humano-computador tradicional.
O número de dispositivos categorizados como IoT na década passada teve
crescimento sem precedentes e, proporcionalmente, cresceu o volume de dados
gerados por esses dispositivos.

A análise desses dados pode trazer novos conhecimentos e foi um tema frequentemente
abordado por trabalhos de pesquisa na última década.
Além dos dados de sensores e atuadores esses dispositivos, quando subvertidos,
podem gerar outro tipo de tráfego, maligno à sociedade, como o gerado pela
\emph{botnet} mirai em 2016 \cite{Kambourakis2017}.
Nesse cenário são fatores que possibilitam a subversão desses dispositivos:
falta de controle sobre a origem do hardware e software embarcado nos
dispositivos além das cruciais atualizações de segurança.

% The Internet of Things (IoT) has been defined in Recommendation ITU-T Y.2060 (06/2012) as
% a global infrastructure for the information society, enabling advanced services
% by interconnecting (physical and virtual) things based on existing and evolving
% interoperable information and communication technologies.
\acronym{DS}{Data Stream, Fluxo de Dados}

Com milhares de dispositivos em redes distantes gerando dados (diretamente
ligados a sua função original e também metadados produzidos como subproduto) com
volume e velocidade consideráveis formando fluxos contínuos de dados (\emph{Data
Stream} - DS) , técnicas de mineração de fluxos de dados
(\emph{Data Stream Mining}) são amplamente necessárias.
Essas técnicas são
aplicadas, por exemplo, em problemas de monitoramento e classificação de valores
originários de sensores para tomada de decisão tanto em nível micro, como
modificação de atuadores remotos, ou macro, na otimização de processos
industriais.
Analogamente, as mesmas técnicas de classificação podem ser aplicadas para os
metadados gerados pela comunicação entre esses nós e a Internet num serviço de
detecção de intrusão.

\acronym{ND}{Novelty Detection}{Detecção de Novidade}
% \New

Técnicas de \emph{Data Stream Mining} envolvem mineração de dados
(\emph{Data Mining}), aprendizado de
máquina (\emph{Machine Learning}) e recentemente Detecção de Novidades
(\emph{Novelty Detection}, \gls{ND}).
ND além de classificar em modelos conhecidos
permite classificar novos padrões e, consequentemente, atuar corretamente mesmo
em face a padrões nunca vistos.
Essa capacidade é relevante em especial para o
exemplo de detecção de intrusão, onde novidades na rede podem distinguir novas
funcionalidades (entregues aos dispositivos após sua implantação em campo) de
ataques por agentes externos sem assinatura existente em bancos de
dados de ataques conhecidos.

% finalizo com "por exemplo"?

Análises como \emph{Data Stream Mining} e ND são tradicionalmente implementadas
sobre o paradigma de computação na nuvem
(\emph{Cloud Computing}) e, recentemente, paradigmas como computação em névoa
(\emph{Fog Computing}). Para \emph{fog}, além dos recursos em \emph{cloud}, são
explorados os recursos distribuídos pela rede desde o nó remoto até a
\emph{cloud}. Processos que dependem desses recursos são distribuídos de acordo
com características como sensibilidade à latência, privacidade,
consumo computacional ou energético.

\section{Motivação}\label{sec:motivo}

No contexto de detecção de novidades para fluxos de dados em \emph{fog}, uma
arquitetura recente proposta por \citeonline{Cassales2019a} mostra resultados promissores.

% baseada nos algoritmos de detecção de novidades em fluxo de dados ECSMiner, AnyNovel e MINAS \cite{Faria2016minas},
ECSMiner, AnyNovel e MINAS


A arquitetura proposta foi avaliada com conjunto de dados (\emph{data set}) \emph{Kyoto 2006+} 
composto de dados coletados de 348 \emph{Honeypots} (máquinas isoladas com diversos softwares
com vulnerabilidades conhecidas expostas à Internet com propósito de atrair
ataques) de 2006 até dezembro 2015.
O \emph{data set} \emph{Kyoto 2006+} contém 24 atributos, 3 etiquetas atribuídas por
detectores de intrusão comerciais e uma etiqueta
distinguindo o tráfego entre normal, ataque conhecido e ataque desconhecido
\cite{Cassales2019a}.
Contudo, o algoritmo MINAS ainda não foi implementado e avaliado com paralelismo
multi-processamento ou distribuído.

Outras propostas tratam do caso de grandes volumes e velocidades, como é o caso
de \citeonline{Viegas2019} que apresenta o \emph{BigFlow} no intuito de detectar
intrusão em redes \emph{10 Gigabit Ethernet}, que é um volume considerável
atualmente impossível de ser processado em um único núcleo de processador
(\emph{single-threaded}). Essa implementação é feita sobre uma plataforma
distribuída processadora de fluxos (\emph{Apache Flink}) executada em um cluster
com até 10 nós de trabalho, cada um com 4 núcleos de processamento totalizando
40 núcleos para atingir taxas de até $10,72 \ Gbps$.

% Além de apresentar uma implementação, \citeonline{Viegas2019} também apresenta o
% \emph{data set} \emph{MAWIFlow}. Esse conjunto é derivado do ponto de coleta
% F, localizado em um elo de comunicação entre o Japão e os EUA (\emph{Backbone})
% com capacidade de $1\ Gbps$, diariamente 15 minutos são capturados desde 2006
% sob supervisão de \citeonline{mawiSamplepointF} \cite{Fontugne2010}. O conjunto
% \emph{MAWIFlow} limita-se às coletas de 2016 ($7.9\ TB$) e estratificado para
% $1\%$ desse tamanho facilitando compartilhamento e avaliação por outros
% softwares. Esse conjunto contempla 158 atributos de nós e fluxos e etiquetado
% por \citeonline{Fontugne2010}.

% \acronym{Drift}{\emph{Concept Drift}, Deriva conceitual: variação temporal de um conceito conhecido}
% \acronym{Evolution}{\emph{Concept Evolution}, Conceitos emergentes: conceitos não  }

% O sistema \emph{BigFlow} é composto de dois estágios: extração de
% características (estatísticas de tráfego da rede) e aprendizado confiável de
% fluxo. O segundo estágio implementa um algoritmo de detecção de novidade
% utilizando classificadores já estabelecidos na biblioteca \emph{Massive Online Analysis framework} (MOA) \cite{MOA} com
% adição de um módulo de verificação que armazena valores classificados com baixa
% confiança para serem manualmente avaliados por um especialista \cite{Viegas2019}.
% A escolha dessa
% abordagem não é nova e visa tratar nuances do problema abordado como
% variação temporal de conceitos conhecidos (\emph{Concept Drift}) e
% conceitos emergentes (\emph{Concept Evolution})
% \cite{Faria2016nd}.
% Essas nuances causam redução de acurácia durante a
% avaliação inicial dos algoritmos tradicionais e devidamente mitigada com a
% atualização constante do modelo. Esses problemas são amplamente abordados e
% tratados em outros algoritmos como o MINAS \cite{Faria2016minas}.

Os trabalhos de \citeonline{Cassales2019a} e \citeonline{Viegas2019} abordam
detecção de intrusão em redes utilizando algoritmos de ND em DS porém com
perspectivas diferentes.
O primeiro observa \emph{IoT} e processamento em \emph{fog}, baseia-se em um
algoritmo genérico de detecção de novidade.
O segundo trabalho observa \emph{backbones} e processamento em \emph{cloud},
implementa o próprio algoritmo de detecção de novidade.
Essas diferenças deixam uma lacuna onde de um lado tem-se uma
arquitetura mais adequada para \emph{fog} com um algoritmo estado da arte de
detecção de novidades porém sem paralelismo e, de outro, tem-se um sistema
escalável de alto desempenho porém almejando outra arquitetura (\emph{cloud}) e
com um algoritmo menos preparado para os desafios de detecção de novidades.

% \nota{
% \\ deixar mais claro o contraste entre cassales e bigflow
% \\ abordar o **gap** no qual a minha pesquisa entra
% \\ mostrar que modelo bigflow não considera fog
% \\ arquiteutra distribuida em fog com minas e alto desempenho bigflow
% }

\section{Objetivos}\label{sec:objetivos}

Como estabelecido na \refsec{motivo}, a lacuna no estado da arte observada é
constituída da ausência de uma implementação de algoritmo de detecção de
novidade que considere o ambiente de computação em névoa aplicada à detecção de
intrusão.
Portanto, seguindo os trabalhos do Grupo de Sistemas Distribuídos e Redes
(GSDR), propõem-se então a expansão do tra


% Hermes
\nota{Citar que o MINAS já foi testado para detecção de intrusão [Guilherme2019].}

\nota{Falar do resultado bem resumuidamente}

\nota{Porem a implementação não distribuída.}


%Com a avaliação inicial formulou-se a questão \emph{``quais resultados podem ser esperados de um sistema que implementa um algoritmo de detecção de novidades em fluxo de dados?"} que engloba sucintamente
% os temas abordados nesse trabalho.

% c. Proposta: executar em nós multi-core de maneira escalável,
% d. Resultado Esperado: minas com mesma qualidade porém escalável.
% 
% Linha de argumentação: discussão de localidade de dados e paralelização
% - implementar
% - paralelizar
% - avaliar
% - avaliar aprendizado local ou global trade-offs

Propõem-se então a construção de uma aplicação que implemente o algoritmo MINAS
de maneira escalável e distribuível para ambientes de computação em névoa e a avaliação
dessa implementação com experimentos baseados na literatura e conjunto de dados
públicos relevantes.
O resultado esperado é uma implementação compatível em qualidade de
classificação ao algoritmo MINAS e passível de ser distribuído em um ambiente
de computação em névoa.

Com foco no objetivo geral, alguns objetivos secundários são propostos:

\nota{citar ferramentas e a escolha só depois do python e kafka}

\nota{entre flink e spark, outro grupo de pesquisa já está explorando spark}

\begin{itemize}
    \item Implementar o algoritmo MINAS de maneira distribuída sobre uma plataforma de processamento
    distribuída de fluxos de dados;
    
    \item Avaliar a qualidade de detecção de intrusão em ambiente distribuído conforme a arquitetura IDSa-IOT e os conjuntos de dados associados;
    
    \item Avaliar o desempenho da implementação em ambiente de computação em névoa.

    % \item Implementar algoritmo MINAS sobre \emph{Apache Flink};
    % \item Executar a implementação com \emph{data sets} públicos relevantes;
    % \item Validar, por meio de comparação com o algoritmo original, se a
    % implementação gera os mesmos resultados quanto a classificação;
    % \item Extrair métricas de escalabilidade por meio de experimentação;
    % \item Avaliar estratégias de distribuição 
    % (execução nas bordas, na nuvem e mista) do algoritmo em \emph{fog} 
    % e seu o impacto na qualidade de classificação e volume computado.
\end{itemize}

\nota{
estudar o algoritmo\\
fazer uma implementação\\
pegar datasets relevantes\\
comparar a corretude  com o sequencial\\
avaliar desempenho e escalabilidade
}

% There is a need for real-time stream processing, as data is arriving as
% continuous flows of events; for example, cars in motion emitting GPS signals;
% financial transactions; the interchange of signals between cellphone towers; web
% traffic including things like session tracking and understanding user behavior
% on websites; and measurements from industrial sensors.
% https://dzone.com/articles/streaming-in-spark-flink-and-kafka-1

\section{Proposta Metodológica}

% (Metodologia) (como)

Para cumprir os objetivos citados na \refsec{objetivos}, foi identificado a necessidade
de um processo exploratório seguido de experimentação. Tal processo inclui a
revisão da literatura, tanto acadêmica quanto técnica, seguida da experimentação
através de implementação de aplicação e testes.

O foco da revisão da literatura acadêmica é em trabalhos que abordem:
processamento de fluxos de dados, classificação de fluxo de dados, detecção de
novidades em fluxo de dados e processamento e distribuído de fluxo de dados.
O objetivo da revisão é o estabelecimento do estado da arte desses assuntos
e para que alguns desses trabalhos sirvam para comparações e relacionamentos.
Além disso, desses trabalhos extrai-se métricas de qualidade de classificação
(por exemplo taxa de falso positivo e matriz de confusão) e métricas de
escalabilidade (taxa de mensagens por segundo e escalabilidade vertical ou
horizontal).

A revisão da literatura técnica foca em plataformas, ferramentas e técnicas
para realizar a implementação proposta.
Portanto, serão selecionadas plataformas de processamento distribuído de DS
e técnicas de aprendizado de máquina associadas a elas.
Dessa revisão também são obtidas técnicas ou ferramentas necessárias
para extração das métricas de avaliação bem como \emph{data sets}
públicos relevantes para detecção de novidades em DS.

Uma vez definidos o estado da arte, as ferramentas técnicas e os
\emph{data sets}, o passo seguinte é a experimentação.
Nesse passo é desenvolvida uma aplicação na plataforma escolhida que, com base no
algoritmo MINAS \cite{Faria2016minas}, classifica e detecta novidades em DS.
Também nesse passo a implementação é validada comparando os resultados de
classificação obtidos com os resultados de classificação do algoritmo original
MINAS.
Posteriormente, serão realizados experimentos com a implementação e variações em \emph{data sets} e
cenários de distribuição em \emph{fog} coletando as métricas de classificação e escalabilidade.

Ao final, a aplicação, resultados, comparações e discussões serão publicados
nos meios e formatos adequados como repositórios técnicos, eventos ou revistas
acadêmicas.

\section{Organização do trabalho}


% Flink  - dizer que vai usar, para fazer o que e porque escolheu 

% Kafka ?

% Raspberry - para todos

% A base - 

% 1.2 (Cassales)
%  Metodologia
% A seguir será descrita a metodologia que pretende-se utilizar para atingir aos objetivos e
% responder às questões de pesquisa propostas.
% Primeiramente será realizado um levantamento do estado da arte no que diz respeito aos
% Sistemas de Detecção de Intrusão e às técnicas de Detecção de Novidade aplicadas aos Fluxos
% Contı́nuos de Dados. Com base neste levantamento, serão determinadas as lacunas dos sistemas
% da área e quais as técnicas mais adequadas para a utilização neste trabalho. Dado o foco nas
% técnicas e na arquitetura, é necessário que se possua mais de uma base de dados para realizar
% 1.2 Metodologia
%  13
% a avaliação. Tendo em vista que existem bases de dados de segurança que não representam
% o cenário atual de maneira fidedigna, é necessário selecionar bases que possibilitem que os
% resultados sejam confiáveis.
% Além das bases, serão utilizadas diversas técnicas, as quais deverão ser avaliadas e estuda-
% das para que possam ser propostas melhorias ou mesmo uma nova técnica. Finalmente, serão
% implementadas melhorias e, se necessário, uma nova técnica e será feita avaliação do desempe-
% nho desta técnica.

% \section{Organização do trabalho}

O restante desse trabalho segue a estrutura:
\refcap{fundamentos} aborda conceitos teóricos e técnicos que embasam
esse trabalho;
\refcap{related} enumera e discute trabalhos relacionados e estabelece
o estado da arte do tema detecção de novidade em fluxos de dados e seu processamento;
\refcap{proposta} descreve a proposta de implementação, discute
as escolhas de plataformas e resultados esperados.
Também são distidos no \refcap{proposta} os desafios e resultados preliminares encontrados
durante o desenvolvimento do trabalho;
\refcap{final} adiciona considerações gerais e apresenta o plano de trabalho
e cronograma até a defesa.
